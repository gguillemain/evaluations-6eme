%%!TEX TS-program = latex
\documentclass[12pt]{article}
\usepackage[utf8]{inputenc}
\usepackage[T1]{fontenc}
\usepackage[french]{babel}
\usepackage{amsmath,amssymb}
\usepackage{geometry}
\geometry{a4paper, margin=2cm}
\usepackage{pstricks,pstricks-add,pst-plot,pst-tree,pst-eps}
\usepackage{multicol}
\usepackage{graphicx}
\usepackage{enumitem}
\usepackage{amsfonts}
\usepackage{siunitx}
\usepackage{mathrsfs}
\usepackage{pgf,tikz}
\usepackage{wasysym}

% Commandes personnalisées
\newcommand{\euro}{\text{\euro}}
\newcommand{\rep}[1]{\rule{#1cm}{0.2mm}}
\newcommand{\ladate}[1]{\hfill \textit{#1} \hfill\null}
\newcommand{\exo}[1]{\vspace{0.5cm}\textbf{#1 }}
\newcommand{\bemerkung}{\textbf{Remarque :}}

% Configuration de la page
\pagestyle{empty}
\parindent=0mm
\parskip=5mm

\begin{document}
\pagestyle{empty}
\renewcommand{\theenumi}{\arabic{enumi}}
\section*{6ème - Interrogation écrite - Sujet G}
\addcontentsline{toc}{section}{Interrogation-Nombres décimaux}
\ladate{\textbf{Calculatrice non autorisée -- Pas de prêt ni d'échange de matériel}}

\exo{Exercice 1}
\begin{enumerate}
\item Complète le tableau de position suivant et places-y le nombre $1~234~567,89$\\
\begin{minipage}{16cm}
\begin{tabular}{*{14}{| p{0,8cm}}|}
\hline
\multicolumn{9}{| l |}{Partie entière} & \multicolumn{5}{l |}{Partie décimale}\rule[-7pt]{0pt}{40pt}\\
\hline
\multicolumn{3}{| c |}{} & \multicolumn{3}{c |}{} & \multicolumn{3}{c |}{}\rule[-7pt]{0pt}{40pt}&\multirow{2}{0,1cm}&&&&\\
\cline{1-9}
\rule{0cm}{0.5cm}&&&&&&&&&&&&&\\
&&&&&&&&&&&&&\\
\hline
&&&&&&&&&&&&&\\
&&&&&&&&&&&&&\\
\rule{0cm}{0.5cm}&&&&&&&&&&&&&\\
\hline
\end{tabular}
\end{minipage}
\item Dans le nombre $1~234~567,89$ :\\
Quel est le chiffre des dizaines ?\\
\\
Combien y a-t-il de dixièmes ?\\
\\
Quelle est la position du chiffre 9 ?\\ 
\end{enumerate}

\exo{Exercice 2}
Ecris en lettres :\\
\begin{enumerate}
\begin{multicols}{2}
\setlength{\columnseprule}{0pt}
\item quatre-vingt-cinq centaines\\
\item deux millions huit\\
\columnbreak
\item vingt-trois dixièmes\\
\item cent six mille cent six millièmes\\
\end{multicols}
\end{enumerate}

\end{document}