%%!TEX TS-program = latex
\documentclass[12pt]{article}
\usepackage[utf8]{inputenc}
\usepackage[T1]{fontenc}
\usepackage[french]{babel}
\usepackage{amsmath,amssymb}
\usepackage{geometry}
\geometry{a4paper, margin=2cm}
\usepackage{pstricks,pstricks-add,pst-plot,pst-tree,pst-eps}
\usepackage{multicol}
\usepackage{graphicx}
\usepackage{enumitem}
\usepackage{amsfonts}
\usepackage{siunitx}
\usepackage{mathrsfs}
\usepackage{pgf,tikz}
\usepackage{wasysym}

% Commandes personnalisées
\newcommand{\euro}{\text{\euro}}
\newcommand{\rep}[1]{\rule{#1cm}{0.2mm}}
\newcommand{\ladate}[1]{\hfill \textit{#1} \hfill\null}
\newcommand{\exo}[1]{\vspace{0.5cm}\textbf{#1 }}
\newcommand{\bemerkung}{\textbf{Remarque :}}

% Configuration de la page
\pagestyle{empty}
\parindent=0mm
\parskip=5mm

\begin{document}
\pagestyle{empty}
\addcontentsline{toc}{section}{Interrogation écrite-Angles}
\section*{6ème - Interrogation écrite - Sujet D}

\exo{Exercice 1:}\\
\begin{enumerate}
\item Nomme les angles suivants, puis mesure les.\\
\vspace{0.1cm}\\
\begin{pspicture}(0,0)(19,6)
\psline(0,6)(4,6)(3,1)
\psline(1,0.2)(4,0.2)(6,6)
\psline(7,1)(8.5,1)(6.5,5)
\end{pspicture}

\item Trace les triangles en vraie grandeur.\\
\begin{pspicture}(-1,0)(9,4.5)
\psset{unit=0.7cm}
\footnotesize
\pscurve(0,0)(0.2,1)(0.6,2.7)(1,4)
\pscurve(0,0)(1,0.9)(1.95,1.6)(3,2.5) 
\pscurve(1,4)(2,3.05)(3,2.5)
\rput(0,-0.30){{\ECFAugie\fontsize{10pt}{13pt}\selectfont D}}
\rput(1,4.3){{\ECFAugie\fontsize{10pt}{13pt}\selectfont E}}
\rput(3.3,2.5){{\ECFAugie\fontsize{10pt}{13pt}\selectfont G}}
\begin{cursive}
\rput{75}(0.2,2.15){\begin{cursive}4,5~cm \end{cursive}}
\rput{40}(1.7,1.7){\begin{cursive}5,4~cm\end{cursive}}
\rput(0.6,1){35\degres}
\end{cursive}
\psarc(0,0){0.8}{41}{80}
\end{pspicture}
\end{enumerate}

\exo{Exercice 2:}\\
Trace un angle $\widehat{ABC}$ de $67$\degres et sa bissectrice au compas.

\end{document}