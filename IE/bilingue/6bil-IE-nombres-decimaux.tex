\documentclass[10pt,openany]{book}
\documentclass[12pt]{article}
\usepackage[utf8]{inputenc}
\usepackage[T1]{fontenc}
\usepackage[french]{babel}
\usepackage{amsmath,amssymb}
\usepackage{geometry}
\geometry{a4paper, margin=2cm}
\usepackage{pstricks,pstricks-add,pst-plot,pst-tree,pst-eps}
\usepackage{multicol}
\usepackage{graphicx}
\usepackage{enumitem}
\usepackage{amsfonts}
\usepackage{siunitx}
\usepackage{mathrsfs}
\usepackage{pgf,tikz}
\usepackage{wasysym}

% Commandes personnalisées
\newcommand{\euro}{\text{\euro}}
\newcommand{\rep}[1]{\rule{#1cm}{0.2mm}}
\newcommand{\ladate}[1]{\hfill \textit{#1} \hfill\null}
\newcommand{\exo}[1]{\vspace{0.5cm}\textbf{#1 }}
\newcommand{\bemerkung}{\textbf{Remarque :}}

% Configuration de la page
\pagestyle{empty}
\parindent=0mm
\parskip=5mm

\begin{document}

\section*{6\`eme - Interrogation \'ecrite - Sujet G}
\ladate{\textbf{Calculatrice non autoris\'ee -- Pas de pr\^et ni d'\'echange de mat\'eriel}}

\exo{Exercice 1}
	\begin{enumerate}
		\item Compl\`ete le tableau de position suivant et places-y le nombre $1~234~ 567,89$\\
			\begin{minipage}{16cm}
				\begin{tabular}{*{14}{| p{0,8cm}}|}
				\hline
				\multicolumn{9}{| l |}{Partie...} & \multicolumn{5}{l |}{Partie...}\rule[-7pt]{0pt}{40pt}\\
				\hline
				\multicolumn{3}{| c |}{} & \multicolumn{3}{c |}{} & \multicolumn{3}{c |}{}\rule[-7pt]{0pt}{40pt}&\multirow{2}{0,1cm}&&&&\\
				\cline{1-9}
				\rule{0cm}{0.5cm}&&&&&&&&&&&&&\\
				&&&&&&&&&&&&&\\
				\hline
				\rule{0cm}{0.5cm}&&&&&&&&&&&&&\\
				&&&&&&&&&&&&&\\
				&&&&&&&&&&&&&\\
				\hline
				\end{tabular}
			\end{minipage}
		\item Bei der Zahl $1~234~567,89$ :\\
		Welche Ziffer steht an der Zehnerstelle ?\\
		\\
		Wie viele Zehntel gibt es ?\\
		\\
		Welche Stellenwert hat die Ziffer 9 ?\\ 
\end{enumerate}

\exo{\"Ubung 2}
Schreibe folgende Zahlen mit Ziffern.\\
\begin{enumerate}
\begin{multicols}{2}
\setlength{\columnseprule}{0pt}
\item f\"unfundachtzig Hunderter\\
\item zweimillionenacht\\
\columnbreak
\item dreiundzwanzig Zehntel\\
\item hundertsechstausendhundertsechs Tausendstel\\
\end{multicols}
\end{enumerate}

\exo{Exercice 3}
\begin{multicols}{2}
\setlength{\columnseprule}{1pt}
\begin {enumerate}
\item Donner l'\'ecriture d\'ecimale des nombres suivants\\
\begin{enumerate}
\item $(2\times10)+(5\times1)+(9\times0,1)+(7\times0,01)$\\
\vspace{0,7cm}
\item $(5\times100)+(2\times\dfrac{1}{100})$\\
\vspace{0,7cm}
\item$\dfrac{135}{10}=$\\
\vspace{0,7cm}
\item$\dfrac{6~237}{10~000}=$\\
\vspace{0,7cm}
\end{enumerate}
\columnbreak
\item D\'ecompose les nombres suivants en fractions d\'ecimales.\\
\begin{enumerate}
\item $1,36=$\\
\vspace{0,7cm}
\item $102,2=$\\
\vspace{0,7cm}
\item$2~000~001=$\\
\vspace{0,7cm}
\item$0,2011=$\\
\vspace{0,7cm}
\end{enumerate}
\end{enumerate}
\end{multicols}

\end{document}