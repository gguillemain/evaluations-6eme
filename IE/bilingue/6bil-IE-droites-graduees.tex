%%!TEX TS-program = latex
\documentclass[12pt]{article}
\usepackage[utf8]{inputenc}
\usepackage[T1]{fontenc}
\usepackage[french]{babel}
\usepackage{amsmath,amssymb}
\usepackage{geometry}
\geometry{a4paper, margin=2cm}
\usepackage{pstricks,pstricks-add,pst-plot,pst-tree,pst-eps}
\usepackage{multicol}
\usepackage{graphicx}
\usepackage{enumitem}
\usepackage{amsfonts}
\usepackage{siunitx}
\usepackage{mathrsfs}
\usepackage{pgf,tikz}
\usepackage{wasysym}

% Commandes personnalisées
\newcommand{\euro}{\text{\euro}}
\newcommand{\rep}[1]{\rule{#1cm}{0.2mm}}
\newcommand{\ladate}[1]{\hfill \textit{#1} \hfill\null}
\newcommand{\exo}[1]{\vspace{0.5cm}\textbf{#1 }}
\newcommand{\bemerkung}{\textbf{Remarque :}}

% Configuration de la page
\pagestyle{empty}
\parindent=0mm
\parskip=5mm

\begin{document}

\section*{6\`eme - Interrogation \'{e}crite - Sujet D}
\addcontentsline{toc}{section}{Interrogation-Droites gradu\'{e}es}

\exo{Exercice 1}\\
F\"ur jeden Zahlenstrahl schreibe die Abszisse der Punkte A, B und C.
\begin{enumerate}
\item
\begin{minipage}{18cm}
\begin{DroiteGraduee}[none]{15}{0}{7}{5}{1}{0}{0}
\MarquePoint{1}
\AfficheTexte{1}{0.5}{$A$}
\MarquePoint{3.2}
\AfficheTexte{3.2}{0.5}{$B$}
\MarquePoint{6.8}
\AfficheTexte{6.8}{0.5}{$C$}
\AfficheTexte{3}{-0.5}{300}
\AfficheTexte{0}{-0.5}{0}
\end{DroiteGraduee}
\end{minipage}

\item
\begin{minipage}{18cm}
\begin{DroiteGraduee}[none]{15}{0}{7}{10}{1}{0}{0}
\MarquePoint{1.5}
\AfficheTexte{1.5}{0.5}{$B$}
\MarquePoint{3}
\AfficheTexte{3}{0.5}{$A$}
\MarquePoint{7}
\AfficheTexte{7}{0.5}{$C$}
\AfficheTexte{1}{-0.5}{1}
\AfficheTexte{0}{-0.5}{0}
\end{DroiteGraduee}
\end{minipage}

\item
\begin{minipage}{18cm}
\begin{DroiteGraduee}[none]{15}{0}{5}{10}{1}{0.08}{0}
\MarquePoint{5}
\AfficheTexte{5}{0.5}{$A$}
\MarquePoint{0.6}
\AfficheTexte{0.6}{0.5}{$B$}
\MarquePoint{3.7}
\AfficheTexte{3.7}{0.5}{$C$}
\AfficheTexte{2}{-0.5}{0,4}
\AfficheTexte{0}{-0.5}{0,2}
\end{DroiteGraduee}
\end{minipage}

\item
\begin{minipage}{18cm}
\begin{DroiteGraduee}[none]{15}{0}{2}{10}{1}{0}{0}
\MarquePoint{1.1}
\AfficheTexte{1.1}{0.5}{$A$}
\MarquePoint{0.6}
\AfficheTexte{0.6}{0.5}{$B$}
\MarquePoint{1.4}
\AfficheTexte{1.4}{0.5}{$C$}
\AfficheTexte{2}{-0.5}{2}
\AfficheTexte{0}{-0.5}{0}
\end{DroiteGraduee}
\end{minipage}
\end{enumerate}

\exo{Exercice 2 :}
\begin{enumerate}
\item Ecrire un encadrement \`a l'unit\'e des nombres suivants.
\begin{multicols}{3}
\setlength{\columnseprule}{0pt}
$...........<42,58<............$

\columnbreak

$............<99,99<..........$

\columnbreak

$...........<15,021<...........$

\end{multicols}
\item En d\'eduire la valeur approch\'ee par d\'efaut de 42,58.
\vspace{1cm}
\item En d\'eduire la valeur approch\'ee par exc\`es de 15,021.
\end{enumerate}

\exo{"Ubung 3 : }\\
\begin{enumerate}
\item Zeichne einen Abschnitt von dem Zahlenstrahl von 1,3 bis 2,8.\\
Setze die Punkte G und H mit den jeweiligen Abszissen 1,3 und 2,8 , sodass GH = 15 cm.    
\item Auf dem Zahlenstrahl setze folgende Punkte : \\
J(1,5)~;~~K(2,4)~;~~L(2,35)~;~~M(2,04)
\end{enumerate}

\end{document}