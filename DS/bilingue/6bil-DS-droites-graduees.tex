\documentclass[10pt,openany]{book}
\documentclass[12pt]{article}
\usepackage[utf8]{inputenc}
\usepackage[T1]{fontenc}
\usepackage[french]{babel}
\usepackage{amsmath,amssymb}
\usepackage{geometry}
\geometry{a4paper, margin=2cm}
\usepackage{pstricks,pstricks-add,pst-plot,pst-tree,pst-eps}
\usepackage{multicol}
\usepackage{graphicx}
\usepackage{enumitem}
\usepackage{amsfonts}
\usepackage{siunitx}
\usepackage{mathrsfs}
\usepackage{pgf,tikz}
\usepackage{wasysym}

% Commandes personnalisées
\newcommand{\euro}{\text{\euro}}
\newcommand{\rep}[1]{\rule{#1cm}{0.2mm}}
\newcommand{\ladate}[1]{\hfill \textit{#1} \hfill\null}
\newcommand{\exo}[1]{\vspace{0.5cm}\textbf{#1 }}
\newcommand{\bemerkung}{\textbf{Remarque :}}

% Configuration de la page
\pagestyle{empty}
\parindent=0mm
\parskip=5mm

\begin{document}
\section*{Devoir de synth\`ese}
\addcontentsline{toc}{section}{Devoir de synth\`ese - Droites gradu\'ees}

\begin{multicols}{2}
\exo{\"Ubung 1 : Zahlenstrahl}\\
\begin{enumerate}
\item Wie hei\ss en die Br\"uche, die zu den folgenden Punkten geh\"oren ?\\
  Gib, falls m\"oglich, die gleichwertige Dezimalzahl an.
  
\begin{DroiteGraduee}[none]{8}{0}{13}{1}{1}{0}{0}
\AfficheTexte{0}{-0.3}{0}
\AfficheTexte{0}{0.3}{O}
\AfficheTexte{2}{0.3}{M}
\AfficheTexte{4}{0.3}{A}
\AfficheTexte{8}{-0.3}{1}
\AfficheTexte{10}{0.3}{T}
\end{DroiteGraduee}

\columnbreak
\item Setze folgende Punkte auf dem Zahlenstrahl : \\
$H(\dfrac{1}{6})$ ; $E(\dfrac{1}{2})$ ; $P(\dfrac{4}{3})$

\begin{DroiteGraduee}[none]{8}{0}{13}{1}{1}{0}{0}
\AfficheTexte{0}{-0.3}{0}
\AfficheTexte{6}{-0.3}{1}
\end{DroiteGraduee}

\end{enumerate}

\end{multicols}

\exo{\"Ubung 2 : Gleichwertige Br\"uche}\\
Erg\"anze die L\"ucken.\\
\begin{multicols}{2}
$\dfrac{5}{8}=\dfrac{20}{\ldots}=\dfrac{\ldots}{64}$\\


$\dfrac{12}{9}=\dfrac{4}{\ldots}=\dfrac{\ldots}{24}$
\columnbreak

$\dfrac{49}{56}=\dfrac{\ldots}{8}=\dfrac{35}{\ldots}$\\

$0,25=\dfrac{1}{\ldots}=\dfrac{\ldots}{8}$

\end{multicols}

\exo{Exercice 3 : Probl\`emes}
\begin{enumerate}
\item Dans une classe de $26$ \'el\`eves, on forme des \'equipes de volley-ball ($6$ joueurs par \'equipe).\\
Combien d'\'equipes peut on former ?

\item B\"arbel kauft $5$ Hefte f\"ur  je $3,44$~\euro und $6$ P\"ackchen von Din-A4-Bl\"atter.\\
Sie zahlt insgesamt $24,70$~\euro.\\
Wieviel kostet ein P\"ackchen Din-A4-Bl\"atter ?
\end{enumerate}

\exo{Exercice 4 :}\\
Convertir $7~653~s$ en $\ldots h \ldots min \ldots s$.

\exo{\"Ubung 5 : } Zeichne das Bild den Figuren an der Gerade $(\Delta)$\\
(Benutze das Gitter wenn es m\"oglich ist.)\\
\begin{pspicture}(-6,0)(6,8)
\psgrid[subgriddiv=2,gridlabels=0pt,gridcolor=darkgray](-6,6)
\psline[linewidth=1.5pt](-6,3)(12,3)
\pspolygon[linewidth=1.5pt](-6,5)(-4,3)(-3,6)
\pspolygon[linewidth=1.5pt](-2,2)(0,4)(3,4.5)(1,2.5)
\pstGeonode[dotscale=1.5,PointSymbol=x](8,5){O}
\pstGeonode[PointName=none,PointSymbol=none](8,8){A}
\pstCircleOA[linewidth=1.5pt]{O}{A}
\rput(12.4,3){($\Delta$)}
\end{pspicture}

\end{document}