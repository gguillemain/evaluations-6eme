\documentclass[10pt,openany]{book}
\documentclass[12pt]{article}
\usepackage[utf8]{inputenc}
\usepackage[T1]{fontenc}
\usepackage[french]{babel}
\usepackage{amsmath,amssymb}
\usepackage{geometry}
\geometry{a4paper, margin=2cm}
\usepackage{pstricks,pstricks-add,pst-plot,pst-tree,pst-eps}
\usepackage{multicol}
\usepackage{graphicx}
\usepackage{enumitem}
\usepackage{amsfonts}
\usepackage{siunitx}
\usepackage{mathrsfs}
\usepackage{pgf,tikz}
\usepackage{wasysym}

% Commandes personnalisées
\newcommand{\euro}{\text{\euro}}
\newcommand{\rep}[1]{\rule{#1cm}{0.2mm}}
\newcommand{\ladate}[1]{\hfill \textit{#1} \hfill\null}
\newcommand{\exo}[1]{\vspace{0.5cm}\textbf{#1 }}
\newcommand{\bemerkung}{\textbf{Remarque :}}

% Configuration de la page
\pagestyle{empty}
\parindent=0mm
\parskip=5mm

\begin{document}

\section*{Devoir de synth\`ese}
\addcontentsline{toc}{section}{Devoir de synth\`ese-Droites parall\`eles et perpendiculaires}

\exo{\"Ubung 1 : Besondere Geraden}\\
Zu jeder Geraden zeichne in rot die senkrechte Gerade, die durch den Punkt geht.\\
In gr\"un zeichne zu der Geraden die parallele Gerade, die durch den Punkt geht.\\
\emph{(Gerade und zugeh\"origen Punkt haben denselben Name)}

\exo{\"Ubung 2 : Des figures et des triangles}\\
Construis les figures suivantes au dos de la feuille.\\
\begin{multicols}{3}
\begin{pspicture*}(-5.27,-1.13)(15,7.25)
\psgrid[subgriddiv=0,gridlabels=0,gridcolor=lightgray](0,0)(-5.27,-1.13)(14.72,7.25)
\psset{xunit=0.5cm,yunit=0.5cm,algebraic=true,dotstyle=o,dotsize=3pt 0,linewidth=1pt,arrowsize=3pt 2,arrowinset=0.25}
\psline[linecolor=blue](-2,5)(0,5)
\psline[linecolor=blue](0,5)(-1,4)
\psline[linecolor=blue](-1,4)(1,3)
\psline[linecolor=blue](1,3)(0,2)
\psline[linecolor=blue](0,2)(-2,3)
\psline[linecolor=blue](-2,3)(-2,5)
\psline(1,6)(1,1)
\psline(6,0)(12,6)
\pscircle[linecolor=blue](7.96,4){1.41}
\begin{scriptsize}
\psdots[dotstyle=+,linecolor=blue](7.96,4)
\end{scriptsize}
\rput(12.5,6.5){$\Delta$}
\rput(1.2,0.5){$\Delta$}
\end{pspicture*}

\setlength{\columnseprule}{1pt}
\end{multicols}

\begin{multicols}{2}
\exo{\"Ubung 3 : }\\
Zeichne das Bildfigur bei der Achsenspiegelung an $(g)$.\\
\psset{xunit=1cm,yunit=1cm,algebraic=true,dotstyle=o,dotsize=3pt 0,linewidth=1pt,arrowsize=3pt 2,arrowinset=0.25}
\begin{pspicture*}(-2,-4.24)(7,8.9)
\pspolygon[linecolor=blue](2,6)(0.83,3.71)(-1,6)(-1,2)(2,-2)
\psline[linecolor=blue](2,6)(0.83,3.71)
\psline[linecolor=blue](0.83,3.71)(-1,6)
\psline[linecolor=blue](-1,6)(-1,2)
\psline[linecolor=blue](-1,2)(2,-2)
\psline[linecolor=blue](2,-2)(2,6)
\psplot{-8}{14.54}{(-4--8*x)/4}
\rput(-1,-4){$(g)$}
\end{pspicture*}

\columnbreak

\exo{Exercice 4 : }
\begin{enumerate}
\item Trace un rectangle $ABCD$ tel que : $AB=8$cm et $BC=4$cm.
\item Trace $E$ le sym\'etrique du point $A$ par la sym\'etrie d'axe $(BD)$.
\item Quel est le sym\'etrique du segment $AB$ par la sym\'etrie d'axe $(BD)$. Quelle est sa mesure ? Justifie.
\item Quelle est la mesure de l'angle $\widehat{DEB}$ ? Justifie.
\item Quel est le p\'erim\`etre du rectangle $ABCD$.
\item Sans calcul donne la mesure du p\'erim\`etre de $ABED$. Justifie.
\end{enumerate}

\exo{\"Ubung 5 : }
\begin{enumerate}
\item Schreibe in \textbf{mm} : $4$cm ; $3$dm
\item Schreibe in \textbf{cm} : $4,3$ dm ; $4,30$m
\item Schreibe in \textbf{g} : $40$dg ; $2$kg
\item Schreibe in \textbf{dag} : $457$hg ; $2,8$t
\end{enumerate}

\exo{Bonus : Francis et le camembert}\\
Francis veut servir huit parts de camembert \`a ses invit\'es.\\
Il coupe le camembert entier en trois coups de couteau.\\
Comment obtient-il 8 parts de camembert en 3 coups de couteau ? 
\setlength{\columnseprule}{1pt}
\end{multicols}

\end{document}