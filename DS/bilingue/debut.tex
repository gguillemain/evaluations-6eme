\usepackage{auto-pst-pdf}
\usepackage{pstricks-add}
%\usepackage{etex}

% Polices
\usepackage[T1]{fontenc}
\usepackage{lmodern}

% Packages de base
\usepackage[french]{babel}
\usepackage[utf8]{inputenc}
\usepackage{textcomp}

% Packages mathématiques
\usepackage{amsmath}
\usepackage{amsfonts}
\usepackage{makeidx}
\usepackage{mathcomp}
\usepackage{pst-eucl}

% Packages de mise en page
\usepackage{fancybox}
\usepackage{tabularx}
\usepackage{pifont}
\usepackage{dcolumn}
\usepackage{multicol}
\usepackage{xifthen}
\usepackage{lscape}
\usepackage{numprint}
\usepackage{multirow}
\usepackage{array}
\usepackage{hhline}
\usepackage{cancel}
\usepackage{fp}
\usepackage{setspace}
\usepackage{enumitem}
\usepackage[a4paper,
            top=1cm,
            left=1.5cm,
            bottom=1cm, 
            right=1.5cm,
            heightrounded,
            includeheadfoot]{geometry}
\usepackage[sf,sl,outermarks]{titlesec}
\usepackage[Glenn]{fncychap}
\usepackage{dashrule}

% Packages graphiques 
\usepackage{tikz}
\usepackage{graphicx}
\usepackage{colortbl}
\usepackage{lcg}
\usepackage{bclogo}
\usepackage{soul}
\usepackage{pst-platon}

% Configuration du document
\usepackage{titletoc}
\usepackage{tocloft}
\usepackage{imakeidx}
\usepackage{chngcntr}
\usepackage{pst-solides3d}
\usepackage{pgfplots}
\pgfplotsset{compat=1.17}

% Bibliothèques TikZ
\usetikzlibrary{matrix,positioning,shapes,decorations.text,calc,decorations.pathmorphing,patterns}

% Resto de votre configuration (couleurs, commandes personnalisées, etc.)


%ne pas se fier au nom
\definecolor{jorange}{rgb}{0.179,0.628,0.898}
\definecolor{violet}{rgb}{0.695,0.07,0.558}
\definecolor{violetclair}{rgb}{0.998,0.5,1}
\definecolor{violetfonce}{rgb}{0.3,0.03,0.238}
\definecolor{vertclair}{rgb}{0.89,0.95,0.40}
\definecolor{vert}{rgb}{0.552,0.764,0.509}
\definecolor{vert1}{rgb}{0.79,1,0.43}
\definecolor{orange}{rgb}{0.179,0.628,0.898}
\definecolor{orrange}{rgb}{1,0.69,0}
\definecolor{bleuclair}{rgb}{0.468,0.867,0.998}
\definecolor{coulecon}{rgb}{0.758,0.805,0.945}



%\renewcommand \thesection {\@arabic \c@section~-} %Tu as acc\`es a� \Alph, \alph, \Roman, \roman, \arabic

\newsavebox{\circlebox}
\savebox{\circlebox}{\fontencoding{OMS}\selectfont\Large\char13}
\newlength{\circleboxwdht}
\newcommand{\centercircle}[1]{%
  \setlength{\circleboxwdht}{\wd\circlebox}%
  \addtolength{\circleboxwdht}{\dp\circlebox}%
  \raisebox{0.4\dp\circlebox}{%
    \parbox[][\circleboxwdht][c]{\wd\circlebox}{\centering#1}}%
  \llap{\usebox{\circlebox}}%
}
%sectionsssssss
%%%%%%%%%%%%%%%%Probleme
%#1 ontro
%#2 calcul ligne
%#3 calucl pos\'e
%#4 ccl
\newcommand{\pbm}[4]{
\begin{tabular}{p{5cm}|c}
\multicolumn{2}{c}{#1}\\
&\multirow{3}{3cm}{#3}\\
{#2}&\\
&\\
\rule{1cm}{0cm}&\\
\multicolumn{2}{c}{#4}
\end{tabular}
}

\newcommand{\pbmsup}[3]{
\begin{multicols}{2}
\textbf{\underline{#1}}
\columnbreak
#2\\
#3
\end{multicols}

}

\newcommand{\trirectcomp}[5]{
%#1 : point rectangle #2 pointangledroit #3 pointcompasinter #4longueur cot\'e angle droit #5 long. hypot\'enuse 
\pstGeonode[CurveType=polygon,PointSymbol=none,PosAngle={90,0}](0,0){#1}(#4,0){#2}
\psset{PointSymbol=none, PointName=none}
\pstRotation[RotAngle=90]{#1}{#2}[K1]
\pstGeonode(0,#5){Z1}
\pstInterLC{#1}{K1}{#2}{Z1}{N1}{M1}
\psset{PointSymbol=none, PointName=default}
\pstGeonode(N1){#3}
\pstRightAngle[RightAngleSize=0.15]{#2}{#1}{#3}
\pstLineAB{#1}{#3}
\pstLineAB{#2}{#3}
}

\newcommand{\tricomp}[6]{
%#1 : point1 #2 point2 #3 pointcompasinter #4 longueur cot\'e point1point2 #5 longueur cot\'e point1point3 #6 longueur cot\'e point2point3

\pstGeonode[CurveType=polygon,PointSymbol=none,PosAngle={180,0}](0,0){#1}(#4,0){#2}
\psset{PointSymbol=none, PointName=none}
\pstGeonode(#5,0){P1}(#4,#6){P2}
\pstInterCC{#1}{P1}{#2}{P2}{N1}{M1}
\psset{PointSymbol=none, PointName=default}
\pstGeonode[PosAngle=90](N1){#3}
\pstLineAB{#1}{#3}
\pstLineAB{#2}{#3}
}

\newcounter{Num}
\setcounter{Num}{0}
\newcounter{Oppose}
\setcounter{Oppose}{0}
%\newcounter{Coef}
%\setcounter{Coef}{1}

\newcommand{\trirapp}[7][m]{
%%# options : m par defaut marque angle et mesure -  t marque que les angles seulement - r deuxieme angle est droit - i isoc\`ele angle base - s isoc\`ele angle sommet  #2 : point1 #3 point3 #4 pointrapporteurinter #5 longueur cot\'e point1point2 #6 angle point2point1point3 #7 angle point1point2point3 

\pstGeonode[CurveType=polygon,PointSymbol=none,PointNameSep=0.2,PosAngle={180,0}](0,0){#2}(#5,0){#3}
%\psset{PointSymbol=none, PointName=none}

\setcounter{Num}{#7}
\setcounter{Oppose}{\value{Num}*(-1)}

\pstRotation[RotAngle=#6,PointSymbol=none, PointName=none]{#2}{#3}[Z]
\pstRotation[RotAngle=\theOppose,PointSymbol=none, PointName=none]{#3}{#2}[Y]
\pstInterLL[PosAngle=90,PointSymbol=none,PointNameSep=0.2]{#2}{Z}{#3}{Y}{#4}
\psset{PointSymbol=none, PointName=default}
\pstLineAB{#2}{#4}
\pstLineAB{#3}{#4}
  \ifx m#1	
	\pstMarkAngle[linecolor=cyan,MarkAngleRadius=0.4,linewidth=1.5pt]{#3}{#2}{#4}{}
	\pstMarkAngle[linecolor=cyan,MarkAngleRadius=0.6,linewidth=1.5pt,LabelSep=0.9]{#3}{#2}{#4}{\small\textcolor{cyan}{$#6$\degres}}
	\pstMarkAngle[linecolor=cyan,MarkAngleRadius=0.6,linewidth=1.5pt,LabelSep=0.9]{#4}{#3}{#2}{\small\textcolor{cyan}{$#7$\degres}}

 \else
  \ifx t#1
  	\pstMarkAngle[linecolor=magenta,MarkAngleRadius=0.4,linewidth=1.5pt]{#2}{#4}{#3}{}
	\pstMarkAngle[linecolor=magenta,MarkAngleRadius=0.6,linewidth=1.5pt]{#2}{#4}{#3}{}
	\pstMarkAngle[linecolor=cyan,MarkAngleRadius=0.6,linewidth=1.5pt]{#4}{#3}{#2}{}
	\else
 \ifx r#1
  	%\pstMarkAngle[linecolor=cyan,MarkAngleRadius=0.4,linewidth=1.5pt]{#2}{#4}{#3}{}
	\pstMarkAngle[linecolor=cyan,MarkAngleRadius=0.6,linewidth=1.5pt,LabelSep=0.9]{#3}{#2}{#4}{\small\textcolor{cyan}{$#6$\degres}}
	\pstRightAngle[linecolor=cyan,RightAngleSize=0.6,linewidth=1.5pt]{#4}{#3}{#2}
	\else
 \ifx i#1
  	%\pstMarkAngle[linecolor=cyan,MarkAngleRadius=0.4,linewidth=1.5pt]{#2}{#4}{#3}{}
	\pstMarkAngle[linecolor=cyan,MarkAngleRadius=0.6,linewidth=1.5pt,LabelSep=0.9]{#4}{#3}{#2}{\small\textcolor{cyan}{$#7$\degres}}
%	\pstRightAngle[linecolor=cyan,RightAngleSize=0.6,linewidth=1.5pt]{#4}{#3}{#2}
	\pstSegmentMark{#2}{#4}
	\pstSegmentMark{#3}{#4}
	\else
 \ifx s#1
  	%\pstMarkAngle[linecolor=cyan,MarkAngleRadius=0.4,linewidth=1.5pt]{#2}{#4}{#3}{}
	\pstMarkAngle[linecolor=cyan,MarkAngleRadius=0.6,linewidth=1.5pt,LabelSep=0.9]{#4}{#3}{#2}{\small\textcolor{cyan}{$#7$\degres}}
%	\pstRightAngle[linecolor=cyan,RightAngleSize=0.6,linewidth=1.5pt]{#4}{#3}{#2}
	\pstSegmentMark{#2}{#3}
	\pstSegmentMark{#3}{#4}
	\else
  \mathrm{Illegal~option}%
  \fi\fi\fi\fi\fi

}

\newcommand{\rectcomp}[6]{
%#1 : point rectangle #2 pointangledroit #3 pointcompasinter #4 point oppose angle droit #5longueur cot\'e angle droit #6 long. diag 
\pstGeonode[CurveType=polygon,PointSymbol=none,PosAngle={-45,0}](0,0){#1}(#5,0){#2}
\psset{PointSymbol=none, PointName=none}
\pstRotation[RotAngle=90]{#1}{#2}[K1]
\pstGeonode(0,#6){Z1}
\pstInterLC{#1}{K1}{#2}{Z1}{N1}{M1}
\psset{PointSymbol=none, PointName=default}
\pstGeonode[PosAngle=-90](N1){#4}
%\pstRightAngle[RightAngleSize=0.15]{#2}{#1}{#3}
\pstTranslation[PosAngle=-90]{#1}{#2}{#4}[#3]
\pstLineAB{#1}{#2}
\pstLineAB{#2}{#3}
\pstLineAB{#3}{#4}
\pstLineAB{#4}{#1}
}

\newcommand{\trirectdroit}[5]{
%#1 : point rectangle #2 pointangledroit1 #3 pointangledroit2 #4longueur cot\'e angle droit1 #5 long. cot\'eangle droit2 
\pstGeonode[CurveType=polygon,PointSymbol=none,PosAngle={-45,0}](0,0){#1}(#4,0){#2}(0,#5){#3}
\pstRightAngle{#2}{#1}{#3}
}

\newcommand{\rectdroit}[6]{
%#1 : point rectangle #2 pointangledroit1 #3 pointangledroit2 #4pointopppose angle droit #5longueur cot\'e angle droit1 #6 long. cot\'eangle droit2 
\pstGeonode[CurveType=polygon,PointSymbol=none,PosAngle={-45,0}](0,0){#1}(#5,0){#2}(0,#6){#4}
%\pstRightAngle{#2}{#1}{#3}
\pstTranslation{#1}{#2}{#4}[#3]
\pstLineAB{#1}{#2}
\pstLineAB{#2}{#3}
\pstLineAB{#3}{#4}
\pstLineAB{#4}{#1}
}
\newcounter{SecNb}
\setcounter{SecNb}{0}
\newcounter{SecDebut}
\setcounter{SecDebut}{0}
\newcounter{SecFin}
\setcounter{SecFin}{0}
\newcounter{SecTaille}
\setcounter{SecTaille}{0}

\newcommand{\anglesec}[4]{
% #1 Nb secteurs #2 couleur #3 secDebut #4 label
\setcounter{SecNb}{#1}
\setcounter{SecDebut}{#3}
\setcounter{SecTaille}{360/\value{SecNb}}
\setcounter{SecFin}{\value{SecTaille}+\value{SecDebut}}
 		 \begin{pspicture}(-1,-1)(1,1)
		 \psset{linewidth=1.2pt,linecolor=RoyalBlue}
		 \SpecialCoor
		 \pstGeonode[PointSymbol=none,PointName=none](0,0){O}(1;\theSecDebut){A}(1;\theSecFin){B}
		\psset{unit=1}
		\multido{\n=0+\theSecTaille}{\theSecNb}{
		\rput{\n}(0,0){\pswedge[fillstyle=solid,fillcolor=#2,linecolor=white]{1}{0}{\theSecTaille}}
		}
				 \pstMarkAngle[MarkAngleRadius=0.5,LabelSep=0.7]{A}{O}{B}{\textcolor{white}{$#4$}}
		 \pswedge[fillstyle=solid,fillcolor=red,linecolor=white]{0.5}{\theSecDebut}{\theSecFin}
		\end{pspicture}
}

\newcommand{\anglesecfin}[5]{
% #1 Nb secteurs #2 couleur #3 secDebut #4Secfin #5 label
\setcounter{SecNb}{#1}
\setcounter{SecDebut}{#3}
\setcounter{SecTaille}{360/\value{SecNb}}
\setcounter{SecFin}{{\value{SecDebut}+35}}%+{\value{SecFin}/2}}%+\value{SecDebut}}
 		 \begin{pspicture}(-1,-1)(1,1)
		 \psset{linewidth=1.2pt,linecolor=RoyalBlue}
		 \SpecialCoor
		 \pstGeonode[PointSymbol=none,PointName=none](0,0){O}(1;\theSecDebut){A}(1;#4){B}
		\psset{unit=1}
		\multido{\n=0+\theSecTaille}{\theSecNb}{
		\rput{\n}(0,0){\pswedge[fillstyle=solid,fillcolor=#2,linecolor=white]{1}{0}{\theSecTaille}}
		}
		\pstMarkAngle[MarkAngleRadius=0.5,LabelSep=0.7]{A}{O}{B}{}

		 \pswedge[fillstyle=solid,fillcolor=red,linecolor=white]{0.5}{\theSecDebut}{#4}
		 		\rput(0.3;\theSecFin){\textcolor{white}{$#5$}}
		\end{pspicture}
}


%%%%%%%%%%%%%%%%%simplification \'ecriture
\newcommand{\gv}{Gr\"o\ss envorstellung~}

\renewcommand{\thesubsubsection}{\arabic{subsubsection}}



\titleformat{\section}[display]
  {\normalfont}%
  {\filright%
   \normalsize\tiny\hskip 5ptChap.~\thechapter~-~Feuille d'exercices  \tiny \thesection\enspace\vspace{-1cm}}
  {8pt}
{\Large\bfseries\filleft\textcolor{violetfonce}}
[\vspace{-10pt}]



\titleformat*{\section}{\itshape}

\titleformat{\subsection}
 {\normalfont\large\bfseries}{\thesubsection}{1em}{}

\titleformat{\subsection}[hang]
  {\normalfont}%
  {\Large\psframe[fillcolor=jorange,fillstyle=solid,linewidth=0pt,linecolor=orange](-0.1,-0.2)(0.5,0.48)\thesubsection}  
  {0.3cm}
  {\Large\scshape\bfseries}
 
%\chapterfont{\color{white}}{\selectfont}


\renewcommand{\thesubsection}{\Alph{subsection}}

%\renewcommand\section{\@startsection{section}{1}
     %{0mm}% Indentation
      %{0.3cm \@plus -1ex \@minus -.2ex}%espace au dessus du titre
      %{2.3ex \@plus .2ex}%espace en dessous du titre
      %{\reset@font\Large\scshape\bfseries}}%mise en forme taille Large, capitales, gras

%\renewcommand\subsection{\@startsection{subsection}{1}
   %   {0mm}% Indentation
     % {0.3cm \@plus -1ex \@minus -.2ex}%espace au dessus du titre
      %{2.3ex \@plus .2ex}%espace en dessous du titre
      %{\reset@font\Large\scshape\bfseries}}%mise en forme taille Large, capitales, gras





\makeatother

\makeatletter
\renewcommand{\@chapapp}{Kapitel} %redefini le mot chapitre
\makeatother




%%%%%%%%%%%%%%%%%%%%%%%%%%%%%%% d\'efinition PUCES
\renewcommand{\theenumi}{\textbf{\arabic{enumi}}}
\renewcommand{\labelenumi}{\textbf{\theenumi.}}
\renewcommand{\theenumii}{\textbf{\alph{enumii}}}
\renewcommand{\labelenumii}{\textbf{\theenumii.}}


%%%%%%%%%%%%%%%%%%%%%%%%%%%%%%%%%%PRESENTATION
\newcommand\NomPrenom{\textbf{\textit{Nom :\hfill Pr\'enom :\hfill Classe :}}\hspace*{2cm}}

\newcommand*{\titre}[1]{{\centering\bfseries\scshape\Large#1\par}}

\newcommand*{\ladate}[1]{\vspace{0.1cm}{\centering\itshape#1\par}\vspace{0.1cm}}

\newcommand*{\ubungbox}[1]{\fboxsep 0pt%
  \fcolorbox{yellow}{orrange}{#1}}

\newcommand*{\exo}[1]{\vspace{0.35cm plus 0.15cm minus 0.15cm}\rule{1ex}{1ex}\hspace{1ex}{\ubungbox{\textbf{#1}}}\vspace{0.1cm plus 0.1cm minus 0.1cm}}

\newcommand*{\aufgabe}[1]{\vspace{0.35cm plus 0.15cm minus 0.15cm}\rule{6ex}{1ex}\hspace{3ex}\textsc{\textbf{#1}}\vspace{0.1cm plus 0.1cm minus 0.1cm}}

\newcommand\ligne[5]{\vskip#1\hrule height#2\vskip#3\hrule height#4\vskip#5}
\newcommand*{\DoubleLigne}[1]{#1\ligne{5pt plus 2pt minus 2pt}{1.5pt}{2pt}{0.3pt}{5pt plus 2pt minus 2pt}}
\newcommand*{\SimpleLigne}[1]{#1\ligne{4pt plus 2pt minus 2pt}{0.3pt}{0pt}{0pt}{5pt plus 2pt minus 2pt}}


%
% Puces
\newenvironment{Puces}[1][1cm]%
{\begin{list}%
	{$\bullet$}%
	{	\setlength{\leftmargin}{#1}% marge ?		\setlength{\itemsep}{3ptplus3ptminus2pt}% espacement entre item
		\setlength{\topsep}{3ptplus3ptminus2pt}% espacement entre le paragraphe 
	}}%
{\end{list}}


\newcommand{\beispiel}{
\textit{\large\ding{253}Beispiel : }
\smallskip
}

\newcommand{\bemerkung}{
\textit{\large\ding{46}Bemerkung : }
\smallskip
}

%%%%%%%%%%%%%%%%%%%%%%%%%%%%%LES CADRES

\newcommand{\cadrevok}[1]{
\begin{bclogo}[logo=\bcdallemagne,couleur = cyan, arrondi = 0.1, ombre = true, epOmbre = 0.15, couleurOmbre = black!30,barre=snake,couleurBarre=white,blur]{~\textcolor{white}{Vokabeln}}
\noindent#1
\end{bclogo}
}
\newcommand{\cadretrad}[1]{
\begin{bclogo}[logo=\bcdallemagne,couleur = cyan, arrondi = 0.1,barre=snake,couleurBarre=white,blur,couleurBord=cyan]{~\textcolor{white}{Vokabeln}}
\noindent#1
\end{bclogo}
}

\newcommand{\cadreinfo}[2]{
\begin{bclogo}[logo=\bcinfo,couleur=coulecon, epBord=0, arrondi = 0.1, ombre = false, epOmbre = 0.15, couleurOmbre = black!30]{~\fboxsep 2pt\fcolorbox{coulecon}{cyan}{\textcolor{white}{#1}}}
#2
\end{bclogo}
}

\newcommand{\cadrecrayon}[2]{
\begin{bclogo}[logo=\bccrayon,couleur=coulecon, epBord=0, arrondi = 0.1, ombre = false, epOmbre = 0.15, couleurOmbre = black!30]{~\fboxsep 2pt\fcolorbox{coulecon}{cyan}{\textcolor{white}{#1}}}
#2
\end{bclogo}
}

\newcommand{\horloge}[2]{
\psset{yunit=0.8}
\pstGeonode[PointName=none](1,1){O}
\psset{PointSymbol=none,PointName=none}
\pstGeonode(1,3){A}(1,2.775){B}(1,2.565){C}(1,2.85){D}
\pstGeonode(1,2.7){P}(1,2){G}
\footnotesize
\pstCircleOA{O}{A}
\pstLineAB{A}{B}
\pstLineAB{A}{D}
\rput(C){12}
\multido{\ntr=-30+{-30},\i=1+1}{11}{
\pstRotation[RotAngle=\ntr]{O}{A}[A\ntr]
\pstRotation[RotAngle=\ntr]{O}{B}[B\ntr]
\pstRotation[RotAngle=\ntr]{O}{C}[C\ntr]
\pstLineAB{A\ntr}{B\ntr}
\rput(C\ntr){\i}
}
\multido{\r=6+6}{59}{
\pstRotation[RotAngle=\r]{O}{A}[A\r]
\pstRotation[RotAngle=\r]{O}{D}[D\r]
\pstLineAB{A\r}{D\r}
}
\pstRotation[RotAngle=#1]{O}{P}[P1]
\pstRotation[RotAngle=#2]{O}{G}[G1]
\psline[linecolor=red](O)(P1)
\psline[linecolor=blue,linewidth=2pt](O)(G1)
\normalsize

}

\newcommand{\horlogenue}{
\psset{yunit=0.8}
\pstGeonode[PointName=none](1,1){O}
\psset{PointSymbol=none,PointName=none}
\pstGeonode(1,3){A}(1,2.775){B}(1,2.565){C}(1,2.85){D}
\footnotesize
\pstCircleOA{O}{A}
\pstLineAB{A}{B}
\pstLineAB{A}{D}
\rput(C){12}
\multido{\ntr=-30+{-30},\i=1+1}{11}{
\pstRotation[RotAngle=\ntr]{O}{A}[A\ntr]
\pstRotation[RotAngle=\ntr]{O}{B}[B\ntr]
\pstRotation[RotAngle=\ntr]{O}{C}[C\ntr]
\pstLineAB{A\ntr}{B\ntr}
\rput(C\ntr){\i}
}
\multido{\r=6+6}{59}{
\pstRotation[RotAngle=\r]{O}{A}[A\r]
\pstRotation[RotAngle=\r]{O}{D}[D\r]
\pstLineAB{A\r}{D\r}
}

\normalsize

}


\newcommand{\horlogesec}[2]{%deBut et fin du secteur
\psset{yunit=0.8}
\pswedge[fillstyle=solid,fillcolor=violet!60](1,1){1.6}{#1}{#2}
\pstGeonode[PointName=none](1,1){O}
\psset{PointSymbol=none,PointName=none}
\pstGeonode(1,3){A}(1,2.775){B}(1,2.565){C}(1,2.85){D}

\footnotesize
\pstCircleOA{O}{A}
\pstLineAB{A}{B}
\pstLineAB{A}{D}
\rput(C){12}
\multido{\ntr=-30+{-30},\i=1+1}{11}{
\pstRotation[RotAngle=\ntr]{O}{A}[A\ntr]
\pstRotation[RotAngle=\ntr]{O}{B}[B\ntr]
\pstRotation[RotAngle=\ntr]{O}{C}[C\ntr]
\pstLineAB{A\ntr}{B\ntr}
\rput(C\ntr){\i}
}
\multido{\r=6+6}{59}{
\pstRotation[RotAngle=\r]{O}{A}[A\r]
\pstRotation[RotAngle=\r]{O}{D}[D\r]
\pstLineAB{A\r}{D\r}
}

\normalsize

}

\newcommand{\horlogeseccorr}[3]{%deBut et fin du secteur
\psset{yunit=0.8}
\pswedge[fillstyle=solid,fillcolor=#3](1,1){1.6}{#1}{#2}
\pstGeonode[PointName=none](1,1){O}
\psset{PointSymbol=none,PointName=none}
\pstGeonode(1,3){A}(1,2.775){B}(1,2.565){C}(1,2.85){D}

\footnotesize
\pstCircleOA{O}{A}
\pstLineAB{A}{B}
\pstLineAB{A}{D}
\rput(C){12}
\multido{\ntr=-30+{-30},\i=1+1}{11}{
\pstRotation[RotAngle=\ntr]{O}{A}[A\ntr]
\pstRotation[RotAngle=\ntr]{O}{B}[B\ntr]
\pstRotation[RotAngle=\ntr]{O}{C}[C\ntr]
\pstLineAB{A\ntr}{B\ntr}
\rput(C\ntr){\i}
}
\multido{\r=6+6}{59}{
\pstRotation[RotAngle=\r]{O}{A}[A\r]
\pstRotation[RotAngle=\r]{O}{D}[D\r]
\pstLineAB{A\r}{D\r}
}

\normalsize

}


%%%%%%%%%%%%%%%%%%%%%%%%%%% PAVAGE
\newcommand{\penrosevar}[1]{
\begin{pspicture}(-5,2)(6,10)%pavage de penrose
\psset{unit=#1,linewidth=0.5pt,linecolor=violetfonce,PointSymbol=none,PointName=none}
\SpecialCoor
\pstGeonode(1;0){A0}(1;72){A1}(1;144){A2}(1;216){A3}(1;288){A4}
\pstInterLL{A2}{A1}{A4}{A0}{O0}
\pspolygon(A0)(A1)(A2)(A3)(A4)
\newcounter{m}
\setcounter{m}{36}
\multido{\n=0+1}{9}{
\pstRotation[RotAngle=\them]{O0}{A0,A1,A2,A3,A4}[B0\n,B1\n,B2\n,B3\n,B4\n]
\pspolygon(B0\n)(B1\n)(B2\n)(B3\n)(B4\n)
\addtocounter{m}{36}
}

\pstRotation[RotAngle=108]{B00}{B01}[C0]
\pstRotation[RotAngle=-108]{B01}{B00}[C1]
\pspolygon(B01)(B00)(C0)(O0)(C1)
\pstRotation[RotAngle=108,PointName=none]{O0}{B01,B00,C0,C1}[A01,A11,C01,C11]
\pspolygon(A01)(A11)(C01)(O0)(C11)
\pstRotation[RotAngle=108,PointName=none]{O0}{A01,A11,C01,C11}[A02,A12,C02,C12]
\pspolygon(A02)(A12)(C02)(O0)(C12)

\pstTranslation{A1}{B43}{A0,A1,A2,A3,A4,O0}[D0,D1,D2,D3,D4,O1]
\multido{\n=0+1}{9}{
\pstTranslation{A1}{B43}{B0\n,B1\n,B2\n,B3\n,B4\n}[D0\n,D1\n,D2\n,D3\n,D4\n]
\pspolygon(D0\n)(D1\n)(D2\n)(D3\n)(D4\n)
}

\pstInterLL{O0}{B32}{O1}{D30}{X0}
\newcounter{p}
\setcounter{p}{72}
\multido{\i=72+72}{3}{
\multido{\n=0+1}{9}{
\pstRotation[RotAngle=\i]{X0}{B0\n,B1\n,B2\n,B3\n,B4\n}[E0\i\n,E1\i\n,E2\i\n,E3\i\n,E4\i\n]
\pspolygon(E0\i\n)(E1\i\n)(E2\i\n)(E3\i\n)(E4\i\n)
}
\pstRotation[RotAngle=\i]{X0}{A0,A1,A2,A3,A4}[E0\i,E1\i,E2\i,E3\i,E4\i]
\pspolygon(E0\i)(E1\i)(E2\i)(E3\i)(E4\i)
}

\pstMiddleAB[PointName=none,PointSymbol=none]{D04}{D05}{M1}
\pstMiddleAB[PointName=none,PointSymbol=none]{D35}{D25}{M2}
\pstInterLL[PointName=none]{D35}{M1}{M2}{D05}{X2}


\pstRotation[RotAngle=108]{D04}{D05}[N1]
\pstRotation[RotAngle=-108]{D05}{D04}[N2]
\pspolygon(D04)(D05)(N2)(O1)(N1)
\pstRotation[RotAngle=108,PointName=none]{O1}{D04,D05,N2,N1}[D041,D051,N21,N11]
\pspolygon(D041)(D051)(N21)(O1)(N11)
\pstRotation[RotAngle=-108,PointName=none]{O1}{D04,D05,N2,N1}[D042,D052,N22,N12]
\pspolygon(D042)(D052)(N22)(O1)(N12)

\multido{\i=-72+-72}{3}{
\pstRotation[RotAngle=\i]{X0}{O1}[O\i]
}

\pstRotation[RotAngle=108]{E02166}{E02167}[P1]
\pstRotation[RotAngle=-108]{E02167}{E02166}[P2]
\pspolygon(E02166)(E02167)(P2)(O-72)(P1)
\pstRotation[RotAngle=108,PointName=none]{O-72}{E02166,E02167,P2,P1}[E021661,E021671,P21,P11]
\pspolygon(E021661)(E021671)(P21)(O-72)(P11)
\pstRotation[RotAngle=216,PointName=none]{O-72}{E02166,E02167,P2,P1}[E021662,E021672,P22,P12]
\pspolygon(E021662)(E021672)(P22)(O-72)(P12)

\pstRotation[RotAngle=108]{E01440}{E01441}[Q1]
\pstRotation[RotAngle=-108]{E01441}{E01440}[Q2]
\pspolygon(E01440)(E01441)(Q2)(O-144)(Q1)
\pstRotation[RotAngle=108,PointName=none]{O-144}{E01440,E01441,Q2,Q1}[E014401,E014411,Q21,Q11]
\pspolygon(E014401)(E014411)(Q21)(O-144)(Q11)
\pstRotation[RotAngle=216,PointName=none]{O-144}{E01440,E01441,Q2,Q1}[E014402,E014412,Q22,Q12]
\pspolygon(E014402)(E014412)(Q22)(O-144)(Q12)


\pstRotation[RotAngle=108]{E0726}{E0727}[R1]
\pstRotation[RotAngle=-108]{E0727}{E0726}[R2]
\pspolygon(E0726)(E0727)(R2)(O-216)(R1)
\pstRotation[RotAngle=108,PointName=none]{O-216}{E0726,E0727,R2,R1}[E07261,E07271,R21,R11]
\pspolygon(E07261)(E07271)(R21)(O-216)(R11)
\pstRotation[RotAngle=216,PointName=none]{O-216}{E0726,E0727,R2,R1}[E07262,E07272,R22,R12]
\pspolygon(E07262)(E07272)(R22)(O-216)(R12)



\multido{\i=72+72}{4}{
\pstRotation[RotAngle=\i]{X2}{D00,D01,D02,D03,D04,D05,D06,D07,D08,B33}[D00\i,D01\i,D02\i,D03\i,D04\i,D05\i,D06\i,D07\i,D08\i,B33\i]
\pstRotation[RotAngle=\i]{X2}{D041,D042,D051,D052,O1,N1,N11,N12,N2,N21,N22}[D041\i,D042\i,D051\i,D052\i,O1\i,N1\i,N11\i,N12\i,N2\i,N21\i,N22\i]
\pspolygon(D04\i)(D05\i)(N2\i)(O1\i)(N1\i)
\pspolygon(D041\i)(D051\i)(N21\i)(O1\i)(N11\i)
\pspolygon(D042\i)(D052\i)(N22\i)(O1\i)(N12\i)
\pspolygon(D00\i)(D01\i)(D02\i)(D03\i)(D04\i)(D05\i)(D06\i)(D07\i)(D08\i)(B33\i)
}
\end{pspicture}

\pspolygon[fillcolor=vert,fillstyle=solid](O0)(C0)(E2724)(E3724)(B08)(B07)(C12)
\pspolygon[fillcolor=vert,fillstyle=solid](O-216)(R1)(E0726)(E0725)(E0724)(E0723)(R22)
\pspolygon[fillcolor=vert,fillstyle=solid](O1)(N12)(D01)(D00)(B33)(B43)(N21)
\pspolygon[fillcolor=vert,fillstyle=solid](O-72)(P1)(E02166)(E02165)(E02164)(E02163)(P22)
\pspolygon[fillcolor=vert,fillstyle=solid](O-144)(Q1)(E01440)(E0144)(E01448)(E01447)(Q22)
\pspolygon[fillcolor=vert,fillstyle=solid](O1288)(N21288)(D051288)(B33288)(D00288)(D01288)(N12288)
\pspolygon[fillcolor=vert,fillstyle=solid](O1216)(N21216)(D051216)(B33216)(D00216)(D01216)(N12216)
\pspolygon[fillcolor=vert,fillstyle=solid](O1144)(N12144)(D042144)(D00144)(B33144)(D051144)(N21144)
\pspolygon[fillcolor=vert,fillstyle=solid](O172)(N1272)(D04272)(D0072)(B3372)(D05172)(N2172)

\pspolygon[fillcolor=violet,fillstyle=solid](O0)(C1)(B14)(B15)(C11)
\pspolygon[fillcolor=violet,fillstyle=solid](O1)(D0672)(D0572)(D04288)(D03288)
\pspolygon[fillcolor=violet,fillstyle=solid](O1288)(D03216)(D04216)(D04288)(N22288)
\pspolygon[fillcolor=violet,fillstyle=solid](O172)(D06144)(D05144)(D0572)(N1172)
\pspolygon[fillcolor=violet,fillstyle=solid](O172)(D06144)(D05144)(D0572)(N1172)
}


\newcommand{\penrose}{
\begin{pspicture}(-5,2)(6,10)%pavage de penrose
\psset{unit=0.5}%,linewidth=0.5pt,PointSymbol=none,PointName=none}
\SpecialCoor
\pstGeonode(1;0){A0}(1;72){A1}(1;144){A2}(1;216){A3}(1;288){A4}
\pstInterLL{A2}{A1}{A4}{A0}{O0}
\pspolygon(A0)(A1)(A2)(A3)(A4)
\newcounter{m}
\setcounter{m}{36}
\multido{\n=0+1}{9}{
\pstRotation[RotAngle=\them]{O0}{A0,A1,A2,A3,A4}[B0\n,B1\n,B2\n,B3\n,B4\n]
\pspolygon(B0\n)(B1\n)(B2\n)(B3\n)(B4\n)
\addtocounter{m}{36}
}

\pstRotation[RotAngle=108]{B00}{B01}[C0]
\pstRotation[RotAngle=-108]{B01}{B00}[C1]
\pspolygon(B01)(B00)(C0)(O0)(C1)
\pstRotation[RotAngle=108,PointName=none]{O0}{B01,B00,C0,C1}[A01,A11,C01,C11]
\pspolygon(A01)(A11)(C01)(O0)(C11)
\pstRotation[RotAngle=108,PointName=none]{O0}{A01,A11,C01,C11}[A02,A12,C02,C12]
\pspolygon(A02)(A12)(C02)(O0)(C12)

\pstTranslation{A1}{B43}{A0,A1,A2,A3,A4,O0}[D0,D1,D2,D3,D4,O1]
\multido{\n=0+1}{9}{
\pstTranslation{A1}{B43}{B0\n,B1\n,B2\n,B3\n,B4\n}[D0\n,D1\n,D2\n,D3\n,D4\n]
\pspolygon(D0\n)(D1\n)(D2\n)(D3\n)(D4\n)
}

\pstInterLL{O0}{B32}{O1}{D30}{X0}
\newcounter{p}
\setcounter{p}{72}
\multido{\i=72+72}{3}{
\multido{\n=0+1}{9}{
\pstRotation[RotAngle=\i]{X0}{B0\n,B1\n,B2\n,B3\n,B4\n}[E0\i\n,E1\i\n,E2\i\n,E3\i\n,E4\i\n]
\pspolygon(E0\i\n)(E1\i\n)(E2\i\n)(E3\i\n)(E4\i\n)
}
\pstRotation[RotAngle=\i]{X0}{A0,A1,A2,A3,A4}[E0\i,E1\i,E2\i,E3\i,E4\i]
\pspolygon(E0\i)(E1\i)(E2\i)(E3\i)(E4\i)
}

\pstMiddleAB[PointName=none,PointSymbol=none]{D04}{D05}{M1}
\pstMiddleAB[PointName=none,PointSymbol=none]{D35}{D25}{M2}
\pstInterLL[PointName=none]{D35}{M1}{M2}{D05}{X2}


\pstRotation[RotAngle=108]{D04}{D05}[N1]
\pstRotation[RotAngle=-108]{D05}{D04}[N2]
\pspolygon(D04)(D05)(N2)(O1)(N1)
\pstRotation[RotAngle=108,PointName=none]{O1}{D04,D05,N2,N1}[D041,D051,N21,N11]
\pspolygon(D041)(D051)(N21)(O1)(N11)
\pstRotation[RotAngle=-108,PointName=none]{O1}{D04,D05,N2,N1}[D042,D052,N22,N12]
\pspolygon(D042)(D052)(N22)(O1)(N12)

\multido{\i=-72+-72}{3}{
\pstRotation[RotAngle=\i]{X0}{O1}[O\i]
}

\pstRotation[RotAngle=108]{E02166}{E02167}[P1]
\pstRotation[RotAngle=-108]{E02167}{E02166}[P2]
\pspolygon(E02166)(E02167)(P2)(O-72)(P1)
\pstRotation[RotAngle=108,PointName=none]{O-72}{E02166,E02167,P2,P1}[E021661,E021671,P21,P11]
\pspolygon(E021661)(E021671)(P21)(O-72)(P11)
\pstRotation[RotAngle=216,PointName=none]{O-72}{E02166,E02167,P2,P1}[E021662,E021672,P22,P12]
\pspolygon(E021662)(E021672)(P22)(O-72)(P12)

\pstRotation[RotAngle=108]{E01440}{E01441}[Q1]
\pstRotation[RotAngle=-108]{E01441}{E01440}[Q2]
\pspolygon(E01440)(E01441)(Q2)(O-144)(Q1)
\pstRotation[RotAngle=108,PointName=none]{O-144}{E01440,E01441,Q2,Q1}[E014401,E014411,Q21,Q11]
\pspolygon(E014401)(E014411)(Q21)(O-144)(Q11)
\pstRotation[RotAngle=216,PointName=none]{O-144}{E01440,E01441,Q2,Q1}[E014402,E014412,Q22,Q12]
\pspolygon(E014402)(E014412)(Q22)(O-144)(Q12)


\pstRotation[RotAngle=108]{E0726}{E0727}[R1]
\pstRotation[RotAngle=-108]{E0727}{E0726}[R2]
\pspolygon(E0726)(E0727)(R2)(O-216)(R1)
\pstRotation[RotAngle=108,PointName=none]{O-216}{E0726,E0727,R2,R1}[E07261,E07271,R21,R11]
\pspolygon(E07261)(E07271)(R21)(O-216)(R11)
\pstRotation[RotAngle=216,PointName=none]{O-216}{E0726,E0727,R2,R1}[E07262,E07272,R22,R12]
\pspolygon(E07262)(E07272)(R22)(O-216)(R12)



\multido{\i=72+72}{4}{
\pstRotation[RotAngle=\i]{X2}{D00,D01,D02,D03,D04,D05,D06,D07,D08,B33}[D00\i,D01\i,D02\i,D03\i,D04\i,D05\i,D06\i,D07\i,D08\i,B33\i]
\pstRotation[RotAngle=\i]{X2}{D041,D042,D051,D052,O1,N1,N11,N12,N2,N21,N22}[D041\i,D042\i,D051\i,D052\i,O1\i,N1\i,N11\i,N12\i,N2\i,N21\i,N22\i]
\pspolygon(D04\i)(D05\i)(N2\i)(O1\i)(N1\i)
\pspolygon(D041\i)(D051\i)(N21\i)(O1\i)(N11\i)
\pspolygon(D042\i)(D052\i)(N22\i)(O1\i)(N12\i)
\pspolygon(D00\i)(D01\i)(D02\i)(D03\i)(D04\i)(D05\i)(D06\i)(D07\i)(D08\i)(B33\i)
}
\end{pspicture}



\rput{72}(4.75,-0.96){
\begin{pspicture}(-5,2)(6,10)%pavage de penrose
\psset{unit=0.5,linewidth=0.5pt,PointSymbol=none,PointName=none}
\SpecialCoor
\pstGeonode(1;0){A0}(1;72){A1}(1;144){A2}(1;216){A3}(1;288){A4}
\pstInterLL{A2}{A1}{A4}{A0}{O0}
\pspolygon(A0)(A1)(A2)(A3)(A4)
\newcounter{om}
\setcounter{om}{36}
\multido{\n=0+1}{9}{
\pstRotation[RotAngle=\theom]{O0}{A0,A1,A2,A3,A4}[B0\n,B1\n,B2\n,B3\n,B4\n]
\pspolygon(B0\n)(B1\n)(B2\n)(B3\n)(B4\n)
\addtocounter{om}{36}
}

\pstRotation[RotAngle=108]{B00}{B01}[C0]
\pstRotation[RotAngle=-108]{B01}{B00}[C1]
\pspolygon(B01)(B00)(C0)(O0)(C1)
\pstRotation[RotAngle=108,PointName=none]{O0}{B01,B00,C0,C1}[A01,A11,C01,C11]
\pspolygon(A01)(A11)(C01)(O0)(C11)
\pstRotation[RotAngle=108,PointName=none]{O0}{A01,A11,C01,C11}[A02,A12,C02,C12]
\pspolygon(A02)(A12)(C02)(O0)(C12)

\pstTranslation{A1}{B43}{A0,A1,A2,A3,A4,O0}[D0,D1,D2,D3,D4,O1]
\multido{\n=0+1}{9}{
\pstTranslation{A1}{B43}{B0\n,B1\n,B2\n,B3\n,B4\n}[D0\n,D1\n,D2\n,D3\n,D4\n]
\pspolygon(D0\n)(D1\n)(D2\n)(D3\n)(D4\n)
}

\pstInterLL{O0}{B32}{O1}{D30}{X0}

\multido{\i=72+72}{3}{
\multido{\n=0+1}{9}{
\pstRotation[RotAngle=\i]{X0}{B0\n,B1\n,B2\n,B3\n,B4\n}[E0\i\n,E1\i\n,E2\i\n,E3\i\n,E4\i\n]
\pspolygon(E0\i\n)(E1\i\n)(E2\i\n)(E3\i\n)(E4\i\n)
}
\pstRotation[RotAngle=\i]{X0}{A0,A1,A2,A3,A4}[E0\i,E1\i,E2\i,E3\i,E4\i]
\pspolygon(E0\i)(E1\i)(E2\i)(E3\i)(E4\i)
}

\pstMiddleAB[PointName=none,PointSymbol=none]{D04}{D05}{M1}
\pstMiddleAB[PointName=none,PointSymbol=none]{D35}{D25}{M2}
\pstInterLL[PointName=none]{D35}{M1}{M2}{D05}{X2}

\pstRotation[RotAngle=108]{D04}{D05}[N1]
\pstRotation[RotAngle=-108]{D05}{D04}[N2]
\pspolygon(D04)(D05)(N2)(O1)(N1)
\pstRotation[RotAngle=108,PointName=none]{O1}{D04,D05,N2,N1}[D041,D051,N21,N11]
\pspolygon(D041)(D051)(N21)(O1)(N11)
\pstRotation[RotAngle=-108,PointName=none]{O1}{D04,D05,N2,N1}[D042,D052,N22,N12]
\pspolygon(D042)(D052)(N22)(O1)(N12)

\multido{\i=-72+-72}{3}{
\pstRotation[RotAngle=\i]{X0}{O1}[O\i]
}

\pstRotation[RotAngle=108]{E02166}{E02167}[P1]
\pstRotation[RotAngle=-108]{E02167}{E02166}[P2]
\pspolygon(E02166)(E02167)(P2)(O-72)(P1)
\pstRotation[RotAngle=108,PointName=none]{O-72}{E02166,E02167,P2,P1}[E021661,E021671,P21,P11]
\pspolygon(E021661)(E021671)(P21)(O-72)(P11)
\pstRotation[RotAngle=216,PointName=none]{O-72}{E02166,E02167,P2,P1}[E021662,E021672,P22,P12]
\pspolygon(E021662)(E021672)(P22)(O-72)(P12)

\pstRotation[RotAngle=108]{E01440}{E01441}[Q1]
\pstRotation[RotAngle=-108]{E01441}{E01440}[Q2]
\pspolygon(E01440)(E01441)(Q2)(O-144)(Q1)
\pstRotation[RotAngle=108,PointName=none]{O-144}{E01440,E01441,Q2,Q1}[E014401,E014411,Q21,Q11]
\pspolygon(E014401)(E014411)(Q21)(O-144)(Q11)
\pstRotation[RotAngle=216,PointName=none]{O-144}{E01440,E01441,Q2,Q1}[E014402,E014412,Q22,Q12]
\pspolygon(E014402)(E014412)(Q22)(O-144)(Q12)


\pstRotation[RotAngle=108]{E0726}{E0727}[R1]
\pstRotation[RotAngle=-108]{E0727}{E0726}[R2]
\pspolygon(E0726)(E0727)(R2)(O-216)(R1)
\pstRotation[RotAngle=108,PointName=none]{O-216}{E0726,E0727,R2,R1}[E07261,E07271,R21,R11]
\pspolygon(E07261)(E07271)(R21)(O-216)(R11)
\pstRotation[RotAngle=216,PointName=none]{O-216}{E0726,E0727,R2,R1}[E07262,E07272,R22,R12]
\pspolygon(E07262)(E07272)(R22)(O-216)(R12)



\multido{\i=72+72}{4}{
\pstRotation[RotAngle=\i]{X2}{D00,D01,D02,D03,D04,D05,D06,D07,D08,B33}[D00\i,D01\i,D02\i,D03\i,D04\i,D05\i,D06\i,D07\i,D08\i,B33\i]
\pstRotation[RotAngle=\i]{X2}{D041,D042,D051,D052,O1,N1,N11,N12,N2,N21,N22}[D041\i,D042\i,D051\i,D052\i,O1\i,N1\i,N11\i,N12\i,N2\i,N21\i,N22\i]
\pspolygon(D04\i)(D05\i)(N2\i)(O1\i)(N1\i)
\pspolygon(D041\i)(D051\i)(N21\i)(O1\i)(N11\i)
\pspolygon(D042\i)(D052\i)(N22\i)(O1\i)(N12\i)
\pspolygon(D00\i)(D01\i)(D02\i)(D03\i)(D04\i)(D05\i)(D06\i)(D07\i)(D08\i)(B33\i)
}

\end{pspicture}
}


\rput{144}(9.64,-2.94){
\begin{pspicture}(-5,2)(6,10)%pavage de penrose
\psset{unit=0.5,linewidth=0.5pt,PointSymbol=none,PointName=none}
\SpecialCoor
\pstGeonode(1;0){A0}(1;72){A1}(1;144){A2}(1;216){A3}(1;288){A4}
\pstInterLL{A2}{A1}{A4}{A0}{O0}
\pspolygon(A0)(A1)(A2)(A3)(A4)
\newcounter{oom}
\setcounter{oom}{36}
\multido{\n=0+1}{9}{
\pstRotation[RotAngle=\theoom]{O0}{A0,A1,A2,A3,A4}[B0\n,B1\n,B2\n,B3\n,B4\n]
\pspolygon(B0\n)(B1\n)(B2\n)(B3\n)(B4\n)
\addtocounter{oom}{36}
}

\pstRotation[RotAngle=108]{B00}{B01}[C0]
\pstRotation[RotAngle=-108]{B01}{B00}[C1]
\pspolygon(B01)(B00)(C0)(O0)(C1)
\pstRotation[RotAngle=108,PointName=none]{O0}{B01,B00,C0,C1}[A01,A11,C01,C11]
\pspolygon(A01)(A11)(C01)(O0)(C11)
\pstRotation[RotAngle=108,PointName=none]{O0}{A01,A11,C01,C11}[A02,A12,C02,C12]
\pspolygon(A02)(A12)(C02)(O0)(C12)

\pstTranslation{A1}{B43}{A0,A1,A2,A3,A4,O0}[D0,D1,D2,D3,D4,O1]
\multido{\n=0+1}{9}{
\pstTranslation{A1}{B43}{B0\n,B1\n,B2\n,B3\n,B4\n}[D0\n,D1\n,D2\n,D3\n,D4\n]
\pspolygon(D0\n)(D1\n)(D2\n)(D3\n)(D4\n)
}

\pstInterLL{O0}{B32}{O1}{D30}{X0}

\multido{\i=72+72}{3}{
\multido{\n=0+1}{9}{
\pstRotation[RotAngle=\i]{X0}{B0\n,B1\n,B2\n,B3\n,B4\n}[E0\i\n,E1\i\n,E2\i\n,E3\i\n,E4\i\n]
\pspolygon(E0\i\n)(E1\i\n)(E2\i\n)(E3\i\n)(E4\i\n)
}
\pstRotation[RotAngle=\i]{X0}{A0,A1,A2,A3,A4}[E0\i,E1\i,E2\i,E3\i,E4\i]
\pspolygon(E0\i)(E1\i)(E2\i)(E3\i)(E4\i)
}

\pstMiddleAB[PointName=none,PointSymbol=none]{D04}{D05}{M1}
\pstMiddleAB[PointName=none,PointSymbol=none]{D35}{D25}{M2}
\pstInterLL[PointName=none]{D35}{M1}{M2}{D05}{X2}

\pstRotation[RotAngle=108]{D04}{D05}[N1]
\pstRotation[RotAngle=-108]{D05}{D04}[N2]
\pspolygon(D04)(D05)(N2)(O1)(N1)
\pstRotation[RotAngle=108,PointName=none]{O1}{D04,D05,N2,N1}[D041,D051,N21,N11]
\pspolygon(D041)(D051)(N21)(O1)(N11)
\pstRotation[RotAngle=-108,PointName=none]{O1}{D04,D05,N2,N1}[D042,D052,N22,N12]
\pspolygon(D042)(D052)(N22)(O1)(N12)

\multido{\i=-72+-72}{3}{
\pstRotation[RotAngle=\i]{X0}{O1}[O\i]
}

\pstRotation[RotAngle=108]{E02166}{E02167}[P1]
\pstRotation[RotAngle=-108]{E02167}{E02166}[P2]
\pspolygon(E02166)(E02167)(P2)(O-72)(P1)
\pstRotation[RotAngle=108,PointName=none]{O-72}{E02166,E02167,P2,P1}[E021661,E021671,P21,P11]
\pspolygon(E021661)(E021671)(P21)(O-72)(P11)
\pstRotation[RotAngle=216,PointName=none]{O-72}{E02166,E02167,P2,P1}[E021662,E021672,P22,P12]
\pspolygon(E021662)(E021672)(P22)(O-72)(P12)

\pstRotation[RotAngle=108]{E01440}{E01441}[Q1]
\pstRotation[RotAngle=-108]{E01441}{E01440}[Q2]
\pspolygon(E01440)(E01441)(Q2)(O-144)(Q1)
\pstRotation[RotAngle=108,PointName=none]{O-144}{E01440,E01441,Q2,Q1}[E014401,E014411,Q21,Q11]
\pspolygon(E014401)(E014411)(Q21)(O-144)(Q11)
\pstRotation[RotAngle=216,PointName=none]{O-144}{E01440,E01441,Q2,Q1}[E014402,E014412,Q22,Q12]
\pspolygon(E014402)(E014412)(Q22)(O-144)(Q12)


\pstRotation[RotAngle=108]{E0726}{E0727}[R1]
\pstRotation[RotAngle=-108]{E0727}{E0726}[R2]
\pspolygon(E0726)(E0727)(R2)(O-216)(R1)
\pstRotation[RotAngle=108,PointName=none]{O-216}{E0726,E0727,R2,R1}[E07261,E07271,R21,R11]
\pspolygon(E07261)(E07271)(R21)(O-216)(R11)
\pstRotation[RotAngle=216,PointName=none]{O-216}{E0726,E0727,R2,R1}[E07262,E07272,R22,R12]
\pspolygon(E07262)(E07272)(R22)(O-216)(R12)



\multido{\i=72+72}{4}{
\pstRotation[RotAngle=\i]{X2}{D00,D01,D02,D03,D04,D05,D06,D07,D08,B33}[D00\i,D01\i,D02\i,D03\i,D04\i,D05\i,D06\i,D07\i,D08\i,B33\i]
\pstRotation[RotAngle=\i]{X2}{D041,D042,D051,D052,O1,N1,N11,N12,N2,N21,N22}[D041\i,D042\i,D051\i,D052\i,O1\i,N1\i,N11\i,N12\i,N2\i,N21\i,N22\i]
\pspolygon(D04\i)(D05\i)(N2\i)(O1\i)(N1\i)
\pspolygon(D041\i)(D051\i)(N21\i)(O1\i)(N11\i)
\pspolygon(D042\i)(D052\i)(N22\i)(O1\i)(N12\i)
\pspolygon(D00\i)(D01\i)(D02\i)(D03\i)(D04\i)(D05\i)(D06\i)(D07\i)(D08\i)(B33\i)
}

\end{pspicture}
}
}
%%%%%%%%%%%%%%%%%%%%%%%%%%Fin PAVAGE



%%%%%%%%%%%%%%%%%%%%%%%%%%%  MATHS

\newcommand{\R}{\mathbb{R}}
\newcommand{\N}{\mathbb{N}}
\newcommand{\D}{\mathbb{D}}
\newcommand{\Z}{\mathbb{Z}}
\newcommand{\Q}{\mathbb{Q}}
\newcommand{\C}{\mathbb{C}}

\newcommand{\vect}[1]{\mathchoice%
{\overrightarrow{\displaystyle\mathstrut#1\,\,}}%
{\overrightarrow{\textstyle\mathstrut#1\,\,}}%
{\overrightarrow{\scriptstyle\mathstrut#1\,\,}}%
{\overrightarrow{\scriptscriptstyle\mathstrut#1\,\,}}}
\def\Oij{$\left(\text{O},~\vect{\imath},~\vect{\jmath}\right)$}
\def\Oijk{$\left(\text{O},~\vect{\imath},~ \vect{\jmath},~ \vect{k}\right)$}
\def\Ouv{$\left(\text{O},~\vect{u},~\vect{v}\right)$}


\DeclareMathSymbol{A}{\mathalpha}{operators}{65}
\DeclareMathSymbol{B}{\mathalpha}{operators}{66}
\DeclareMathSymbol{C}{\mathalpha}{operators}{67}
\DeclareMathSymbol{D}{\mathalpha}{operators}{68}
\DeclareMathSymbol{E}{\mathalpha}{operators}{69}
\DeclareMathSymbol{F}{\mathalpha}{operators}{70}
\DeclareMathSymbol{G}{\mathalpha}{operators}{71}
\DeclareMathSymbol{H}{\mathalpha}{operators}{72}
\DeclareMathSymbol{I}{\mathalpha}{operators}{73}
\DeclareMathSymbol{J}{\mathalpha}{operators}{74}
\DeclareMathSymbol{K}{\mathalpha}{operators}{75}
\DeclareMathSymbol{L}{\mathalpha}{operators}{76}
\DeclareMathSymbol{M}{\mathalpha}{operators}{77}
\DeclareMathSymbol{N}{\mathalpha}{operators}{78}
\DeclareMathSymbol{O}{\mathalpha}{operators}{79}
\DeclareMathSymbol{P}{\mathalpha}{operators}{80}
\DeclareMathSymbol{Q}{\mathalpha}{operators}{81}
\DeclareMathSymbol{R}{\mathalpha}{operators}{82}
\DeclareMathSymbol{S}{\mathalpha}{operators}{83}
\DeclareMathSymbol{T}{\mathalpha}{operators}{84}
\DeclareMathSymbol{U}{\mathalpha}{operators}{85}
\DeclareMathSymbol{V}{\mathalpha}{operators}{86}
\DeclareMathSymbol{W}{\mathalpha}{operators}{87}
\DeclareMathSymbol{X}{\mathalpha}{operators}{88}
\DeclareMathSymbol{Y}{\mathalpha}{operators}{89}
\DeclareMathSymbol{Z}{\mathalpha}{operators}{90}


%==========================papier pointe
\newcommand{\pappointequad}[3]{%#1 nB points largeur, #2 nB points hauteur,#3 distance points
\psset{PointName=none,linecolor=gray}
\pstGeonode(0,0){A11}
\pstGeonode[PointSymbol=none](0,0){A}(#3,0){B}(0,#3){C}
\newcounter{jup}
\setcounter{jup}{2}
\newcounter{pou}
\setcounter{pou}{2}
\newcounter{largeur}
\setcounter{largeur}{#1}
\addtocounter{largeur}{-1}
\newcounter{hauteur}
\setcounter{hauteur}{#2}
\addtocounter{hauteur}{-1}
\multido{\i=1+1}{\thelargeur}{
\pstTranslation{A}{B}{A1\i}[A1\thejup]
\addtocounter{jup}{1}
}
\multido{\n=1+1}{\thehauteur}{
\multido{\i=1+1}{#1}{
\pstTranslation{A}{C}{A\n\i}[A\thepou\i]
}
\addtocounter{pou}{1}
}
}

\newcommand{\pappointetri}[3]{%#1 nB points largeur, #2 nB points hauteur,#3 distance points
\psset{PointName=none,linecolor=gray}
\pstGeonode(0,0){A11}
\pstGeonode[PointSymbol=none](0,0){A}(#3,0){B}
\newcounter{ajup}
\setcounter{ajup}{2}
\newcounter{apou}
\setcounter{apou}{2}
\newcounter{akei}
\setcounter{akei}{3}
\newcounter{alargeur}
\setcounter{alargeur}{#1}
\addtocounter{alargeur}{-1}
\newcounter{ahauteur}
\setcounter{ahauteur}{#2}
\addtocounter{ahauteur}{-1}
\multido{\i=1+1}{\thealargeur}{
\pstTranslation{A}{B}{A1\i}[A1\theajup]
\addtocounter{ajup}{1}
}
\setcounter{ajup}{2}

\multido{\n=1+2}{\theahauteur}{
\setcounter{ajup}{2}
\multido{\i=1+1}{#1}{
\pstRotation[RotAngle=60]{A\n\i}{A\n\theajup}[A\theapou\i]
\addtocounter{ajup}{1}
}
\pstRotation[RotAngle=-120]{A\theapou1}{A\n1}[A\theakei1]
\setcounter{ajup}{2}
\multido{\i=1+1}{\thealargeur}{
\pstTranslation{A}{B}{A\theakei\i}[A\theakei\theajup]
\addtocounter{ajup}{1}
}
\addtocounter{apou}{2}
\addtocounter{akei}{2}
}
}
%%%%%%%%%%%%%%%%%%%%%%%%%%commande fractions


%

\newcounter{secteurf}
\newcounter{secteurd}
\newcounter{Bsecteurf}
\newcounter{Bsecteurd}
\newcounter{nB}
\newcounter{nBv}
\newlength{\shift}
\newcommand{\gemzahlcirc}[6][c]{
%option : c couleur pardefaut t transparent pour mettre une image en dessous
%entier
%numerateur
%denominateur
%echelle
%couleur
  \begin{tikzpicture}[scale=#5]
\pgfmathsetcounter{secteurf}{360/#4}

\pgfmathsetcounter{secteurd}{0}

\pgfmathsetcounter{Bsecteurf}{360/#4}

\pgfmathsetcounter{Bsecteurd}{0}

\pgfmathsetcounter{nB}{#2+1}

\setlength{\shift}{0cm}
\multido{\i=0+1}{\thenB}{
\begin{scope}[xshift=\shift]
\multido{\n=0+1}{#4}{
\ifx c#1 
\filldraw[fill=#6!50!white,draw=black,thick] (0,0) -- (\thesecteurd:1) arc (\thesecteurd:\thesecteurf:1) -- cycle;
\else
\ifx t#1
\filldraw[fill=none,draw=black,thick] (0,0) -- (\thesecteurd:1) arc (\thesecteurd:\thesecteurf:1) -- cycle;
\fi\fi
\pgfmathaddtocounter{secteurf}{360/#4}
\pgfmathaddtocounter{secteurd}{360/#4}
}
\end{scope}
\pgfmathaddtolength{\shift}{2.1cm}
}

\pgfmathsetcounter{nBv}{#4-#3}
\pgfmathaddtolength{\shift}{2.1cm*#2}
\begin{scope}[xshift=\shift]\
\multido{\n=0+1}{\thenBv}{
\ifx c#1 
\filldraw[fill=white,draw=black,thick] (0,0) -- (\theBsecteurd:1) arc (\theBsecteurd:\theBsecteurf:1) -- cycle;
\else
\ifx t#1
\filldraw[fill=none,draw=black,thick] (0,0) -- (\theBsecteurd:1) arc (\theBsecteurd:\theBsecteurf:1) -- cycle;
\fi\fi
\pgfmathaddtocounter{Bsecteurf}{360/#4}
\pgfmathaddtocounter{Bsecteurd}{360/#4}
}


\end{scope}

  \end{tikzpicture}
}


%%_______________________________________________________________________________
%

%
%                           par JCC sur f.c.t.tex
%_______________________________________________________________________________
\def\nombrefr#1{\expandafter{\changecomma{#1}}}
\def\changecomma#1{\expandafter\changecommaaux#1.\changecommaaux}
\def\changecommaaux#1.#2\changecommaaux{#1\ifx\empty#2\else,\expandafter\changecommapt#2\changecommapt\fi}
\def\changecommapt#1.\changecommapt{#1}

\makeatletter
\newcount\integerpart
\newcount\decimalpart
\newcommand\IFnombre[3]{%
	\decimalpart=0
	\afterassignment\defnext\integerpart=0#1\relax\@nil
	\expandafter\@dotorcomma\next\@nil
	\if\relax\@remain
		#2%
	\else
		#3
	\fi
	}
\def\defnext#1\@nil{\def\next{#1}}%
\def\@dotorcomma{\@ifnextchar.{\@decimal}{\@comma}}
\def\@comma{\@ifnextchar,{\@decimal}{\@\endnumber}}
\def\@decimal#1#2\@nil{%
	\afterassignment\defnext\decimalpart=0#2\@nil
	\expandafter\@\endnumber\next\@nil
}
\def\@\endnumber#1\@nil{\def\@remain{#1}}
\makeatother
\newcommand*{\setenumeratedefaut}{
\setenumerate{itemsep=2ptplus2ptminus2pt,topsep=\the\itemsep,partopsep=0cm,parsep=0pt}}
\setenumeratedefaut
\let\oldenumerate=\enumerate
\let\oldendenumerate=\endenumerate

\makeatletter
\newcommand*{\gras}[1]{%
	\ifmmode% si c'est le mode maths
		\begingroup
			\renewcommand*\npunitcommand[1]{\ensuremath{\mathbf{##1}}}% pour \numprint --> met en gras les unit?finit la commande \text du package amstext
			\mathbf{#1}%
		\endgroup
	\else%
		\begingroup
			\mathversion{Bold}% pour mettre les formules en gras s'il y en a
			\textbf{#1}% mode texte
		\endgroup
	\fi}
\makeatother
%
%____________________
%
%    Forme un arc
%____________________
\newcommand*{\Angle}[1]{\ensuremath{\widehat{#1}}}
%
% Forme un arc commande t Bas
\makeatletter
\newcount\r@pport \newdimen\r@ppord
\newcount\kslant \newdimen\kslantd
\newcommand*{\arc}[1]{\setbox0\hbox{$\m@th\displaystyle#1$}\kslant=\ht0
	\divide\kslant by1000\multiply\kslant by\fontdimen1\textfont1
	\divide\kslant by10000\kslantd=\kslant\fontdimen6\textfont1
	\divide\kslantd by7750\kern\kslantd
	\r@ppord=\wd0\multiply\r@ppord by100\divide\r@ppord by\ht0
	\multiply\r@ppord by300\advance\r@ppord by\ht0
	\pspicture(0,0)
	\parabola[linewidth=.3pt]{-}(0,1.05\ht0)(.5\wd0,1.15\r@ppord)
	\endpspicture
	\kern-\kslantd\box0}
\makeatother
%
% Forme un arc commande paisseur de la ligne
\newcommand*{\Arc}[2][0.3pt]{%
	\newlength{\longarc}
	\settowidth{\longarc}{$#2$}%
		\unitlength\longarc%
		\ensuremath{%
			\stackrel{\begin{picture}(1,0.2)\linethickness{#1}\qbezier(0,0)(0.5,0.2)(1,0)\end{picture}}%
			{#2}%
		}%
}
% Met entre guillemets fran

\def\guill#1{\og{}#1\fg{}}
% _______________________________________________________________________
%|                                                                       |
%|   Met un signe = si \Delta est suffisemment petit, met \approx sinon  |
%|_______________________________________________________________________|
\newcommand*{\SigneEgal}[1]{\FPabs{\Delta}{#1}\FPiflt{\Delta}{0.000000001}=\else\approx\fi}
\newcommand*{\RectangleEn}[4]{%
	\ifthenelse{#1=1}
		{#2}% #1=2, renvoie #2
		{\ifthenelse{#1=2}
			{#3}% #1=3, renvoie #3
			{\ifthenelse{#1=3}
				{#4}% #1=4, renvoie #4
				{??}% #1 est autre chose, renvoie ??
			}}}
\makeatletter\newcommand*{\PythagoreDirect}{\@ifstar{\def\AvecEq{0}\PythagoreDirecT}{\def\AvecEq{1}\PythagoreDirecT}}\makeatother
\newcommand*{\PythagoreDirecT}[4][2]{%
% [#1] optionnel : position de la lettre o? -dire la lettre du milieu)
	Le triangle #2#3#4 est rectangle en \RectangleEn{#1}{#2}{#3}{#4}, donc d'apr\`e le th\'eor\^eme de Pythagore%
	\ifthenelse{\AvecEq=1}%
		{\ifthenelse{#1=1}%
			{ : ${#3#4}^2={#2#3}^2+{#2#4}^2$}%
				{\ifthenelse{#1=2}%
					{ : ${#2#4}^2={#3#2}^2+{#3#4}^2$}%
					{\ifthenelse{#1=3}%
						{ : ${#2#3}^2={#4#2}^2+{#4#3}^2$}%
						{ : ??}%
					}%
				}%
		}%
		{}}
\makeatletter\newcommand*{\ThalesDirectTrois}{\@ifstar{\def\MaFrac{\dfrac}\ThalesDirectTroiS}{\def\MaFrac{\frac}\ThalesDirectTroiS}}\makeatother
\newcommand*{\ThalesDirectTroiS}[5]{%
	%#1:centre homot direct
	Les droites (#2#3) et (#4#5) se coupent en #1, les droites (#2#4) et (#3#5) sont : $\MaFrac{#1#2}{#1#3}=\MaFrac{#1#4}{#1#5}=\MaFrac{#2#4}{#3#5}$%
}
\newcommand*{\ThalesReciproquE}[5]{%
	%les rapports #1#2/#1#3 et #1#4/#1#5 sont ?s
	On obtient l$\MaFrac{#1#2}{#1#3}=\MaFrac{#1#4}{#1#5}$~, les points #1, #2, #3 et #1, #4, #5 sont align?â\"aBciproque du ths, \textbf{les droites (#2#4) et (#3#5) sont parall?les}.
}
\makeatletter\newcommand*{\ThalesReciproque}{\@ifstar{\def\MaFrac{\dfrac}\ThalesReciproquE}{\def\MaFrac{\frac}\ThalesReciproquE}}
\makeatother

%% COMMANDE POUR FAIRE UN QUADRILLAGE "QUELCONQUE"

%% Le quadrillage sera engendr?â\"aB quadrillage par un argument optionnel.
%% Par exemple \quadrillageTEST[linecolor=green](0,0)(1,0)(1,1)(-5,-7)(3,4).



% Pour comprendre facilement la macro, on se place dans le
% du quadrillage sont soit "horizontaux" soit "verticaux".

\makeatletter
% Des compteurs utiles pour la commande \quadrillageTEST.
\newcounter{compteur@xquadrillage}
\newcounter{compteur@yquadrillage}

\def\quadrillageTEST{%
   \@ifnextchar[{\def@quad@option}{\def@quad@sansOption}}

\def\def@quad@option[#1](#2)(#3)(#4)(#5,#6)(#7,#8){
   %  #1 = les attriButs en option
   %  #2 = O'
   %  #3 = I'
   %  #4 = J'
   % (#5,#6) = (Xmin,Ymin) "coin infrieur gauche"
   % (#7,#8) = (Xmax,Ymax) "coin suprieur droit"
   %
   % le repre qui gre le quadrillage est (O',I',J')
   % Chaque point devient un noeud.
   \pnode(#2){O'}%
   \pnode(#3){I'}%
   \pnode(#4){J'}%
   %
   % On calcule le nombre de traits "verticaux"   tracer
   % Il y en a Xmax - Xmin + 1 = #7 - #5 + 1
   \setcounter{compteur@xquadrillage}{#7}%
   \addtocounter{compteur@xquadrillage}{1}%
   \addtocounter{compteur@xquadrillage}{-#5}%
   % On calcule le nombre de traits "horizontaux"  tracer
   % Il y en a Ymax - Ymin + 1 = #8 - #6 + 1
   \setcounter{compteur@yquadrillage}{#8}%
   \addtocounter{compteur@yquadrillage}{1}%
   \addtocounter{compteur@yquadrillage}{-#6}%
   %
   % On dessine enfin le quadrillage %
   %
   % On dessine les traits "verticaux" forms par les deux points
   % de coordonnes (\i,Ymin) et (\i,Ymax) avec \i partant de Xmin
   % et sachant que le nombre de traits "verticaux"
   % est \value{compteur@xquadrillage}
   \multido{\i=#5+1}{\value{compteur@xquadrillage}}{%
      \pstOIJGeonode[PointSymbol=none,PointName=none](\i,#6){A}{O'}{I'}{J'}%
      \pstOIJGeonode[PointSymbol=none,PointName=none](\i,#8){B}{O'}{I'}{J'}%
      \pstLineAB[#1]{A}{B}}%
   %
   % On dessine les traits "horizontaux" form?s par les deux points
   % de coordonn?es (Xmin,\i) et (Xmax,\i) avec \i partant de Ymin
   % et sachant que le nombre de traits "horizontaux"
   % est \value{compteur@yquadrillage}   
   \multido{\i=#6+1}{\value{compteur@yquadrillage}}{%
      \pstOIJGeonode[PointSymbol=none,PointName=none](#5,\i){A}{O'}{I'}{J'}
      \pstOIJGeonode[PointSymbol=none,PointName=none](#7,\i){B}{O'}{I'}{J'}
      \pstLineAB[#1]{A}{B}}%   
   }% fin de \def@quad@option

% Pour la dfinition sans option, on prend \def@quad@option
% avec un paramtre optionnel vide.
\def\def@quad@sansOption(#1)(#2)(#3)(#4,#5)(#6,#7){%
   \def@quad@option[](#1)(#2)(#3)(#4,#5)(#6,#7)}

\makeatother
\newcommand{\perpbf}{\mathop{~\rule{1.5pt}{6pt}\rput(-0.025,0){\rule{6pt}{1.5pt}}}\nolimits~}
\newcommand{\parabf}{\mathop{~~\rput{60}(0,0.08){\rule{8pt}{1.5pt}}\rput{60}(0.15,0.08){\rule{8pt}{1.5pt}}}\nolimits~~}

\newcommand{\para}{\mathop{~~\rput{60}(0,0.08){\rule{8pt}{.5pt}}\rput{60}(0.1,0.08){\rule{8pt}{.5pt}}}\nolimits~~}


%########################################################################################
%########################### MACROS POUR LES DROITES GRADUES ###########################
%###########################         v 0.2 du 7/1/2008        ###########################
%########################################################################################
%
% Auteur : Christian TELLECHEA
% Site   : www.Bricamath.net
%
\newcommand*{\Position}[1]{\ifdim #1cm=0cm c \fi \ifdim #1cm>0cm B \fi \ifdim #1cm<0cm t \fi}
% _________________________________________________
%|                                                 |
%|      Marque un point sur la droite gradue      |
%|_________________________________________________|
% #1 optionnel : paisseur du point, par dfaut 5pt
\newcommand*{\MarquePoint}[2][5]{\psdots[dotstyle=x,dotsize=#1pt 0](! #2 0)}
% __________________________________________________________
%|                                                          |
%|      Marque un trait vertical sur la droite gradue      |
%|__________________________________________________________|
% #1 optionnel : hauteur du trait de part et d'autre de la droite. Par defaut 0.25cm
\newcommand*{\MarqueTrait}[2][0.25]{\psline[linewidth=1pt](! #2 -#1)(! #2 #1)}
% _________________________________________________________________
%|                                                                 |
%|      Affiche un texte une abscisse et une hauteur donne      |
%|_________________________________________________________________|
%#1 : abscisse
%#2 : decalage vertical (espace en cm entre le haut ou la Bas de la lettre et la droite gradue)
%#3 : texte
\newcommand*{\AfficheTexte}[3]{\rput[\Position{#2}](! #1 #2){#3}}
% ________________________________________________________
%|                                                        |
%|             Affiche une fleche verticale               |
%|      au dessus ou au dessous de la droite gradue      |
%|                  avec ou sans texte                    |
%|________________________________________________________|
\makeatletter
\newcommand{\AfficheFleche}[3][0.15]{\@ifnextchar[{\AfficheFleche@i[#1]{#2}{#3}}{\AfficheFleche@i[#1]{#2}{#3}[]}}
\def\AfficheFleche@i[#1]#2#3[#4]{%
% #1  :   dcalage vertical entre Bout de la flche et la droite. Par dfaut=0.15
% #2  :   abscisse de la flche
% #3  :   hauteur de la flche en cm (positif:flche au dessus, ngatif:flche au dessous)
% #4  :   Optionnel : texte afficher
\pstVerB{/Signe #3 dup 0 eq {pop 0}{dup 0 lt {pop 1 neg}{pop 1} ifelse} ifelse def}% renvoie -1 0 ou 1 selon le signe de #3
\psline[arrowsize=2pt 3]{->}(! #2 #3 abs #1 abs add Signe mul)(!#2 #1 abs Signe mul)%
\rput[\Position{#3}](! #2 #3 abs #1 abs add 0.1 add Signe mul){#4}%
}
% ___________________________________________
%|                                           |
%|        ENVIRONNEMENT DroiteGradue        |
%|          Trace une droite gradue___________________________________________|
\newenvironment{DroiteGraduee}[8][all]{%
% Les arguments de l'environnement
% #1 : Optionnel = affichage des nombres. Par dfaut all (none est le contraire)
% #2 : largeur totale de la droite en cm
% #3 : abscisse de la 1re graduation
% #4 : abscisse de la dernire graduation
% #5 : nombre d'intervalles sudivisant la graduation principale
% #6 : Incrment entre 2 graduations principales
% #7 : nombre de graduations principales re
% #8 : nombre de graduations principales re
	\newdimen\x@gauche \newdimen\x@droite \newdimen\x@@unite
	\newdimen\c@rrection \newdimen\n@m \newdimen\d@n
	\x@@unite=1cm%       pour les conversions
	\c@rrection=0.6cm%   largeur de la flche
	\x@gauche=-#7cm \advance\x@gauche #3cm \x@droite=#4cm \advance\x@droite #8cm
	\pst@divide\x@gauche\x@@unite\x@@gauche%      conversion pt-->cm
	\n@m=-\c@rrection \advance\n@m #2cm%          n@m=largeur-epaisseur_fleche
	\d@n=-\x@gauche \advance\d@n\x@droite%        d@n=xdroite-xgauche
	\pst@divide\n@m\d@n\x@unite%                  x@unit=(largeur-largeur_fleche)/(xdroite-xgauche)
	\pst@divide\c@rrection{\x@unite cm}{\Fl@che}
	\newdimen\EpFl@che \EpFl@che=\Fl@che cm
	\advance\x@droite\EpFl@che \pst@divide\x@droite\x@@unite\x@@droite% on corrige de la largeur de la flche, et on convertit
	\psset{xunit=\x@unite cm,yunit=1cm,arrowsize=4pt 3}%
	\begin{pspicture}(\x@@gauche,-1)(\x@@droite,1)%
		\psaxes[comma,labelsep=5pt,labels=#1,Ox=#3,Dx=#6,yAxis=false,subticks=#5,ticksize=-0.15cm 0.15cm,tickwidth=1.5pt,subticksize=0.7,subtickwidth=0.8pt,subtickcolor=black]{->}(#3,0)(\x@@gauche,-1)(\x@@droite,1)
	}
	{\end{pspicture}}

\newenvironment{Thermometre}[8][all]{%
% Les arguments de l'environnement
% #1 : Optionnel = affichage des nombres. Par dfaut all (none est le contraire)
% #2 : largeur totale de la droite en cm
% #3 : abscisse de la 1re graduation
% #4 : abscisse de la dernire graduation
% #5 : nombre d'intervalles sudivisant la graduation principale
% #6 : Incrment entre 2 graduations principales
% #7 : nombre de graduations principales re
% #8 : nombre de graduations principales re
	\newdimen\x@gauche \newdimen\x@droite \newdimen\x@@unite
	\newdimen\c@rrection \newdimen\n@m \newdimen\d@n
	\x@@unite=1cm%       pour les conversions
	\c@rrection=0.6cm%   largeur de la flche
	\x@gauche=-#7cm \advance\x@gauche #3cm \x@droite=#4cm \advance\x@droite #8cm
	\pst@divide\x@gauche\x@@unite\x@@gauche%      conversion pt-->cm
	\n@m=-\c@rrection \advance\n@m #2cm%          n@m=largeur-epaisseur_fleche
	\d@n=-\x@gauche \advance\d@n\x@droite%        d@n=xdroite-xgauche
	\pst@divide\n@m\d@n\x@unite%                  x@unit=(largeur-largeur_fleche)/(xdroite-xgauche)
	\pst@divide\c@rrection{\x@unite cm}{\Fl@che}
	\newdimen\EpFl@che \EpFl@che=\Fl@che cm
	\advance\x@droite\EpFl@che \pst@divide\x@droite\x@@unite\x@@droite% on corrige de la largeur de la flche, et on convertit
	\psset{xunit=\x@unite cm,yunit=1cm,arrowsize=4pt 3}%
	\begin{pspicture}(\x@@gauche,-1)(\x@@droite,1)%
		\psaxes[comma,labelsep=5pt,labels=#1,Ox=#3,Dx=#6,yAxis=false,subticks=#5,ticksize=-0.4cm 0.4cm,tickwidth=1.5pt,subticksize=0.7,subtickwidth=0.8pt,subtickcolor=black]{-}(#3,0)(\x@@gauche,-1)(\x@@droite,1)
	}
	{\end{pspicture}
}

\newenvironment{DroiteGradueequad}[8][all]{%
% Les arguments de l'environnement
% #1 : Optionnel = affichage des nombres. Par dfaut all (none est le contraire)
% #2 : largeur totale de la droite en cm
% #3 : abscisse de la 1re graduation
% #4 : abscisse de la dernire graduation
% #5 : nombre d'intervalles sudivisant la graduation principale
% #6 : Incrment entre 2 graduations principales
% #7 : nombre de graduations principales re
% #8 : nombre de graduations principales afficher aprs la derni�re
	\newdimen\x@gauche \newdimen\x@droite \newdimen\x@@unite
	\newdimen\c@rrection \newdimen\n@m \newdimen\d@n
	\x@@unite=1cm%       pour les conversions
	\c@rrection=0.6cm%   largeur de la flche
	\x@gauche=-#7cm \advance\x@gauche #3cm \x@droite=#4cm \advance\x@droite #8cm
	\pst@divide\x@gauche\x@@unite\x@@gauche%      conversion pt-->cm
	\n@m=-\c@rrection \advance\n@m #2cm%          n@m=largeur-epaisseur_fleche
	\d@n=-\x@gauche \advance\d@n\x@droite%        d@n=xdroite-xgauche
	\pst@divide\n@m\d@n\x@unite%                  x@unit=(largeur-largeur_fleche)/(xdroite-xgauche)
	\pst@divide\c@rrection{\x@unite cm}{\Fl@che}
	\newdimen\EpFl@che \EpFl@che=\Fl@che cm
	\advance\x@droite\EpFl@che \pst@divide\x@droite\x@@unite\x@@droite% on corrige de la largeur de la flche, et on convertit
	\quadrillageTEST[linecolor=green](0,-1)(0,-1.1)(0.1,-1)(-5,-3)(17,90)
	\psset{xunit=1cm,yunit=1cm,arrowsize=4pt 3}%
 
	\begin{pspicture}(-1,1.35)(0,0)%
	%\x@@gauche \x@@droite
		\psaxes
		[comma,labelsep=5pt,labels=#1,Ox=#3,Dx=#6,yAxis=false,subticks=#5,ticksize=-0.15cm 0.15cm,tickwidth=1.5pt,subticksize=0.7,subtickwidth=0.8pt,subtickcolor=black]{->}(#3,0)(\x@@gauche,-1)(\x@@droite,1)
	}
	{\end{pspicture}
}



\newenvironment{DroiteGradueeangle}[9][all]{%A UTILISER DANS un PSPICTURE pas comme un droitegraduee traditionnel
% Les arguments de l'environnement
% #1 : Optionnel = affichage des nombres. Par daut all (none est le contraire)
% #2 : largeur totale de la droite en cm
% #3 : abscisse de la 1re graduation
% #4 : abscisse de la dernire graduation
% #5 : nombre d'intervalles sudivisant la graduation principale
% #6 : Incrment entre 2 graduations principales
% #7 : nombre de graduations principales afficher avant la premire
% #8 : nombre de graduations principales afficher aprs la dernire
	\newdimen\x@gauche \newdimen\x@droite \newdimen\x@@unite
	\newdimen\c@rrection \newdimen\n@m \newdimen\d@n
	\x@@unite=1cm%       pour les conversions
	\c@rrection=0.6cm%   largeur de la flche
	\x@gauche=-#7cm \advance\x@gauche #3cm \x@droite=#4cm \advance\x@droite #8cm
	\pst@divide\x@gauche\x@@unite\x@@gauche%      conversion pt-->cm
	\n@m=-\c@rrection \advance\n@m #2cm%          n@m=largeur-epaisseur_fleche
	\d@n=-\x@gauche \advance\d@n\x@droite%        d@n=xdroite-xgauche
	\pst@divide\n@m\d@n\x@unite%                  x@unit=(largeur-largeur_fleche)/(xdroite-xgauche)
	\pst@divide\c@rrection{\x@unite cm}{\Fl@che}
	\newdimen\EpFl@che \EpFl@che=\Fl@che cm
	\advance\x@droite\EpFl@che \pst@divide\x@droite\x@@unite\x@@droite% on corrige de la largeur de la flche, et on convertit
	%\psset{xunit=1cm,yunit=1cm,arrowsize=4pt 3}%
	%\x@@gauche \x@@droite
		\rput{#9}(0,0){
	
		\psaxes
		[comma,labelsep=5pt,labels=#1,Ox=#3,Dx=#6,yAxis=false,subticks=#5,ticksize=-0.15cm 0.15cm,tickwidth=1.5pt,subticksize=0.7,subtickwidth=0.8pt,subtickcolor=black]{->}(#3,0)(\x@@gauche,-1)(\x@@droite,1)}
		
	}
	{
}
	
% ___________________________________________________
%|                                                   |
%|      Reprsentation graphique de l'ensemBle       |
%|    des solutions d'une inquation du 1degr
%|___________________________________________________|
%
\makeatletter
\newcommand*{\GraphiqueInequation}[3][n]{\@ifnextchar[{\GraphiqueInequation@i[#1]{#2}{#3}}{\GraphiqueInequation@i[#1]{#2}{#3}[5]}}
\def\GraphiqueInequation@i[#1]#2#3[#4]{%
% #1   :   optionnel = 0 par dfaut (pas de hachure pour ce qui ne convient pas) sinon #1=H ou h
% #2   :   signe de l'in  < ou <= ou =< ou > ou >= ou =>
% #3   :   nomafficher au dessus de la borne
% #4   :   optionnel = largeur de la reprsentation en cm. Par dfaut 5.
	\newdimen\xMaxi \xMaxi=#4cm \divide\xMaxi2
	\psset{unit=1 cm,arrowsize=4pt 3}
	\begin{pspicture*}(-\xMaxi,-0.3)(\xMaxi,1.2)%
		\pstVerB{%out code PostScript
			/DecalCrochet 0.07 def /DecalOrigine 0.07 def /EpFleche 0.2 def /Angle 0 def
			/Inferieur (#2)(<)eq (#2)(<=)eq or (#2)(=<)eq or def
			/Superieur (#2)(>)eq (#2)(>=)eq or (#2)(=>)eq or def
			/Large (#2)(<=)eq (#2)(=<)eq or (#2)(>=)eq or (#2)(=>)eq or def
			/Stricte (#2)(<)eq (#2)(>)eq or def
			Superieur {/Signe 1 def} if
			Inferieur	{/Signe 1 neg def} if
			(#1)(H)eq (#1)(h)eq or {/yHachures 0.15 def}{/yHachures 0 def} ifelse
			Large {
				/TaillePoint 0.075 def /TailleTrait 0 def /xOrigine 0 def
				/xCrochet DecalCrochet Signe neg mul def /xHachures xCrochet def
				Inferieur {/Angle 180 def} if
				} if
			Stricte {
				/TaillePoint 0 def /TailleTrait 0.15 def /xOrigine DecalOrigine def
				/xCrochet DecalCrochet Signe mul def /xHachures 0 def
				Superieur {/Angle 180 def} if
				} if
			/xExtremeHachures #4 2 div Signe neg mul def
			/xExtreme #4 2 div Signe mul def
			Superieur {/xExtreme xExtreme EpFleche sub def} if
			Inferieur {/xExtremeHachures xExtremeHachures 2 EpFleche mul sub def} if
			/xOrigine xOrigine Signe mul def
		}%% fin code PostScript
	\psline[linewidth=1pt]{->}(! #4 2 div neg 0)(! #4 2 div 0)%      trac de la droite
	\psline[linewidth=2.5pt](! xExtreme 0)(! xOrigine 0)%            on repasse en gras
	\psellipse*(! 0 0)(! TaillePoint TaillePoint)%                   marque ventuellement le point
	\psline[linewidth=1pt](! 0 TailleTrait neg)(! 0 TailleTrait)%    trace ventuellement un trait
	\rput[c]{! Angle}(! xCrochet 0){\LARGE\textbf{[}}%               on place le crochet
	\rput[B](! 0 0.4){#3}%                                           affichage de la borne
	\psframe[linestyle=none,fillstyle=hlines,hatchwidth=0.5pt,hatchsep=3pt](! xHachures yHachures neg)(! xExtremeHachures yHachures)% hachures
	\end{pspicture*}
}
\makeatother
\newcommand{\carrecercle}[7]{%#1coordx en cm#2 corrd y #3 dimension carre #4567 lettre sur les nodes dans les cercles
\psset{PointName=none,PointSymbol=none}
\small
\pstGeonode[PointName=none, PointSymbol=none](0,0){X}
\pstGeonode[PointName=none, PointSymbol=none](#3,0){Y}
\pstGeonode[PointName=none, PointSymbol=none](#1,#2){A}
\pstTranslation{X}{Y}{A}[B]
\pstRotation[RotAngle=90]{A}{B}[D]
\pstRotation[RotAngle=-90]{B}{A}[C]
\pstMiddleAB{A}{B}{I}
\uput{4pt}[90](I){#4}
\pstMiddleAB{A}{I}{I'}
\pstMiddleAB{I}{B}{I''}
\pstMiddleAB{B}{C}{J}
\uput{4pt}[180](J){#5}
\pstMiddleAB{B}{J}{J'}
\pstMiddleAB{J}{C}{J''}
\pstMiddleAB{C}{D}{K}
\uput{4pt}[270](K){#6}
\pstMiddleAB{C}{K}{K'}
\pstMiddleAB{K}{D}{K''}
\pstMiddleAB{D}{A}{L}
\uput{4pt}[0](L){#7}
\pstMiddleAB{L}{A}{L'}
\pstMiddleAB{D}{L}{L''}
\pstArcnOAB{I}{I'}{I''}
\pstArcnOAB{J}{J'}{J''}
\pstArcnOAB{K}{K'}{K''}
\pstArcOAB{L}{L'}{L''}
\pspolygon(A)(B)(C)(D)
}

\newcommand{\carrecerclecoul}[7]{%#1coordx en cm#2 corrd y #3 dimension carre #4567 lettre sur les nodes dans les cercles #891011 les couleurs des demis cercles
\psset{PointName=none,PointSymbol=none}
%\newlength{\xcoordeb}
%\setlength{\xcoordeb}{0cm}
%\addtolength{\xcoordeb}{#1}
%\newlength{\ycoordeb}
%\setlength{\ycoordeb}{0cm}
%\addtolength{\xcoordeb}{#2}
\small
\pstGeonode[PointName=none, PointSymbol=none](0,0){X}
\pstGeonode[PointName=none, PointSymbol=none](#3,0){Y}
\pstGeonode[PointName=none, PointSymbol=none](#1,#2){A}
\pstTranslation{X}{Y}{A}[B]
\pstRotation[RotAngle=90]{A}{B}[D]
\pstRotation[RotAngle=-90]{B}{A}[C]
\pstMiddleAB{A}{B}{I}

\pstMiddleAB{A}{I}{I'}
\pstMiddleAB{I}{B}{I''}
\pstMiddleAB{B}{C}{J}

\pstMiddleAB{B}{J}{J'}
\pstMiddleAB{J}{C}{J''}
\pstMiddleAB{C}{D}{K}

\pstMiddleAB{C}{K}{K'}
\pstMiddleAB{K}{D}{K''}
\pstMiddleAB{D}{A}{L}

\pstMiddleAB{L}{A}{L'}
\pstMiddleAB{D}{L}{L''}
\pstArcnOAB[fillstyle=solid,fillcolor=white]{I}{I'}{I''}
\pstArcnOAB[fillstyle=solid,fillcolor=white]{J}{J'}{J''}
\pstArcnOAB[fillstyle=solid,fillcolor=white]{K}{K'}{K''}
\pstArcOAB[fillstyle=solid,fillcolor=white]{L}{L'}{L''}
\uput{4pt}[90](I){#4}
\uput{4pt}[180](J){#5}
\uput{4pt}[270](K){#6}
\uput{4pt}[0](L){#7}
\pspolygon(A)(B)(C)(D)
}
%%%%%%%%%%%%%%%%%%%%%%%%%%%%%%%%%%%%% reponse avec cadre et pointill\'e
%autant de ldots qu'on veut
\newcommand{\rep}[1]{
%paremetre = nombre de \ldots ATTENTION MATHMODE NECESSAIRE
\fcolorbox{white}{white}{
\rule{0cm}{0.3cm}\hspace{-0.3cm}
\multido{\n=1+1}{#1}{
\ldots\hspace{-0.235cm}
}
}
}
%
%#####################################################################################
%######################## MACROS : SOLIDES \`a SURFACES COURBES ########################
%########################         v 0.5 du 05/03/2008         ########################
%#####################################################################################
%
% Auteur : Christian TELLECHEA
% Site   : www.bricamath.net
\makeatletter

\edef\dashDefaut{3pt 3pt}%   pointill\'es par d\'efaut
\edef\widthDefaut{0.7pt}%    \'epaisseur par d\'efaut

% _________________________________________________
%|                                                 |
%|           Cotation et noms des points           |
%|_________________________________________________|
\define@key{TraitRappel}{styleTraitRappel}{\edef\@styleTraitRappel{#1}}
\define@key{TraitRappel}{widthTraitRappel}{\edef\@widthTraitRappel{#1}}
\define@key{TraitRappel}{dashTraitRappel}{\edef\@dashTraitRappel{#1}}
\newcommand*{\@TraitRappel}[7][]{%
% #1	style des Traits de Rappel
% #2	abscisse d\'ebut       % #3 ordonn\'ee d\'ebut
% #4	abscisse fin         % #5 ordonn\'ee fin
% #6	d\'epassement trait
% #7	d\'ecalage fl\`eche
	\presetkeys{TraitRappel}{styleTraitRappel=none,widthTraitRappel=0.4pt,dashTraitRappel=1.5pt 1.5pt}{}%
	\setkeys{TraitRappel}{#1}%
	\@@TraitRappel{#2}{#3}{#4}{#5}{#6}{#7}%
}
\newcommand{\@@TraitRappel}[6]{%
	\pstVerb{%
		/signe {dup 0 eq {pop 0}{dup 0 lt {pop 1 neg}{pop 1} ifelse} ifelse} def%
		/depass #5 #6 abs add #6 signe mul #3 #1 sub dup mul #4 #2 sub dup mul add sqrt div def%
		/decalx #2 #4 sub depass mul def%
		/decaly #3 #1 sub depass mul def}%
	\begingroup%
		\psset{linestyle=\@styleTraitRappel,dash=\@dashTraitRappel,linewidth=\@widthTraitRappel}%
		\psline(! #1 #2)(! #1 decalx add #2 decaly add)%
		\psline(! #3 #4)(! #3 decalx add #4 decaly add)%
	\endgroup%
}

\define@key{Cotation}{styleTraitRappel}{\edef\@styleTraitRappel{#1}}
\define@key{Cotation}{widthTraitRappel}{\edef\@widthTraitRappel{#1}}
\define@key{Cotation}{dashTraitRappel}{\edef\@dashTraitRappel{#1}}
\define@key{Cotation}{styleFleche}{\edef\@styleFleche{#1}}
\define@key{Cotation}{dashFleche}{\edef\@dashFleche{#1}}
\define@key{Cotation}{widthFleche}{\edef\@widthFleche{#1}}
\define@key{Cotation}{decalFleche}{\edef\@decalFleche{#1}}
\define@key{Cotation}{decalTexte}{\edef\@decalTexte{#1}}
\define@key{Cotation}{depassTrait}{\edef\@depassTrait{#1}}
\define@key{Cotation}{rotationTexte}{\edef\@rotationTexte{#1}}
\newcommand*{\Cotation}{\@ifstar{\def\@@put{\lput*}\@cotation@}{\def\@@put{\aput}\@cotation@}}%
\newcommand*{\@cotation@}[1][]{\@cotation@i[#1]}%
\def\@cotation@i[#1](#2,#3)(#4,#5)#6{%
%               [styles](coordonn\'ees d\'ebut)(coordonn\'ees fin){Texte}
	\presetkeys{Cotation}{%
		styleTraitRappel=dashed,dashTraitRappel=1.5pt 1.5pt,widthTraitRappel=0.4pt,%
		styleFleche=solid,dashFleche=\dashDefaut,widthFleche=0.7pt,%
		decalFleche=0.5,%
		decalTexte=0.5ex,%
		depassTrait=0.15,%
		rotationTexte=:U}{}%
	\setkeys{Cotation}{#1}%
	\@@TraitRappel{#2}{#3}{#4}{#5}{\@depassTrait}{\@decalFleche}%
	\pcline[offset=\@decalFleche,linestyle=\@styleFleche,dash=\@dashFleche,linewidth=\@widthFleche]{<->}(! #2 #3)(! #4 #5)% trac\'e de la fl\`eche
	\setlength{\pslabelsep}{\@decalTexte}% d\'efinit la osition du texte avec le d\'ecalage
	\@@put{\@rotationTexte}{#6}% affichage du texte
}

\newcommand*{\@SiMoins}[1]{% est ce que #1 commence par un - ?
	\def\@TesteAvantMoins##1-##2\@nil{%
		\ifx\@empty##1\@empty
			\expandafter\@firstoftwo%
		\else%
			\expandafter\@secondoftwo%
		\fi%
	}%
	\@TesteAvantMoins#1-\@nil%
}

\newcommand*{\@MoinsEnNeg}[1]{%
% si #1 est de la forme "-4" renvoie "4 neg"
	\@SiMoins{#1}%
		{\def\@@MoinsEnNeg-##1\@nil{##1 neg}}% si n\'egatif -> conversion
		{\def\@@MoinsEnNeg##1\@nil{##1}}% sinon -> ne rien faire
	\edef\@Resultat@MoinsEnNeg{\@@MoinsEnNeg#1\@nil}% on transmet le r\'esultat dans \@Resultat@MoinsEnNeg
}

\newcommand*{\@PlaceNom}[6]{%
% #1 : nom du point
% #2 : point de r\'ef\'erence
% #3 et #4 : abscisse et ordonn\'ee
% #5 et #6 : d\'ecalages horizontal et vertical
	\@MoinsEnNeg{#5}%
	\edef\@DecalH{\@Resultat@MoinsEnNeg}%
	\@MoinsEnNeg{#6}%
	\edef\@DecalV{\@Resultat@MoinsEnNeg}% #5 et #6 sont convertis en postscript et donc,
	\@@PlaceNom{#1}{#2}{#3}{#4}{\@DecalH}{\@DecalV}% on appelle la macro
}

\newcommand{\@@PlaceNom}[6]{%
	\rput[#2](! #3 #5 add #4 #6 add){#1}%
}

\define@key{NomPoints}{sizePointCentre}{\edef\@sizePointCentre{#1}}
\define@key{NomPoints}{HdecalNomCentre}{\edef\@HdecalNomCentre{#1}}
\define@key{NomPoints}{VdecalNomCentre}{\edef\@VdecalNomCentre{#1}}
\define@key{NomPoints}{HdecalNomSommet}{\edef\@HdecalNomSommet{#1}}
\define@key{NomPoints}{VdecalNomSommet}{\edef\@VdecalNomSommet{#1}}
\define@key{NomPoints}{HdecalNomGauche}{\edef\@HdecalNomGauche{#1}}
\define@key{NomPoints}{VdecalNomGauche}{\edef\@VdecalNomGauche{#1}}
\define@key{NomPoints}{HdecalNomDroite}{\edef\@HdecalNomDroite{#1}}
\define@key{NomPoints}{VdecalNomDroite}{\edef\@VdecalNomDroite{#1}}
\newcommand*{\NomPoints}[8][]{%
% Affiche les noms des points remarquables du c‚\`aö¬•ne, du Cylindre ou  de la sph\`ere
% #1 : options
% #2 : rayon                  R
% #3 : coeff d'applatissement k
% #4 : Hauteur                H
% #5 : Nom de l'extr\'emit\'e diam\`etre gauche
% #6 : Nom du centre du cercle
% #7 : Nom de l'extr\'emit\'e diam\`etre droit
% #8 : Nom du sommet
	\ifdim #4cm<0cm%
		\edef\@PC{b}%
		\edef\@PS{t}%
	\else%
		\edef\@PC{br}%
		\edef\@PS{b}%
	\fi%
	\presetkeys{NomPoints}{%
		sizePointCentre=3pt,%
		HdecalNomGauche=0,VdecalNomGauche=0,%
		HdecalNomCentre=0,VdecalNomCentre=0,%
		HdecalNomDroite=0,VdecalNomDroite=0,%
		HdecalNomSommet=0,VdecalNomSommet=0}{}
	\setkeys{NomPoints}{#1}%
	\@PlaceNom{#5}{r}{#2 abs 0.2 add neg}{0}{\@HdecalNomGauche}{\@VdecalNomGauche}% nom gauche
	\@PlaceNom{#6}{\@PC}{#4 0 lt{0}{0.1 neg} ifelse}{0.1}{\@HdecalNomCentre}{\@VdecalNomCentre}% nom centre
	\@PlaceNom{#7}{l}{#2 abs 0.2 add}{0}{\@HdecalNomDroite}{\@VdecalNomDroite}% nom droite
	\@PlaceNom{#8}{\@PS}{0}{#4 0 lt{#4 abs 0.2 add neg}{#4 0.2 add} ifelse}{\@HdecalNomSommet}{\@VdecalNomSommet}% nom sommet
	\psdots[dotsize=\@sizePointCentre](0,0)% le point repr\'esentant le centre
}

\newcommand*{\@AngleDroit}[5][1]{%
% [signe]{style}{\'epaisseur}{taille}
	\pstVerb{/yy {#1 0 lt {#5 neg}{#5} ifelse} def}
	\psline[linestyle=#2,dash=#3,linewidth=#4](! #5 0)(! #5 yy)(! 0 yy)
}

% _________________________________________________
%|                                                 |
%|                     Ellipse                     |
%|_________________________________________________|







\define@key{Ellipse}{styleHautEllipse}{\edef\@styleHautEllipse{#1}}%
\define@key{Ellipse}{dashHautEllipse}{\edef\@dashHautEllipse{#1}}%
\define@key{Ellipse}{widthHautEllipse}{\edef\@widthHautEllipse{#1}}%
\define@key{Ellipse}{styleBasEllipse}{\edef\@styleBasEllipse{#1}}%
\define@key{Ellipse}{dashBasEllipse}{\edef\@dashBasEllipse{#1}}%
\define@key{Ellipse}{widthBasEllipse}{\edef\@widthBasEllipse{#1}}%
\newcommand*{\Ellipse}[4][]{%
% #1	styles
% #2	ordonn\'ee de coupure (d\'elimite 2 parties de l'Ellipse en styles de Trait diff\'erents)
% #3	demi Axe horizontal >0
% #4 demi Axe vertical >0
	\presetkeys{Ellipse}{%
		styleHautEllipse=dashed,dashHautEllipse=\dashDefaut,widthHautEllipse=\widthDefaut,%
		styleBasEllipse=solid,dashBasEllipse=\dashDefaut,widthBasEllipse=\widthDefaut}{}%
	\setkeys{Ellipse}{#1}%
	\@Ellipse{#2}{#3}{#4}%
}
\newcommand{\@Ellipse}[3]{%
	\pstVerb{%
		/AAsin {dup 1 gt {pop 90}{dup 1 neg lt {pop 90 neg}{asin} ifelse} ifelse} def%
		/AAngl #1 #3 div AAsin def}%
	\psellipticarc[linestyle=\@styleHautEllipse,dash=\@dashHautEllipse,linewidth=\@widthHautEllipse](0,0)(! #2 #3){! AAngl}{! 180 AAngl sub}% partie Haute
	\psellipticarc[linestyle=\@styleBasEllipse,dash=\@dashBasEllipse,linewidth=\@widthBasEllipse](0,0)(! #2 #3){! AAngl 180 add neg}{! AAngl}% partie Basse
}


%%%%%%%%%GregEllipse
\define@key{GregEllipse}{styleHautGregEllipse}{\edef\@styleHautGregEllipse{#1}}%
\define@key{GregEllipse}{dashHautGregEllipse}{\edef\@dashHautGregEllipse{#1}}%
\define@key{GregEllipse}{widthHautGregEllipse}{\edef\@widthHautGregEllipse{#1}}%
\define@key{GregEllipse}{styleBasGregEllipse}{\edef\@styleBasGregEllipse{#1}}%
\define@key{GregEllipse}{dashBasGregEllipse}{\edef\@dashBasGregEllipse{#1}}%
\define@key{GregEllipse}{widthBasGregEllipse}{\edef\@widthBasGregEllipse{#1}}%
\newcommand*{\GregEllipse}[4][]{%
% #1	styles
% #2	ordonn\'ee de coupure (d\'elimite 2 parties de l'Ellipse en styles de Trait diff\'erents)
% #3	demi Axe horizontal >0
% #4 demi Axe vertical >0
	\presetkeys{GregEllipse}{%
		styleHautGregEllipse=solid,dashHautGregEllipse=\dashDefaut,widthHautGregEllipse=\widthDefaut,%
		styleBasGregEllipse=solid,dashBasGregEllipse=\dashDefaut,widthBasGregEllipse=\widthDefaut}{}%
	\setkeys{GregEllipse}{#1}%
	\@GregEllipse{#2}{#3}{#4}%
}
\newcommand{\@GregEllipse}[3]{%
	\pstVerb{%
		/AAsin {dup 1 gt {pop 90}{dup 1 neg lt {pop 90 neg}{asin} ifelse} ifelse} def%
		/AAngl #1 #3 div AAsin def}%
	\psellipticarc[linestyle=\@styleHautGregEllipse,dash=\@dashHautGregEllipse,linewidth=\@widthHautGregEllipse](0,0)(! #2 #3){! AAngl}{! 180 AAngl sub}% partie Haute
	\psellipticarc[linestyle=\@styleBasGregEllipse,dash=\@dashBasGregEllipse,linewidth=\@widthBasGregEllipse](0,0)(! #2 #3){! AAngl 180 add neg}{! AAngl}% partie Basse
}

% _________________________________________________
%|                                                 |
%|                        Cone                     |
%|_________________________________________________|

\define@key{Cone}{styleHautCone}{\edef\@styleHautEllipse{#1}}%
\define@key{Cone}{dashHautCone}{\edef\@dashHautEllipse{#1}}%
\define@key{Cone}{widthHautCone}{\edef\@widthHautEllipse{#1}}%
\define@key{Cone}{styleBasCone}{\edef\@styleBasEllipse{#1}}%
\define@key{Cone}{dashBasCone}{\edef\@dashBasEllipse{#1}}%
\define@key{Cone}{widthBasCone}{\edef\@widthBasEllipse{#1}}%
\define@key{Cone}{styleGeneratriceCone}{\edef\@styleGeneratriceCone{#1}}%
\define@key{Cone}{dashGeneratriceCone}{\edef\@dashGeneratriceCone{#1}}%
\define@key{Cone}{widthGeneratriceCone}{\edef\@widthGeneratriceCone{#1}}%
\newcommand*{\Cone}[4][]{%
% #1	styles
% #2	rayon R
% #3	coeff d'aplatissement k
% #4	Hauteur du c‚\`aö¬•ne H
	\ifdim#4pt<0pt%
		\def\@@styleHautCone{solid}%  si H<0, l'ellipse est en trait plein
	\else%
		\def\@@styleHautCone{dashed}% si H>0, partie haute en pointill\'es
	\fi%
	\presetkeys{Cone}{%
		styleHautCone=\@@styleHautCone,dashHautCone=\dashDefaut,widthHautCone=\widthDefaut,%
		styleBasCone=solid,dashBasCone=\dashDefaut,widthBasCone=\widthDefaut,%
		styleGeneratriceCone=solid,dashGeneratriceCone=\dashDefaut,widthGeneratriceCone=\widthDefaut}{}%
	\setkeys{Cone}{#1}%
	\pstVerb{%
		/bb #2 #3 mul def%                      bb:=kR             : demi grand Axe vertical
		/yy bb dup mul #4 div def%              yy:=R^2*k^2/H      : ordonn\'ee du point de tangence
		/xx 1 yy #4 div sub sqrt #2 mul def%    xx:=R*sqrt(1-yy/H) : abscisse du point de tangence
	}%
	\@Ellipse{yy}{#2}{bb}% trac\'e de la Base
	\psline[linestyle=\@styleGeneratriceCone,dash=\@dashGeneratriceCone,linewidth=\@widthGeneratriceCone](! xx neg yy)(! 0 #4)(! xx yy)% les g\'en\'eratrices
}

\define@key{SectionCone}{styleHautSectionCone}{\edef\@styleHautEllipse{#1}}%
\define@key{SectionCone}{dashHautSectionCone}{\edef\@dashHautEllipse{#1}}%
\define@key{SectionCone}{widthHautSectionCone}{\edef\@widthHautEllipse{#1}}%
\define@key{SectionCone}{styleBasSectionCone}{\edef\@styleBasEllipse{#1}}%
\define@key{SectionCone}{dashBasSectionCone}{\edef\@dashBasEllipse{#1}}%
\define@key{SectionCone}{widthBasSectionCone}{\edef\@widthBasEllipse{#1}}%
\newcommand*{\SectionCone}[5][]{%
% #1 : styles
% #2 : rayon                  R
% #3 : coeff d'applatissement k
% #4 : Hauteur                H
% #5 : ordonn\'ee de la section z (0<z<H)
	\presetkeys{SectionCone}{%
		styleHautSectionCone=dashed,dashHautSectionCone=\dashDefaut,widthHautSectionCone=\widthDefaut,%
		styleBasSectionCone=solid,dashBasSectionCone=\dashDefaut,widthBasSectionCone=\widthDefaut%
		}{}%
	\setkeys{SectionCone}{#1}%
	\pstVerb{%
		/rr #2 #4 #5 sub mul #4 div def%    rr = R*(H-z)/H) : rayon de la section
		/bb #3 rr mul def%                  bb:=k*rr        : demi grand Axe vertical
		/yy bb dup mul #4 div def%          yy:=bb^2/H      : ordonn\'ee du point de tangence
	}%
	\rput(! 0 #5){\@Ellipse{yy}{rr}{bb}}%
}

\define@key{TroncCone}{styleTroncCone}{\edef\@styleTroncCone{#1}}
\define@key{TroncCone}{dashTroncCone}{\edef\@dashTroncCone{#1}}
\define@key{TroncCone}{widthTroncCone}{\edef\@widthTroncCone{#1}}
\define@key{TroncCone}{styleTraitCacheTroncCone}{\edef\@styleTraitCacheTroncCone{#1}}
\define@key{TroncCone}{dashTraitCacheTroncCone}{\edef\@dashTraitCacheTroncCone{#1}}
\define@key{TroncCone}{widthTraitCacheTroncCone}{\edef\@widthTraitCacheTroncCone{#1}}
\newcommand{\TroncCone}[5][]{\@ifnextchar[{\@TroncCone[#1]{#2}{#3}{#4}{#5}}{\@TroncCone[#1]{#2}{#3}{#4}{#5}[#3]}}%
\def\@TroncCone[#1]#2#3#4#5[#6]{%
% #1	styles
% #2	rayon disque bas                     R1
% #3	coeff d'aplatissement disque bas     k1
% #4	hauteur du c‚\`aö¬•ne virtuel total        H
% #5	hauteur du tronc de cone (|#3|<|#2|) h (du m\^eme signe que H)
% #6	coeff d'aplatissement disque haut    k2
	\presetkeys{TroncCone}{%
		styleTroncCone=solid,dashTroncCone=\dashDefaut,widthTroncCone=\widthDefaut,%
		styleTraitCacheTroncCone=dashed,dashTraitCacheTroncCone=\dashDefaut,widthTraitCacheTroncCone=\widthDefaut}{}%
	\setkeys{TroncCone}{#1}%
	\ifdim#4pt<0pt%
		\def\@styleHautGrandeEllise{\@styleTroncCone}%
		\def\@dashHautGrandeEllise{\@dashTroncCone}%
		\def\@widthHautGrandeEllise{\@widthTroncCone}%
		\def\@styleHautPetiteEllipse{\@styleTraitCacheTroncCone}%
		\def\@dashHautPetiteEllipse{\@dashTraitCacheTroncCone}%
		\def\@widthHautPetiteEllipse{\@widthTraitCacheTroncCone}%
	\else%
		\def\@styleHautGrandeEllise{\@styleTraitCacheTroncCone}%
		\def\@dashHautGrandeEllise{\@dashTraitCacheTroncCone}%
		\def\@widthHautGrandeEllise{\@widthTraitCacheTroncCone}%
		\def\@styleHautPetiteEllipse{\@styleTroncCone}%
		\def\@dashHautPetiteEllipse{\@dashTroncCone}%
		\def\@widthHautPetiteEllipse{\@widthTroncCone}%
	\fi%
	\pstVerb{%
		/bb #2 #3 mul def%                           bb :=k1*R1               : demi grand Axe vertical bas
		/yb bb dup mul #4 div def%                   yb :=R1^2*k1^2/H         : ordonn\'ee du point de tangence bas
		/xb 1 yb #4 div sub sqrt #2 mul def%         xb :=R1*sqrt(1-yb/H)     : abscisse du point de tangence bas
		/rh #2 #4 #5 sub mul #4 div def%             rh :=R1*(H-h)/H)         : rayon de la section haut
		/bh #6 rh mul def%                           bh :=k2*rh               : demi grand Axe vertical haut
		/yh bh dup mul #4 #5 sub div def%            yh :=bh^2/(H-h)          : ordonn\'ee du point de tangence haut
		/xh 1 yh #4 #5 sub div sub sqrt rh mul def%  xh:=rh*sqrt(1-yh/(H-h))  : abscisse du point de tangence haut
	}%
	\Ellipse[% grande ellipse
		styleHautEllipse=\@styleHautGrandeEllise,%
		dashHautEllipse=\@dashHautGrandeEllise,%
		widthHautEllipse=\@widthHautGrandeEllise,
		styleBasEllipse=\@styleTroncCone,%
		dashBasEllipse=\@dashTroncCone,%
		widthBasEllipse=\@widthTroncCone]{yb}{#2}{bb}%
	\rput(! 0 #5){%
		\Ellipse[% petite ellipse
			styleHautEllipse=\@styleHautPetiteEllipse,%
			dashHautEllipse=\@dashHautPetiteEllipse,%
			widthHautEllipse=\@widthHautPetiteEllipse,%
			styleBasEllipse=\@styleTroncCone,%
			dashBasEllipse=\@dashTroncCone,%
			widthBasEllipse=\@widthTroncCone]{yh}{rh}{bh}}%
	\begingroup%
		\psset{linestyle=\@styleTroncCone,dash=\@dashTroncCone,linewidth=\@widthTroncCone}%
		\psline(! xb yb)(! xh yh #5 add)% les g\'en\'eratrices
		\psline(! xb neg yb)(! xh neg yh #5 add)%
	\endgroup%
}

\define@key{AxesCone}{styleAxeCone}{\edef\@styleAxeCone{#1}}
\define@key{AxesCone}{dashAxeCone}{\edef\@dashAxeCone{#1}}
\define@key{AxesCone}{widthAxeCone}{\edef\@widthAxeCone{#1}}
\define@key{AxesCone}{styleDiametreCone}{\edef\@styleDiametreCone{#1}}
\define@key{AxesCone}{dashDiametreCone}{\edef\@dashDiametreCone{#1}}
\define@key{AxesCone}{widthDiametreCone}{\edef\@widthDiametreCone{#1}}
\define@key{AxesCone}{styleCodageCone}{\edef\@styleCodageCone{#1}}
\define@key{AxesCone}{dashCodageCone}{\edef\@dashCodageCone{#1}}
\define@key{AxesCone}{widthCodageCone}{\edef\@widthCodageCone{#1}}
\define@key{AxesCone}{tailleCodageCone}{\edef\@tailleCodageCone{#1}}
\newcommand*{\AxesCone}[4][]{%
% #1 : styles
% #2 : rayon                  R
% #3 : coeff d'applatissement k
% #4 : Hauteur                H
	\presetkeys{AxesCone}{%
		styleAxeCone=dashed,dashAxeCone=\dashDefaut,widthAxeCone=\widthDefaut,%
		styleDiametreCone=dashed,dashDiametreCone=\dashDefaut,widthDiametreCone=0.5pt,%
		styleCodageCone=solid,dashCodageCone=1.5pt 1.5pt,widthCodageCone=0.5pt,tailleCodageCone=0.2%
		}{}%
	\setkeys{AxesCone}{#1}%
	\@AxesCone{#2}{#3}{#4}%
}
\newcommand{\@AxesCone}[3]{%
	\psline[linestyle=\@styleAxeCone,dash=\@dashAxeCone,linewidth=\@widthAxeCone](! 0 0)(! 0 #3)% Axe
	\psline[linestyle=\@styleDiametreCone,dash=\@dashDiametreCone,linewidth=\@widthDiametreCone](! #1 neg 0)(! #1 0)% diam\`etre
	\@AngleDroit[#3]{\@styleCodageCone}{\@dashCodageCone}{\@widthCodageCone}{\@tailleCodageCone}%
}

\define@key{AxesSectionCone}{styleDiametreSectionCone}{\edef\@styleDiametreCone{#1}}
\define@key{AxesSectionCone}{dashDiametreSectionCone}{\edef\@dashDiametreCone{#1}}
\define@key{AxesSectionCone}{widthDiametreSectionCone}{\edef\@widthDiametreCone{#1}}
\define@key{AxesSectionCone}{styleCodageSectionCone}{\edef\@styleCodageCone{#1}}
\define@key{AxesSectionCone}{dashCodageSectionCone}{\edef\@dashCodageCone{#1}}
\define@key{AxesSectionCone}{widthCodageSectionCone}{\edef\@widthCodageCone{#1}}
\define@key{AxesSectionCone}{tailleCodageSectionCone}{\edef\@tailleCodageCone{#1}}
\newcommand*{\AxesSectionCone}[5][]{%
% #1 : styles
% #2 : rayon                  R
% #3 : coeff d'applatissement k
% #4 : Hauteur                H
% #5 : ordonn\'ee de la section z

	\edef\@styleAxeCone{none}% pas d'axe
	\edef\@dashAxeCone{\dashDefaut}% inutile puisque non trac\'e
	\edef\@widthAxeCone{\widthDefaut}% inutile puisque non trac\'e
	\presetkeys{AxesSectionCone}{%
		styleDiametreSectionCone=dashed,dashDiametreSectionCone=\dashDefaut,widthDiametreSectionCone=0.5pt,%
		styleCodageSectionCone=solid,dashCodageSectionCone=1pt 1pt,widthCodageSectionCone=0.5pt,tailleCodageSectionCone=0.2%
		}{}%
	\setkeys{AxesSectionCone}{#1}%
	\rput(! 0 #5){\@AxesCone{#2 #4 abs #5 abs sub mul #4 abs div}{#3}{#4}}%
}

\newcommand*{\NomSectionCone}[8][]{%
% #1 : options
% #2 : rayon                  R
% #3 : coeff d'applatissement k
% #4 : Hauteur                H
% #5 : ordonn\'ee de la section z
% #6 : Nom de l'extr\'emit\'e diam\`etre gauche
% #7 : Nom du centre du cercle
% #8 : Nom de l'extr\'emit\'e diam\`etre droit
	\rput(! 0 #5){%
		\NomPoints[#1]{#2 #4 abs #5 abs sub mul #4 abs div}{#3}{#4}{#6}{}{#8}{}% tous sauf le centre
		\NomPoints[#1]{1}{1}{1}{}{#7}{}{}% le centre
		}%
}

% _________________________________________________
%|                                                 |
%|                     Cylindre                    |
%|_________________________________________________|
\define@key{Cylindre}{styleCylindre}{\edef\@styleCylindre{#1}}
\define@key{Cylindre}{dashCylindre}{\edef\@dashCylindre{#1}}
\define@key{Cylindre}{widthCylindre}{\edef\@widthCylindre{#1}}
\define@key{Cylindre}{styleTraitCacheCylindre}{\edef\@styleTraitCacheCylindre{#1}}
\define@key{Cylindre}{dashTraitCacheCylindre}{\edef\@dashTraitCacheCylindre{#1}}
\define@key{Cylindre}{widthTraitCacheCylindre}{\edef\@widthTraitCacheCylindre{#1}}
\newcommand*{\Cylindre}[4][]{%
% #1	styles
% #2	rayon
% #3	coeff d'aplatissement
% #4	Hauteur
	\presetkeys{Cylindre}{%
		styleCylindre=solid,dashCylindre=\dashDefaut,widthCylindre=\widthDefaut,%
		styleTraitCacheCylindre=dashed,dashTraitCacheCylindre=\dashDefaut,widthTraitCacheCylindre=\widthDefaut}{}%
	\setkeys{Cylindre}{#1}%
	\pstVerb{/rr #2 abs def /hh #4 abs def /bb rr #3 mul def}%
	\Ellipse[%
		styleHautEllipse=\@styleTraitCacheCylindre,dashHautEllipse=\@dashTraitCacheCylindre,widthHautEllipse=\@widthCylindre,%
		styleBasEllipse=\@styleCylindre,dashBasEllipse=\@dashCylindre,widthBasEllipse=\@widthCylindre]{0}{rr}{bb}% la Base
	\rput(! 0 #4){%
		\Ellipse[%
			styleHautEllipse=\@styleCylindre,dashHautEllipse=\@dashCylindre,widthHautEllipse=\@widthCylindre,%
			styleBasEllipse=\@styleCylindre,dashHautEllipse=\@dashCylindre,widthBasEllipse=\@widthCylindre]{0}{rr}{bb}}% le Haut
	\begingroup%
		\psset{linestyle=\@styleCylindre,linewidth=\@widthCylindre}
		\psline(! rr neg 0)(! rr neg #4)%
		\psline(! rr 0)(! rr #4)% les g\'en\'eratrices
	\endgroup%
}

\define@key{AxeCylindre}{styleAxeCylindre}{\edef\@styleAxeCone{#1}}
\define@key{AxeCylindre}{dashAxeCylindre}{\edef\@dashAxeCone{#1}}
\define@key{AxeCylindre}{widthAxeCylindre}{\edef\@widthAxeCone{#1}}
\define@key{AxeCylindre}{styleDiametreCylindre}{\edef\@styleDiametreCone{#1}}
\define@key{AxeCylindre}{dashDiametreCylindre}{\edef\@dashDiametreCone{#1}}
\define@key{AxeCylindre}{widthDiametreCylindre}{\edef\@widthDiametreCone{#1}}
\define@key{AxeCylindre}{styleCodageCylindre}{\edef\@styleCodageCone{#1}}
\define@key{AxeCylindre}{dashCodageCylindre}{\edef\@dashCodageCone{#1}}
\define@key{AxeCylindre}{widthCodageCylindre}{\edef\@widthCodageCone{#1}}
\define@key{AxeCylindre}{tailleCodageCylindre}{\edef\@tailleCodageCone{#1}}
\newcommand{\AxesCylindre}[4][]{%
	\presetkeys{AxeCylindre}{%
		styleAxeCylindre=dashed,dashAxeCylindre=\dashDefaut,widthAxeCylindre=\widthDefaut,%
		styleDiametreCylindre=dashed,dashDiametreCylindre=\dashDefaut,widthDiametreCylindre=0.5pt,%
		styleCodageCylindre=solid,dashCodageCylindre=1.5pt 1.5pt,widthCodageCylindre=0.5pt,tailleCodageCylindre=0.2%
		}{}%
	\setkeys{AxeCylindre}{#1}%
	\@AxesCone{#2}{#3}{#4}%
	\rput(! 0 #4){%
		\@AngleDroit[-1]{\@styleCodageCone}{\@dashCodageCone}{\@widthCodageCone}{\@tailleCodageCone}%
		\psline[linestyle=\@styleDiametreCone,dash=\@dashDiametreCone,linewidth=\@widthDiametreCone](! #2 neg 0)(! #2 0)}% diam\`etre
}
%%%%%%%%%%%%%%%%%%Cylindre fond blanc
\define@key{GregCylindre}{styleGregCylindre}{\edef\@styleGregCylindre{#1}}
\define@key{GregCylindre}{dashGregCylindre}{\edef\@dashGregCylindre{#1}}
\define@key{GregCylindre}{widthGregCylindre}{\edef\@widthGregCylindre{#1}}
\define@key{GregCylindre}{styleTraitCacheGregCylindre}{\edef\@styleTraitCacheGregCylindre{#1}}
\define@key{GregCylindre}{dashTraitCacheGregCylindre}{\edef\@dashTraitCacheGregCylindre{#1}}
\define@key{GregCylindre}{widthTraitCacheGregCylindre}{\edef\@widthTraitCacheGregCylindre{#1}}
\newcommand*{\GregCylindre}[4][]{%
% #1	styles
% #2	rayon
% #3	coeff d'aplatissement
% #4	Hauteur
	\presetkeys{GregCylindre}{%
		styleGregCylindre=solid,dashGregCylindre=\dashDefaut,widthGregCylindre=\widthDefaut,%
		styleTraitCacheGregCylindre=dashed,dashTraitCacheGregCylindre=\dashDefaut,widthTraitCacheGregCylindre=\widthDefaut}{}%
	\setkeys{GregCylindre}{#1}%
	\pstVerb{/rr #2 abs def /hh #4 abs def /bb rr #3 mul def}%
	\GregEllipse[%
		styleHautGregEllipse=\@styleTraitCacheGregCylindre,dashHautGregEllipse=\@dashTraitCacheGregCylindre,widthHautGregEllipse=\@widthGregCylindre,%
		styleBasGregEllipse=\@styleGregCylindre,dashBasGregEllipse=\@dashGregCylindre,widthBasGregEllipse=\@widthGregCylindre]{0}{rr}{bb}% la Base
	\rput(! 0 #4){%
		\GregEllipse[%
			styleHautGregEllipse=\@styleGregCylindre,dashHautGregEllipse=\@dashGregCylindre,widthHautGregEllipse=\@widthGregCylindre,%
			styleBasGregEllipse=\@styleGregCylindre,dashHautGregEllipse=\@dashGregCylindre,widthBasGregEllipse=\@widthGregCylindre]{0}{rr}{bb}}% le Haut
	\begingroup%
		\psset{linestyle=\@styleGregCylindre,linewidth=\@widthGregCylindre}
		\psline(! rr neg 0)(! rr neg #4)%
		\psline(! rr 0)(! rr #4)% les g\'en\'eratrices
	\endgroup%
}

\define@key{AxeGregCylindre}{styleAxeGregCylindre}{\edef\@styleAxeCone{#1}}
\define@key{AxeGregCylindre}{dashAxeGregCylindre}{\edef\@dashAxeCone{#1}}
\define@key{AxeGregCylindre}{widthAxeGregCylindre}{\edef\@widthAxeCone{#1}}
\define@key{AxeGregCylindre}{styleDiametreGregCylindre}{\edef\@styleDiametreCone{#1}}
\define@key{AxeGregCylindre}{dashDiametreGregCylindre}{\edef\@dashDiametreCone{#1}}
\define@key{AxeGregCylindre}{widthDiametreGregCylindre}{\edef\@widthDiametreCone{#1}}
\define@key{AxeGregCylindre}{styleCodageGregCylindre}{\edef\@styleCodageCone{#1}}
\define@key{AxeGregCylindre}{dashCodageGregCylindre}{\edef\@dashCodageCone{#1}}
\define@key{AxeGregCylindre}{widthCodageGregCylindre}{\edef\@widthCodageCone{#1}}
\define@key{AxeGregCylindre}{tailleCodageGregCylindre}{\edef\@tailleCodageCone{#1}}
\newcommand{\AxesGregCylindre}[4][]{%
	\presetkeys{AxeGregCylindre}{%
		styleAxeGregCylindre=dashed,dashAxeGregCylindre=\dashDefaut,widthAxeGregCylindre=\widthDefaut,%
		styleDiametreGregCylindre=dashed,dashDiametreGregCylindre=\dashDefaut,widthDiametreGregCylindre=0.5pt,%
		styleCodageGregCylindre=solid,dashCodageGregCylindre=1.5pt 1.5pt,widthCodageGregCylindre=0.5pt,tailleCodageGregCylindre=0.2%
		}{}%
	\setkeys{AxeGregCylindre}{#1}%
	\@AxesCone{#2}{#3}{#4}%
	\rput(! 0 #4){%
		\@AngleDroit[-1]{\@styleCodageCone}{\@dashCodageCone}{\@widthCodageCone}{\@tailleCodageCone}%
		\psline[linestyle=\@styleDiametreCone,dash=\@dashDiametreCone,linewidth=\@widthDiametreCone](! #2 neg 0)(! #2 0)}% diam\`etre
}

% _________________________________________________
%|                                                 |
%|                      Sph\`ere                     |
%|_________________________________________________|
\define@key{Sphere}{styleSphere}{\edef\@styleSphere{#1}}
\define@key{Sphere}{dashSphere}{\edef\@dashSphere{#1}}
\define@key{Sphere}{widthSphere}{\edef\@widthSphere{#1}}
\define@key{Sphere}{styleBasEquateur}{\edef\@styleBasEllipse{#1}}
\define@key{Sphere}{dashBasEquateur}{\edef\@dashBasEllipse{#1}}
\define@key{Sphere}{widthBasEquateur}{\edef\@widthBasEllipse{#1}}
\define@key{Sphere}{styleHautEquateur}{\edef\@styleHautEllipse{#1}}
\define@key{Sphere}{dashHautEquateur}{\edef\@dashHautEllipse{#1}}
\define@key{Sphere}{widthHautEquateur}{\edef\@widthHautEllipse{#1}}
\newcommand*{\Sphere}[3][]{%
% #1	styles
% #2	rayon
% #3 aplatissement de l'\'equateur
	\presetkeys{Sphere}{%
		styleSphere=solid,dashSphere=\dashDefaut,widthSphere=\widthDefaut,%
		styleBasEquateur=solid,dashBasEquateur=\dashDefaut,widthBasEquateur=\widthDefaut,%
		styleHautEquateur=dashed,dashHautEquateur=\dashDefaut,widthHautEquateur=\widthDefaut}{}%
	\setkeys{Sphere}{#1}%
	\pscircle[linestyle=\@styleSphere,dash=\@dashSphere,linewidth=\@widthSphere](! 0 0){#2}%
	\@Ellipse{0}{#2}{#3 #2 mul}%
}

\define@key{SectionSphere}{styleBasSectionSphere}{\edef\@styleBasEllipse{#1}}
\define@key{SectionSphere}{dashBasSectionSphere}{\edef\@dashBasEllipse{#1}}
\define@key{SectionSphere}{widthBasSectionSphere}{\edef\@widthBasEllipse{#1}}
\define@key{SectionSphere}{styleHautSectionSphere}{\edef\@styleHautEllipse{#1}}
\define@key{SectionSphere}{dashHautSectionSphere}{\edef\@dashhatEllipse{#1}}
\define@key{SectionSphere}{widthHautSectionSphere}{\edef\@widthHautEllipse{#1}}
\newcommand*{\SectionSphere}[4][]{%
% #1	styles
% #2	rayon
% #3	aplatissement de la section
% #4	c‚\`aö¬•te de la section
	\presetkeys{SectionSphere}{%
		styleBasSectionSphere=solid,dashBasSectionSphere=\dashDefaut,widthBasSectionSphere=\widthDefaut,%
		styleHautSectionSphere=dashed,dashHautSectionSphere=\dashDefaut,widthHautSectionSphere=\widthDefaut}{}%
	\setkeys{SectionSphere}{#1}%
	\pstVerb{/rr #4 dup mul #3 dup mul 1 sub div #2 dup mul add sqrt def}% rayon de l'ellipse repr\'esentant la section
	\rput(! 0 #4){\@Ellipse{#3 dup mul #4 mul 1 #3 dup mul sub div}{rr}{rr #3 mul}}%
}

\define@key{AxesSphere}{styleAxeSphere}{\edef\@styleAxeSphere{#1}}
\define@key{AxesSphere}{dashAxeSphere}{\edef\@dashAxeSphere{#1}}
\define@key{AxesSphere}{widthAxeSphere}{\edef\@widthAxeSphere{#1}}
\define@key{AxesSphere}{styleDimatreSphere}{\edef\@styleDiametreSphere{#1}}
\define@key{AxesSphere}{dashDiametreSphere}{\edef\@dashDiametreSphere{#1}}
\define@key{AxesSphere}{widthDiametreSphere}{\edef\@widthDiametreSphere{#1}}
\define@key{AxesSphere}{styleCodageSphere}{\edef\@styleCodageSphere{#1}}
\define@key{AxesSphere}{dashCodageSphere}{\edef\@dashCodageSphere{#1}}
\define@key{AxesSphere}{widthCodageSphere}{\edef\@widthCodageSphere{#1}}
\define@key{AxesSphere}{tailleCodageSphere}{\edef\@tailleCodageSphere{#1}}
\newcommand{\AxesSphere}[3][]{\@ifnextchar[{\@AxeSphere[#1]{#2}{#3}}{\@AxeSphere[#1]{#2}{#3}[0.5]}}%
\def\@AxeSphere[#1]#2#3[#4]{
% #1	styles
% #2	rayon
% #3	aplatissement \'equateur
% #4	d\'epassement au p‚\`aö¬•les, par d\'efaut 0.5
	\presetkeys{AxesSphere}{%
		styleAxeSphere=dashed,dashAxeSphere=5pt 2pt 2pt 2pt,widthAxeSphere=0.5pt,%
		styleDimatreSphere=dashed,dashDiametreSphere=\dashDefaut,widthDiametreSphere=0.5pt,%
		styleCodageSphere=solid,dashCodageSphere=1.5pt 1.5pt,widthCodageSphere=0.5pt,tailleCodageSphere=0.2}{}
	\setkeys{AxesSphere}{#1}%
	\psline[linestyle=\@styleAxeSphere,dash=\@dashAxeSphere,linewidth=\@widthAxeSphere](! 0 #2 #4 add neg)(! 0 #2 #4 add)% Axe
	\psline[linestyle=\@styleDiametreSphere,dash=\@dashDiametreSphere,linewidth=\@widthDiametreSphere](! #2 neg 0)(! #2 0)% diam\`etre
	\@AngleDroit[1]{\@styleCodageSphere}{\@dashCodageSphere}{\@widthCodageSphere}{\@tailleCodageSphere}% codage
}

\define@key{AxesSectionSphere}{styleDiametreSectionSphere}{\edef\@styleDiametreCone{#1}}
\define@key{AxesSectionSphere}{dashDiametreSectionSphere}{\edef\@dashDiametreCone{#1}}
\define@key{AxesSectionSphere}{widthDiametreSectionSphere}{\edef\@widthDiametreCone{#1}}
\define@key{AxesSectionSphere}{styleCodageSectionSphere}{\edef\@styleCodageCone{#1}}
\define@key{AxesSectionSphere}{dashCodageSectionSphere}{\edef\@dashCodageCone{#1}}
\define@key{AxesSectionSphere}{widthCodageSectionSphere}{\edef\@widthCodageCone{#1}}
\define@key{AxesSectionSphere}{tailleCodageSectionSphere}{\edef\@tailleCodageCone{#1}}
\newcommand*{\AxesSectionSphere}[4][]{%
% #1	styles
% #2	rayon
% #3	coeff d'aplatissement
% #4	c‚\`aö¬•te de la section
	\presetkeys{AxesSectionSphere}{%
		styleDiametreSectionSphere=dashed,dashDiametreSectionSphere=\dashDefaut,widthDiametreSectionSphere=0.5pt,%
		styleCodageSectionSphere=solid,dashCodageSectionSphere=\dashDefaut,widthCodageSectionSphere=0.5pt,tailleCodageSectionSphere=0.2}{}%
	\setkeys{AxesSectionSphere}{#1}%
	\edef\@styleAxeCone{none}% pas d'axe
	\edef\@dashAxeCone{\dashDefaut}% inutile puisque non trac\'e
	\edef\@widthAxeCone{\widthDefaut}% inutile puisque non trac\'e
	\rput(! 0 #4){\@AxesCone{#4 dup mul #3 dup mul 1 sub div #2 dup mul add sqrt}{#3}{#2}}%
}

\newcommand*{\NomSectionSphere}[7][]{%
% #1 : options
% #2 : rayon                  R
% #3 : coeff d'applatissement k
% #4 : ordonn\'ee de la section z
% #5 : Nom de l'extr\'emit\'e diam\`etre gauche
% #6 : Nom du centre du cercle
% #7 : Nom de l'extr\'emit\'e diam\`etre droit
	\rput(! 0 #4){\NomPoints[#1]{#4 dup mul #3 dup mul 1 sub div #2 dup mul add sqrt}{#3}{1}{#5}{#6}{#7}{}}%
}
% _________________________________________________
%|                                                 |
%|                  Calotte sph\'erique              |
%|_________________________________________________|
\define@key{Calotte}{styleCalotte}{\edef\@styleCalotte{#1}}
\define@key{Calotte}{dashCalotte}{\edef\@dashCalotte{#1}}
\define@key{Calotte}{widthCalotte}{\edef\@widthCalotte{#1}}
\define@key{Calotte}{styleBasCalotte}{\edef\@styleBasEllipse{#1}}
\define@key{Calotte}{dashBasCalotte}{\edef\@dashBasEllipse{#1}}
\define@key{Calotte}{widthBasCalotte}{\edef\@widthBasEllipse{#1}}
\define@key{Calotte}{styleHautCalotte}{\edef\@styleHautEllipse{#1}}
\define@key{Calotte}{dashHautCalotte}{\edef\@dashHautEllipse{#1}}
\define@key{Calotte}{widthHautCalotte}{\edef\@widthHautEllipse{#1}}
\newcommand*{\Calotte}[4][]{\@ifnextchar[{\@Calotte[#1]{#2}{#3}{#4}}{\@Calotte[#1]{#2}{#3}{#4}[B]}}%
\def\@Calotte[#1]#2#3#4[#5]{%
% #1	styles
% #2	rayon Sphere
% #3	aplatissement de la section
% #4	c‚\`aö¬•te de la section
% #5	optionnel [H] ou [B] selon que l'on veut la partie [H]aute ou [B]asse restante apr\`es coupe par un plan. Par d\'efaut B
	\presetkeys{Calotte}{%
		dashHautCalotte=\dashDefaut,widthHautCalotte=\widthDefaut,%
		styleCalotte=solid,dashCalotte=\dashDefaut,widthCalotte=\widthDefaut,%
		styleBasCalotte=solid,dashBasCalotte=\dashDefaut,widthBasCalotte=\widthDefaut}{}%
	\pstVerb{%
		/RR #2 dup mul def /kk #3 dup mul def%
		/zz #4 dup mul def%
		/rr zz kk 1 sub div RR add sqrt def% rayon de l'Ellipse repr\'esentant la section
		/yy #4 1 kk sub div def% ordonn\'ee du point de tangence entre Ellipse et cercle
		/AAangle yy #2 div asin def%
	}%
	\ifx B#5%
		\presetkeys{Calotte}{styleHautCalotte=solid}{}%
		\pstVerb{%
			/debut AAangle 180 add neg def%
			/fin AAangle def}%
	\fi%
	\ifx H#5%
		\presetkeys{Calotte}{styleHautCalotte=dashed}{}%
		\pstVerb{%
			/debut AAangle def%
			/fin 180 AAangle sub def}%
	\fi%
	\setkeys{Calotte}{#1}%
	\psellipticarc[linestyle=\@styleCalotte,dash=\@dashCalotte,linewidth=\@widthCalotte](! 0 0)(! #2 #2){! debut}{! fin}%
	\rput(! 0 #4){\@Ellipse{yy #4 sub}{rr}{rr #3 mul}}%
}

\makeatother


%%%%%%%%%%%%%%%%%%%%%%%%AirePLGRect
\newcommand{\AirePlgRect}[9]{
%#(1;2)(3;4)(5;6)(7;8) les 4 points #9 hauteur 
\psset{PointName=none,PointSymbol=none}
\pstGeonode(#1,#2){A}(#3,#4){B}(#5,#6){C}(#7,#8){D}
\pstProjection{A}{B}{C}[H]
\pstProjection{A}{B}{D}[I]
\pspolygon[linestyle=dashed,linewidth=1pt,linecolor=magenta](C)(H)(I)(D)
\pstRightAngle[linewidth=1pt,linecolor=magenta,yunit=2,RightAngleSize=0.2]{C}{H}{I}
\pstRightAngle[linewidth=1pt,linecolor=magenta,yunit=2,RightAngleSize=0.2]{H}{I}{D}
\pspolygon(A)(B)(C)(D)
\pcline[offset=19pt]{<->}(D)(I)\lput*{:U}{#9}

}
%%%%%%%%%%%%%%%%%%%%%%%%une croix PointSymbol#1=dotscale #2 = node
\newcommand{\PtSymbX}[2][2]{
\psset{PointSymbol=none,PointName=none}
\pstGeonode(0,0){Q1}(0.015,-0.01){Q2}
%\newlength{\abscroix}
%\setlength{\abscroix}{\pstAbscissa{#1}}
%\addtolength{\abscroix}{0.015}
\pstTranslation{Q1}{Q2}{#2}[Q3]
\rput{45}(Q3){\pstGeonode[dotscale=#1,PointName=none,PointSymbol=+](0,0){Q}}
%\rput{45}(O){\pstGeonode[PointName=none,PointSymbol=+](0,0){Q}}
}
%%%%%%%%%%%%%%%%%%%%%%



%%%%%%%%%%%%%%%%%nieme essai de coloration d'item en blanc
\newcommand{\enumbl}{\color{white}\item}
%%%%%%%%%%%%%%%%%%%%%%%%%%% pour exclure des section de toc
\newcommand{\stoptocwriting}{%
  \addtocontents{toc}{\protect\setcounter{tocdepth}{-5}}}
\newcommand{\resumetocwriting}{%
  \addtocontents{toc}{\protect\setcounter{tocdepth}{\arabic{tocdepth}}}}
%%%%%%%%%%%%%%%%%%index
%%%%pour classer les umlaut correctement


\renewcommand*\see[2]{\emph{\seename} #1}
\makeindex[program=makeindex,title=Index des mots utilis\'es dans le livre\vspace{-2cm}, columns=3,intoc=true, columnsep=10pt]
%\indexsetup{othercode={\vspace{-2cm}}}