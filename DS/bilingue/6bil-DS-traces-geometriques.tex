\documentclass[10pt,openany]{book}
\documentclass[12pt]{article}
\usepackage[utf8]{inputenc}
\usepackage[T1]{fontenc}
\usepackage[french]{babel}
\usepackage{amsmath,amssymb}
\usepackage{geometry}
\geometry{a4paper, margin=2cm}
\usepackage{pstricks,pstricks-add,pst-plot,pst-tree,pst-eps}
\usepackage{multicol}
\usepackage{graphicx}
\usepackage{enumitem}
\usepackage{amsfonts}
\usepackage{siunitx}
\usepackage{mathrsfs}
\usepackage{pgf,tikz}
\usepackage{wasysym}

% Commandes personnalisées
\newcommand{\euro}{\text{\euro}}
\newcommand{\rep}[1]{\rule{#1cm}{0.2mm}}
\newcommand{\ladate}[1]{\hfill \textit{#1} \hfill\null}
\newcommand{\exo}[1]{\vspace{0.5cm}\textbf{#1 }}
\newcommand{\bemerkung}{\textbf{Remarque :}}

% Configuration de la page
\pagestyle{empty}
\parindent=0mm
\parskip=5mm

\begin{document}
\section*{Devoir de synth\`ese}
\addcontentsline{toc}{section}{Devoir de synth\`ese-Trac\'es de g\'eom\'etrie}

\begin{multicols}{2}
\exo{\"Ubung 1 : Skizze von Vielecken}\\
Zeichne eine Skizze f\"ur alle Figuren ohne Lineal mit der Kodierung.
\begin{enumerate}
\item Das Dreieck $ZOE$, sodass $ZO=5cm$, $OE=5cm$ und $EZ=4,5cm$.
\item Das Dreieck $RAY$ rechtwinklig in $Y$, sodass $RY=3cm$ und $RA=5cm$.
\item Das Dreieck $HEL$ gleichschenklig  in $H$, sodass $HE=2cm$ und $EL=4cm$.
\item Das Parallelogramm $MELI$, sodass $ME=3~cm$,$LE=4~cm$.
\item Das Rechteck $DAMI$, sodass $DA=6~cm$ und $MA=8~cm$.
\item Die Raute $UMRA$,  sodass $UM=5~cm$ und $RU=3~cm$.
\end{enumerate}


\exo{\"Ubung 2 : Zeichnen}
Gisela hat die Skizze von einem Dreieck, einem Parallelogramm, einem Rechteck und einer Raute gezeichnet.\\
Zeichne die Figuren in wahrer Gr\"o\ss e.\\
\begin{pspicture}(-1,-0.4)(9,5.5)
\psset{unit=0.7cm}
\footnotesize
% triangle1
\pscurve(0,0)(0.2,1)(0.7,2.7)(1,3)
\pscurve(0,0)(2,0.4)(3.95,0.9)(5,1) 
\pscurve(1,3)(2.5,2)(4.5,1.2)(5,1)
\rput(0,-0.30){{\ECFAugie\fontsize{10pt}{13pt}\selectfont D}}\rput(1,3.3){{\ECFAugie\fontsize{10pt}{13pt}\selectfont E}}\rput(5.3,1){{\ECFAugie\fontsize{10pt}{13pt}\selectfont G}}

\rput{75}(0.2,2.15){\begin{cursive}3,5~cm \end{cursive}}
\rput{10}(1.8,0.6){\begin{cursive}5,4~cm\end{cursive}}
\rput{-30}(3,2.2){\begin{cursive}4,5~cm\end{cursive}}


%\psarc(0,0){0.8}{41}{80}

%triangle2
\rput(0.8,0){
\psset{yunit=1.2}
\pscurve(6,0)(7,0.11)(8,-0.06)(9,0)
\pscurve(9,0)(8.95,0.6)(9.1,1.5)(9,2) 
\pscurve(9,2)(8,2.1)(7,1.95)(6,2)
\pscurve(6,2)(6.05,1)(5.95,0.31)(6,0)

\rput(6,0){
\pscurve(0,0.3)(0.15,0.28)(0.3,0.3)
\pscurve(0.3,0.3)(0.35,0.15)(0.3,0)
}
\rput{90}(9,0){
\pscurve(0,0.3)(0.15,0.28)(0.3,0.3)
\pscurve(0.3,0.3)(0.35,0.15)(0.3,0)
}
\rput{180}(9,2){
\pscurve(0,0.3)(0.15,0.28)(0.3,0.3)
\pscurve(0.3,0.3)(0.35,0.15)(0.3,0)
}
\rput{270}(6,2){
\pscurve(0,0.3)(0.15,0.28)(0.3,0.3)
\pscurve(0.3,0.3)(0.35,0.15)(0.3,0)
}

\rput(6,-0.30){{\ECFAugie\fontsize{10pt}{13pt}\selectfont R}}\rput(9.3,0){{\ECFAugie\fontsize{10pt}{13pt}\selectfont K}}\rput(9.3,2){{\ECFAugie\fontsize{10pt}{13pt}\selectfont L}}\rput(5.9,2.3){{\ECFAugie\fontsize{10pt}{13pt}\selectfont Y}}

\rput(7.8,0.25){\begin{cursive}5,5~cm \end{cursive}}
\rput{90}(5.6,1){\begin{cursive}4,5~cm \end{cursive}}
}

%triangle3
\rput(2,4){
\pscurve(4,0)(4.5,0.251)(5.6,0.83)(6,1)
\pscurve(6,1)(6.5,2.05)(6.91,2.95)(7,3) 
\pscurve(7,3)(6.5,2.9)(5.7,2.35)(5,2)
\pscurve(5,2)(4.8,1.75)(4.27,0.45)(4,0)

\rput(4,-0.30){{\ECFAugie\fontsize{10pt}{13pt}\selectfont P}}\rput(6,0.7){{\ECFAugie\fontsize{10pt}{13pt}\selectfont T}}\rput(7,3.3){{\ECFAugie\fontsize{10pt}{13pt}\selectfont F}}\rput(5,2.3){{\ECFAugie\fontsize{10pt}{13pt}\selectfont O}}

\rput{75}(6.87,2){\begin{cursive}4,1~cm \end{cursive}}
}

\rput(-1,1.5){
\pscurve(2,3)(3,3.11)(4,2.97)(5,3)
\pscurve(5,3)(5.2,3.25)(5.8,4.85)(6,5) 
\pscurve(6,5)(5,5.05)(4,4.95)(3,5)
\pscurve(3,5)(2.8,4.55)(2.2,3.85)(2,3)

\rput(2,2.77){{\ECFAugie\fontsize{10pt}{13pt}\selectfont M}}\rput(5,2.7){{\ECFAugie\fontsize{10pt}{13pt}\selectfont H}}\rput(6,5.3){{\ECFAugie\fontsize{10pt}{13pt}\selectfont A}}\rput(3,5.3){{\ECFAugie\fontsize{10pt}{13pt}\selectfont S}}

\rput{75}(1.8,4){\begin{cursive}5,1~cm \end{cursive}}
\rput(4.5,5.3){\begin{cursive}7,3~cm \end{cursive}}
}

\normalsize
\end{pspicture}

\columnbreak

\exo{Exercice 3 : }
Trace en vraie grandeur : \\
\begin{enumerate}

\item Le parall\'elogramme $RIVE$, sodass $RI=5,3~cm$ und \\$VR=4,1~cm$.
\item Le rectangle $NEUF$, sodass $NE=39~mm$ und \\$EU=55~mm$.
\begin{enumerate}

\item Quelle est la nature d'un quadrilat\`ere qui a des diagonales qui se coupent en leur milieux.
\item Quelle est la nature d'un quadrilat\`ere qui a $4$ c\^ot\'es de m\^eme longueur ?
\end{enumerate}

\end{enumerate}

\end{multicols}

\end{document}