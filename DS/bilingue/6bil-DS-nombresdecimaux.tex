%%!TEX TS-program = latex
\documentclass[12pt]{article}
\usepackage[utf8]{inputenc}
\usepackage[T1]{fontenc}
\usepackage[french]{babel}
\usepackage{amsmath,amssymb}
\usepackage{geometry}
\geometry{a4paper, margin=2cm}
\usepackage{pstricks,pstricks-add,pst-plot,pst-tree,pst-eps}
\usepackage{multicol}
\usepackage{graphicx}
\usepackage{enumitem}
\usepackage{amsfonts}
\usepackage{siunitx}
\usepackage{mathrsfs}
\usepackage{pgf,tikz}
\usepackage{wasysym}

% Commandes personnalisées
\newcommand{\euro}{\text{\euro}}
\newcommand{\rep}[1]{\rule{#1cm}{0.2mm}}
\newcommand{\ladate}[1]{\hfill \textit{#1} \hfill\null}
\newcommand{\exo}[1]{\vspace{0.5cm}\textbf{#1 }}
\newcommand{\bemerkung}{\textbf{Remarque :}}

% Configuration de la page
\pagestyle{empty}
\parindent=0mm
\parskip=5mm

\begin{document}

\section*{Devoir de synthèse}
\addcontentsline{toc}{section}{Devoir de synthèse-Nombres Décimaux}

\begin{multicols}{2}
\exo{Exercice 1 : Au collège}\\
Le tableau suivant représente le nombre de filles et de garçons dans un collège.
\begin{center}
\begin{tabular}{|c|c|c|c|}
\cline{2-4}
\multicolumn{1}{l|}{}&Filles&Garçons&Total\\
\hline
6ème&&$70$&$130$\\
\hline
5ème&$70$&$65$&\\ 
\hline
4ème&$75$&&\\
\hline 
3ème&$70$&&$140$\\
\hline
Gesamtzahl&&&$560$\\
\hline
\end{tabular}
\end{center}

\begin{enumerate}
\item Complète le tableau.
\item Combien de garçons sont en 4ème ?
\item Combien de filles sont en 5ème ?
\item Combien de filles comporte le collège ?
\item Combien de garçons comporte le collège ?
\end{enumerate}

\columnbreak

\end{multicols}

[Suite du contenu du DS nombres décimaux bilingue]

\end{document}