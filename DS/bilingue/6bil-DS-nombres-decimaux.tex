%%!TEX TS-program = latex
\documentclass[12pt]{article}
\usepackage[utf8]{inputenc}
\usepackage[T1]{fontenc}
\usepackage[french]{babel}
\usepackage{amsmath,amssymb}
\usepackage{geometry}
\geometry{a4paper, margin=2cm}
\usepackage{pstricks,pstricks-add,pst-plot,pst-tree,pst-eps}
\usepackage{multicol}
\usepackage{graphicx}
\usepackage{enumitem}
\usepackage{amsfonts}
\usepackage{siunitx}
\usepackage{mathrsfs}
\usepackage{pgf,tikz}
\usepackage{wasysym}

% Commandes personnalisées
\newcommand{\euro}{\text{\euro}}
\newcommand{\rep}[1]{\rule{#1cm}{0.2mm}}
\newcommand{\ladate}[1]{\hfill \textit{#1} \hfill\null}
\newcommand{\exo}[1]{\vspace{0.5cm}\textbf{#1 }}
\newcommand{\bemerkung}{\textbf{Remarque :}}

% Configuration de la page
\pagestyle{empty}
\parindent=0mm
\parskip=5mm

\begin{document}

\section*{6\`eme - Devoir de synth\`ese}
\addcontentsline{toc}{section}{Devoir de synth\`ese-Nombres Décimaux}

\begin{multicols}{2}

\exo{Übung 1 : }
\begin{enumerate}
\item Gegeben seien : \textbf{452 : 7} und \textbf{9~587 : 23}.
\begin{enumerate}
\item Finde den Quotient und den Rest von beide Divisionen mit Rest.
\item Schreibe die Zerlegung den beiden Zahlen.
\end{enumerate}
\item Gegeben seien : \textbf{31 : 8} und \textbf{563 : 12}.\\
Berechne die Quotienten schriftlich. Wenn es möglich ist gib einen genauen Wert des Quotienten, wenn nicht runde auf Hundertstel.
\end{enumerate}
\columnbreak

\exo{Übung 2 :} Mit hilfe den Teilbarkeitsregeln ergänze die folgende Tabelle mit Kreuzen.\\\rule{0cm}{0cm}\\
\begin{tabular}{|p{2cm}|c|c|c|c|c|}\hline
\rput(0,0){\psline(-0.2,1)(2.2,-.10)}\rput{-28}(1.3,0.6){\footnotesize Zahlen}\rput{-28}(0.6,0.3){\footnotesize Teiler}&\rule{0cm}{1cm}12&15&111&2~538&10~128\\ \hline
2&&&&&\\ \hline
3&&&&&\\ \hline
4&&&&&\\ \hline
5&&&&&\\ \hline
9&&&&&\\ \hline
\end{tabular}
\end{multicols}

\exo{Exercice 3 : Problèmes}
\begin{enumerate}
\item Dans une classe de $26$ élèves, on forme des équipes de volleyball ($6$ joueurs par équipe).\\
Combien d'équipes peut on former ?

\item Moritz hat eine französische Bäckerei geöffnet in Freiburg.\\
Er möchte belegte Brötchen für $16$ Personnen vorbereiten.\\
In einem Baguette macht er $3$ Brötchen. Jeder bekommt zwei Brötchen.\\
Wieviel Baguette muss er backen ?\\
Wieviele belegte Brötchen kann er noch vorbereiten ?

\item Un supermarché met en vente $700$ boîtes de $1~kg$ de sucre en morceaux.\\
Cette vente doit lui rapporter au moins $650$~\euro.
A quel prix minimum, arrondi au centime, doit-il mettre en vente la boîte de $1~kg$ ?

\item Bärbel kauft $5$ Hefte für je $3,44$~\euro und $6$ Päckchen von Din-A4-Blätter.\\
Sie zahlt insgesamt $24,70$~\euro.\\
Wieviel kostet ein Päckchen Din-A4-Blätter ?
\end{enumerate}

\exo{Exercice 4 :}\\
Convertir $7~653~s$ en $\ldots h \ldots min \ldots s$.

\exo{Übung 5 : } Zeichne das Bild den Figuren an der Gerade $(\Delta)$\\
(Benutze das Gitter wenn es möglich ist.)\\
\begin{pspicture}(-6,0)(6,8)
\psgrid[subgriddiv=2,gridlabels=0pt,gridcolor=darkgray](-6,6)
\psline[linewidth=1.5pt](-6,3)(12,3)
\pspolygon[linewidth=1.5pt](-6,5)(-4,3)(-3,6)
\pspolygon[linewidth=1.5pt](-2,2)(0,4)(3,4.5)(1,2.5)
\pstGeonode[dotscale=1.5,PointSymbol=x](8,5){O}
\pstGeonode[PointName=none,PointSymbol=none](8,8){A}
\pstCircleOA[linewidth=1.5pt]{O}{A}
\rput(12.4,3){($\Delta$)}
\end{pspicture}

\end{document}