\documentclass[10pt,openany]{book}
\documentclass[12pt]{article}
\usepackage[utf8]{inputenc}
\usepackage[T1]{fontenc}
\usepackage[french]{babel}
\usepackage{amsmath,amssymb}
\usepackage{geometry}
\geometry{a4paper, margin=2cm}
\usepackage{pstricks,pstricks-add,pst-plot,pst-tree,pst-eps}
\usepackage{multicol}
\usepackage{graphicx}
\usepackage{enumitem}
\usepackage{amsfonts}
\usepackage{siunitx}
\usepackage{mathrsfs}
\usepackage{pgf,tikz}
\usepackage{wasysym}

% Commandes personnalisées
\newcommand{\euro}{\text{\euro}}
\newcommand{\rep}[1]{\rule{#1cm}{0.2mm}}
\newcommand{\ladate}[1]{\hfill \textit{#1} \hfill\null}
\newcommand{\exo}[1]{\vspace{0.5cm}\textbf{#1 }}
\newcommand{\bemerkung}{\textbf{Remarque :}}

% Configuration de la page
\pagestyle{empty}
\parindent=0mm
\parskip=5mm

\begin{document}
\section*{Devoir de synth\`ese}

\begin{multicols}{2}
\exo{\"Ubung 1 :}
\begin{enumerate}
\item Berechne $46,35 + 167,058$
\vspace{4cm}
\item Welche Ziffer ist an der \\Zehnerstelle von dem Ergebnis ? ...........
\item Welche Ziffer ist an der\\ Hunderstelstelle von dem Ergebnis ? ...........\\
\item Im Ergebnis, wie viele Hunderstel gibt es ?...........\\
\item Im Ergebnis, wie viele Hunderter gibt es ? ...........
\end{enumerate}

\columnbreak

\exo{Exercice 2 :} 
\\Compl\`ete ce tableau avec l'aide des crit\`eres de divisibilit\'e\\\rule{0cm}{0cm}\\
\begin{tabular}{|p{2cm}|c|c|c|c|c|}\hline
\rput(0,0){\psline(-0.2,1)(2.2,-.10)}\rput{-28}(1.3,0.6){\footnotesize Nombre}\rput{-28}(0.6,0.3){\footnotesize Diviseur}&\rule{0cm}{1cm}12&15&111&2~538&10~128\\ \hline
2&&&&&\\ \hline
3&&&&&\\ \hline
4&&&&&\\ \hline
5&&&&&\\ \hline
9&&&&&\\ \hline
\end{tabular}

\end{multicols}

\exo{Exercice 3 : Probl\`emes}
\begin{enumerate}
\item Dans une classe de $26$ \'el\`eves, on forme des \'equipes de volley-ball ($6$ joueurs par \'equipe).\\
Combien d'\'equipes peut on former ?

\item Un supermarch\'e met en vente $700$ bo\^ites de $1~kg$ de sucre en morceaux.\\
Cette vente doit lui rapporter au moins $650$~\eurosans.
A quel prix minimum, arrondi au centime, doit-il mettre en vente la bo\^ite de $1~kg$ ?

\item Arthur a invit\'e des amis pour son anniversaire, avec lui ils sont $8$.\\
Pour cette f\^ete il a confectionn\'e $200$ canap\'es.\\
\begin{enumerate}
\item Combien de canap\'es aura chacun des convives.
\item Il veut pr\'esenter ses canap\'e sur des assiettes de $12$ canap\'es chacune.\\
Combien d'assiettes utilisera-t-il au minimum.
\item Il a d\'epens\'e $68$~\eurosans pour les canap\'es combien a-t-il d\'epens\'e par personne ?

\end{enumerate}

\end{enumerate}

\exo{Exercice 4 :}\\
Convertir $7~653~s$ en $\ldots h \ldots min \ldots s$.

\exo{Bonus :} Trace les axes de sym\'etrie des figures.\\ 
\fontsize{36}{16}\selectfont\ding{96}\quad\ding{40}\quad\ding{54}\quad\ding{37}\quad\ding{48}\quad\ding{39}\normalsize

\end{document}