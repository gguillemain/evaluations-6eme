\documentclass[10pt,openany]{book}
\documentclass[12pt]{article}
\usepackage[utf8]{inputenc}
\usepackage[T1]{fontenc}
\usepackage[french]{babel}
\usepackage{amsmath,amssymb}
\usepackage{geometry}
\geometry{a4paper, margin=2cm}
\usepackage{pstricks,pstricks-add,pst-plot,pst-tree,pst-eps}
\usepackage{multicol}
\usepackage{graphicx}
\usepackage{enumitem}
\usepackage{amsfonts}
\usepackage{siunitx}
\usepackage{mathrsfs}
\usepackage{pgf,tikz}
\usepackage{wasysym}

% Commandes personnalisées
\newcommand{\euro}{\text{\euro}}
\newcommand{\rep}[1]{\rule{#1cm}{0.2mm}}
\newcommand{\ladate}[1]{\hfill \textit{#1} \hfill\null}
\newcommand{\exo}[1]{\vspace{0.5cm}\textbf{#1 }}
\newcommand{\bemerkung}{\textbf{Remarque :}}

% Configuration de la page
\pagestyle{empty}
\parindent=0mm
\parskip=5mm

\begin{document}

\section*{Devoir de synth\`ese}
\addcontentsline{toc}{section}{Devoir de synth\`ese-Angles}
\begin{multicols}{2}

\exo{"Ubung 1 : }
Messe den Winkeln und dann erg"anze die Tabelle.
\begin{multicols}{2}
 \begin{pspicture}(-1,0)(5,6)
\SpecialCoor
\pstGeonode[PointSymbol=none,PointName=none](1.5;50){A1}(1.5;180){B1}
\pstGeonode[PointSymbol=none,PointName=default](4,1){C}
\pstTranslation[PointSymbol=+,PointName=default,PosAngle={0,-90}]{O}{C}{A1,B1}[A,B]
\pstLineAB[nodesepB=-0.5]{C}{A}
\pstLineAB[nodesepB=-3]{C}{B}
\pstMarkAngle{A}{C}{B}{}

\pstGeonode[PointSymbol=none,PointName=none](1.5;80){A2}(1.5;35){B2}
\pstGeonode[PointSymbol=none,PointName=default](1,2){S}
\pstTranslation[PointSymbol=+,PointName=default]{O}{S}{A2,B2}[R,T]
\pstLineAB[nodesepB=-0.5]{S}{R}
\pstLineAB[nodesepB=-0.5]{S}{T}
\pstMarkAngle{T}{S}{R}{}

\pstGeonode[PointSymbol=none,PointName=none](1.5;110){A3}(1.5;163){B3}
\pstGeonode[PointSymbol=none,PointName=default](8,1){F}
\pstTranslation[PointSymbol=+,PointName=default,PosAngle={0,-90}]{O}{F}{A3,B3}[E,G]
\pstLineAB[nodesepB=-0.5]{F}{E}
\pstLineAB[nodesepB=-0.5]{F}{G}
\pstMarkAngle{E}{F}{G}{}

\end{pspicture}

\columnbreak
\setlength{\arrayrulewidth}{1pt}
\begin{tabular}{|p{2cm}|p{4.5cm}|c|}
\hline
\rule{0cm}{0.4cm}Name&Natur des Winkels&Winkelma\ss\\
\hline
\rule{0cm}{0.8cm}&&&\\
\hline
\rule{0cm}{0.8cm}&&&\\
\hline
\rule{0cm}{0.8cm}&&&\\
\hline
\rule{0cm}{0.8cm}&&&\\
\hline
\rule{0cm}{0.8cm}&&&\\
\hline
\end{tabular}

\end{multicols}

\exo{"Ubung 2 : }
\begin{enumerate}
\item Zeichne eine Strecke $[AB]$ , sodass $AB=6$cm.
\item Zeichne die Mittelsenkrechte $(d)$ von $[AB]$.
\item Setze einen Punkt $M$ auf $(d)$ , sodass $AM=5$cm.
\item Ohne zu messen finde die L"ange von der Strecke $[MB]$.\\ Begr"unde.
\end{enumerate}

\columnbreak

\exo{"Ubung 3 : }
Zeichne das Bild den Figuren an der Gerade $(\Delta)$.\\
(Benutze das Gitter wenn es m"oglich ist.)\\
\begin{pspicture}(-6,0)(6,8)
\psgrid[subgriddiv=2,gridlabels=0pt,gridcolor=darkgray](-6,6)
\psline[linewidth=1.5pt](-6,3)(12,3)
\pspolygon[linewidth=1.5pt](-6,5)(-4,3)(-3,6)
\pspolygon[linewidth=1.5pt](-2,2)(0,4)(3,4.5)(1,2.5)
\pstGeonode[dotscale=1.5,PointSymbol=x](8,5){O}
\pstGeonode[PointName=none,PointSymbol=none](8,8){A}
\pstCircleOA[linewidth=1.5pt]{O}{A}
\rput(12.4,3){$(\Delta)$}
\end{pspicture}

\end{multicols}
\end{document}