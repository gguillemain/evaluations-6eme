\documentclass[10pt,openany]{book}
\documentclass[12pt]{article}
\usepackage[utf8]{inputenc}
\usepackage[T1]{fontenc}
\usepackage[french]{babel}
\usepackage{amsmath,amssymb}
\usepackage{geometry}
\geometry{a4paper, margin=2cm}
\usepackage{pstricks,pstricks-add,pst-plot,pst-tree,pst-eps}
\usepackage{multicol}
\usepackage{graphicx}
\usepackage{enumitem}
\usepackage{amsfonts}
\usepackage{siunitx}
\usepackage{mathrsfs}
\usepackage{pgf,tikz}
\usepackage{wasysym}

% Commandes personnalisées
\newcommand{\euro}{\text{\euro}}
\newcommand{\rep}[1]{\rule{#1cm}{0.2mm}}
\newcommand{\ladate}[1]{\hfill \textit{#1} \hfill\null}
\newcommand{\exo}[1]{\vspace{0.5cm}\textbf{#1 }}
\newcommand{\bemerkung}{\textbf{Remarque :}}

% Configuration de la page
\pagestyle{empty}
\parindent=0mm
\parskip=5mm

\begin{document}
\section*{Devoir de synth\`ese}
\addcontentsline{toc}{section}{Devoir de synth\`ese-Angles}

\begin{multicols}{2}
\end{multicols}

\exo{\"Ubung 1 : }\\
Messe den Winkeln und dann erg\"anze die Tabelle.
\begin{multicols}{2}
 
 \begin{pspicture}(0,0)(6,6)
\SpecialCoor
\pstGeonode[PointSymbol=none,PointName=none](1.5;50){A1}(1.5;180){B1}
\pstGeonode[PointSymbol=none,PointName=default](4,1){C}
\pstTranslation[PointSymbol=+,PointName=default,PosAngle={0,-90}]{O}{C}{A1,B1}[A,B]
\pstLineAB[nodesepB=-0.5]{C}{A}
\pstLineAB[nodesepB=-3]{C}{B}
\pstMarkAngle{A}{C}{B}{}

\pstGeonode[PointSymbol=none,PointName=none](1.5;80){A2}(1.5;35){B2}
\pstGeonode[PointSymbol=none,PointName=default](1,2){S}
\pstTranslation[PointSymbol=+,PointName=default]{O}{S}{A2,B2}[R,T]
\pstLineAB[nodesepB=-0.5]{S}{R}
\pstLineAB[nodesepB=-0.5]{S}{T}
\pstMarkAngle{T}{S}{R}{}

\pstGeonode[PointSymbol=none,PointName=none](1.5;110){A3}(1.5;163){B3}
\pstGeonode[PointSymbol=none,PointName=default](8,1){F}
\pstTranslation[PointSymbol=+,PointName=default,PosAngle={0,-90}]{O}{F}{A3,B3}[E,G]
\pstLineAB[nodesepB=-0.5]{F}{E}
\pstLineAB[nodesepB=-0.5]{F}{G}
\pstMarkAngle{E}{F}{G}{}

\pstGeonode[PointSymbol=none,PointName=none](1.5;180){A4}(1.5;0){B4}
\pstGeonode[PointSymbol=none,PointName=default,PosAngle=90](2,5.6){I}
\pstTranslation[PointSymbol=+,PointName=default,PosAngle={90,90}]{O}{I}{A4,B4}[H,J]
\pstLineAB[nodesepB=-0.5]{I}{H}
\pstLineAB[nodesepB=-0.5]{I}{J}
\pstMarkAngle{H}{I}{J}{}

\pstGeonode[PointSymbol=none,PointName=none](1.5;180){A5}(1.5;270){B5}
\pstGeonode[PointSymbol=none,PointName=default](6,5){L}
\pstTranslation[PointSymbol=+,PointName=default,PosAngle={90,0}]{O}{L}{A5,B5}[K,M]
\pstLineAB[nodesepB=-0.5]{L}{K}
\pstLineAB[nodesepB=-0.5]{L}{M}
\pstRightAngle{K}{L}{M}
\end{pspicture}

\columnbreak
 \setlength{\arrayrulewidth}{1pt}
 \begin{tabular}{|p{2cm}|p{4.5cm}|c|}
 \hline
 \rule{0cm}{0.4cm}Name&Natur des Winkels&Winkelma\ss\\
 \hline
 \rule{0cm}{0.8cm}&&\\
 \hline
 \rule{0cm}{0.8cm}&&\\
 \hline
 \rule{0cm}{0.8cm}&&\\
 \hline
 \rule{0cm}{0.8cm}&&\\
 \hline
 \rule{0cm}{0.8cm}&&\\
 \hline
 \end{tabular}
 
\end{multicols}

\exo{\"Ubung 2 :}\\
Zeichne in wahrer Gr\"o\ss e.

\begin{multicols}{2}
\begin{pspicture}(-0.5,0)(4,3.5)
\psset{PointSymbol=none,PointName=none,LabelSep=0.8}
\pstGeonode(0,0){O}(2;87){A}(4,0){D}(0.25,0.3){X}(0,0.4){Y}(0,-0.3){Z}
\pstRotation[RotAngle=56]{A}{O}[B]
\pstRotation[RotAngle=-105,PointSymbol=none,PointName=none]{D}{O}[C1]
\pstHomO[HomCoef=0.7]{D}{C1}[C]
%\pspolygon(O)(A)(B)(C)(D)
\pstTranslation{O}{X}{A,C}[A2,C2]
\pstTranslation{O}{Y}{B}[B2]
\pstTranslation{O}{Z}{D}[D2]
\rput(A2){{\ECFAugie\fontsize{10pt}{13pt}\selectfont A}}
\rput(B2){{\ECFAugie\fontsize{10pt}{13pt}\selectfont B}}
\rput(C2){{\ECFAugie\fontsize{10pt}{13pt}\selectfont C}}
\rput(D2){{\ECFAugie\fontsize{10pt}{13pt}\selectfont D}}
\rput(Z){{\ECFAugie\fontsize{10pt}{13pt}\selectfont O}}
\pscurve(O)(0.1,1)(A)
\pscurve(A)(1,1.35)(1.2,1.25)(B)
\pscurve(B)(2,1.1)(2.8,1.7)(C)
\pscurve(C)(4.3,1.9)(4.2,1)(D)
\pscurve(D)(2,-0.03)(1,0.05)(O)
\pstMarkAngle{O}{A}{B}{{\ECFAugie\fontsize{8pt}{10pt}\selectfont 46\degres}}
\pstMarkAngle{D}{O}{A}{{\ECFAugie\fontsize{8pt}{10pt}\selectfont 83\degres}}
\pstMarkAngle{C}{D}{O}{{\ECFAugie\fontsize{8pt}{10pt}\selectfont 105\degres}}
\rput(2,-0.3){{\ECFAugie\fontsize{10pt}{13pt}\selectfont 5~cm}}
\rput{-95}(4.5,1.2){{\ECFAugie\fontsize{10pt}{13pt}\selectfont 6~cm}}
\rput{86}(-0.1,1){{\ECFAugie\fontsize{10pt}{13pt}\selectfont 3~cm}}

\end{pspicture}

\columnbreak

\begin{pspicture}(-1,0)(9,3.5)
\footnotesize
\rput{45}(-3,0){
\pscurve(0,0)(0.2,1)(0.7,2.7)(1,3)
\pscurve(0,0)(2,0.4)(3.95,0.9)(5,1) 
\pscurve(1,3)(2.5,2)(4.5,1.2)(5,1)
\rput{-45}(0,-0.30){{\ECFAugie\fontsize{10pt}{13pt}\selectfont G}}
\rput{-45}(1,3.3){{\ECFAugie\fontsize{10pt}{13pt}\selectfont E}}
\rput{-45}(5.3,1){{\ECFAugie\fontsize{10pt}{13pt}\selectfont F}}

\rput{75}(0.2,2.15){{\ECFAugie\fontsize{8pt}{10pt}\selectfont 5,5~cm}}
\rput{40}(1.7,1.7){{\ECFAugie\fontsize{8pt}{10pt}\selectfont 3,4~cm}}
\rput{-45}(0.6,1){{\ECFAugie\fontsize{8pt}{10pt}\selectfont 52\degres}}

\psarc(0,0){0.8}{41}{80}
}

\rput{10}(-2,-1){
\rput{-90}(2,5){
\psset{yunit=1.2}
\pscurve(0,0)(0.2,1)(0.7,2.7)(1,4)
\pscurve(0,0)(1,0.9)(1.95,1.6)(3,2.5) 
\pscurve(1,4)(2,3.05)(3,2.5)

\psarc(0,0){0.8}{45}{80}
\psarc(1,4){0.8}{255}{310}

\rput{80}(0,-0.30){{\ECFAugie\fontsize{10pt}{13pt}\selectfont H}}
\rput{80}(1,4.3){{\ECFAugie\fontsize{10pt}{13pt}\selectfont K}}
\rput{80}(3.3,2.5){{\ECFAugie\fontsize{10pt}{13pt}\selectfont J}}

\rput{80}(0.2,2.15){{\ECFAugie\fontsize{8pt}{10pt}\selectfont 5,5~cm}}
\rput{80}(1.2,3){{\ECFAugie\fontsize{8pt}{10pt}\selectfont 65\degres}}
\rput{80}(0.6,1){{\ECFAugie\fontsize{8pt}{10pt}\selectfont 41\degres}}
}
}

\normalsize
\end{pspicture}

\end{multicols}

\exo{\"Ubung 3 :}\\
\begin{enumerate}
\item Zeichne ein Dreieck $ABC$ , sodass :\\ $AB=6cm~~;~~BC=8cm;~~;~~AC=10~cm$.\\
\item Markiere $I$ den Mittelpunkt von $[AB]$, und $J$ den Mittelpunkt von $[BC]$\\
\item Zeichne $[CI)$.
\item Zeichne den Kreis mit $[AC]$ als Durchmesser.
\item Markiere $L$ den Schnittpunkt von dem Kreis mit $[CI)$, der nicht auf $C$ liegt.
\end{enumerate}

\begin{multicols}{2}
\exo{Exercice 4 :}\\
Les points $A$, $B$ et $C$ sont-ils align\'es ? Justifier.

\begin{pspicture}(-1,0)(4,2)
\SpecialCoor
\pstGeonode[PointSymbol=+,PosAngle={-90,-90,-90}](0,0){A}(2,0){B}(4.3,0){C}
\pstGeonode[PointSymbol=none,PointName=none](0.5;150.5){A3}(1.5;153.5){B3}(2.2;152){C3}(0.5;60.5){A4}(1.5;63.5){B4}(2.5;62){C4}
\pstTranslation[PointSymbol=none,PointName=none]{A}{B}{A3,B3,C3,A4,B4,C4}[A1,B1,C1,A2,B2,C2]
\pstRotation[RotAngle=-28,PointSymbol=+,PosAngle=90]{B}{A}[M]
\pstRotation[RotAngle=62,PointSymbol=+]{B}{C}[N]
\pstMarkAngle[LabelSep=1.3,MarkAngleRadius=1]{M}{B}{A}{{\ECFAugie\fontsize{8pt}{10pt}\selectfont 28\degres}}
\pstMarkAngle[LabelSep=1,MarkAngleRadius=0.7]{C}{B}{N}{{\ECFAugie\fontsize{8pt}{10pt}\selectfont 62\degres}}
\pscurve(-0.2,0)(1,0.05)(B)(3,-0.05)(4.5,0)
\pscurve(B)(A1)(B1)(C1)
\pscurve(B)(A2)(B2)(C2)
\pstRightAngle{M}{B}{N}
\end{pspicture}

\columnbreak

\exo{Bonus :}\\
\begin{pspicture}(-1,0)(4,3)
\pstGeonode[PointSymbol=none,PosAngle={180,135,90,90,-45}](0,0){B}(1,1){D}(2,2){A}(4,2){E}(5,0){C}
\pspolygon(B)(D)(A)(E)(C)
\pspolygon(C)(D)(E)(A)
\pstMarkAngle{E}{C}{A}{$70$\degres}
\pstMarkAngle{C}{B}{A}{$40$\degres}
\end{pspicture}
\\\rule{1cm}{0cm}\\
Sur la figure ci-dessus : \\
$[AB]=[AC]$ et $[CD)$ est la bissectrice de $\widehat{ACB}$.\\
Quelle est la nature du triangle $CDE$ ? Justifier.

\end{multicols}
\end{document}