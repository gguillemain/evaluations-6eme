%%!TEX TS-program = latex
\documentclass[12pt]{article}
\usepackage[utf8]{inputenc}
\usepackage[T1]{fontenc}
\usepackage[french]{babel}
\usepackage{amsmath,amssymb}
\usepackage{geometry}
\geometry{a4paper, margin=2cm}
\usepackage{pstricks,pstricks-add,pst-plot,pst-tree,pst-eps}
\usepackage{multicol}
\usepackage{graphicx}
\usepackage{enumitem}
\usepackage{amsfonts}
\usepackage{siunitx}
\usepackage{mathrsfs}
\usepackage{pgf,tikz}
\usepackage{wasysym}

% Commandes personnalisées
\newcommand{\euro}{\text{\euro}}
\newcommand{\rep}[1]{\rule{#1cm}{0.2mm}}
\newcommand{\ladate}[1]{\hfill \textit{#1} \hfill\null}
\newcommand{\exo}[1]{\vspace{0.5cm}\textbf{#1 }}
\newcommand{\bemerkung}{\textbf{Remarque :}}

% Configuration de la page
\pagestyle{empty}
\parindent=0mm
\parskip=5mm

\begin{document}

\section*{Devoir de synth\`ese}
\addcontentsline{toc}{section}{Devoir de synth\`ese-Fractions}

\exo{"Ubung 1 : Darstellung von einem Bruch}
\begin{multicols}{4}
\setlength{\columnseprule}{1pt}
Male $\dfrac{2}{3}$ des Rechtecks aus.\\
\begin{pspicture}(0,0)(4,4)
\multido{\i=0+1}{5}{
\psline(\i,0)(\i,3)
}
\multido{\i=0+1}{4}{
\psline(0,\i)(4,\i)
}
\end{pspicture}

\columnbreak

Der Fl"acheninhalt des Quadrates ist die Einheit.\\
Bestimme den Bruch der gef"arbten Fl"ache.\\
\begin{pspicture}(-1.5,0)(4,2.5)
\multido{\i=0+2}{2}{
\psline(0,\i)(2,\i)
\psline(\i,0)(\i,2)
}
\pspolygon[fillstyle=solid,fillcolor=gray](0,0)(2,0)(1,1)
\pspolygon[fillstyle=solid,fillcolor=gray](0,0)(0,2)(1,1)
\pspolygon[fillstyle=solid,fillcolor=gray](1,1)(1,2)(2,2)
\psline(0,1)(2,1)
\psline(1,0)(1,1)
\end{pspicture}

\columnbreak

Mit welcher Zahl muss man die Zahl $7$ multiplizieren um $5$ zu ergeben ?
\columnbreak 

Der Fl"acheninhalt des Rechtecks ist die Einheit.\\
Bestimme den Bruch der gef"arbten Fl"ache.\\
\begin{pspicture}(-1,0)(4,1.5)
\psset{unit=0.6}
\pspolygon[fillstyle=solid,fillcolor=gray](0,0)(3,0)(3,2)(0,2)
\pspolygon[fillstyle=solid,fillcolor=gray](4,0)(5,0)(5,2)(4,2)
\multido{\i=0+1}{3}{
\psline(0,\i)(3,\i)
}
\multido{\i=0+1}{4}{
\psline(\i,0)(\i,2)
}
\multido{\i=0+1}{3}{
\psline(4,\i)(7,\i)
}
\multido{\i=4+1}{4}{
\psline(\i,0)(\i,2)
}
\end{pspicture}
\end{multicols}

\exo{Exercice 2 : Fraction d'une quantit\'e}\\
Calculer de trois fa\c{c}ons diff\'erentes si possible :
\begin{multicols}{2}
\begin{tabular}{llp{7cm}}
$\dfrac{3}{4}$ de $20$&$=$&\dotfill\\
&&\\
&$=$&\dotfill\\
&&\\
&$=$&\dotfill\\
\end{tabular}

\columnbreak

\begin{tabular}{llp{7cm}}
$\dfrac{7}{3}$ de $15$&$=$&\dotfill\\
&&\\
&$=$&\dotfill\\
&&\\
&$=$&\dotfill\\
\end{tabular}
\end{multicols}

\exo{"Ubung 3 : Zahlenstrahl}
\begin{multicols}{2}
\begin{enumerate}
\item Wie hei\ss en die Br"uche, die zu den folgenden Punkten geh"oren ?\\
  Gib, falls m"oglich, die gleichwertige Dezimalzahl an.
  
      \begin{DroiteGraduee}[none]{8}{0}{13}{1}{1}{0}{0}
    \AfficheTexte{0}{-0.3}{0}
\AfficheTexte{0}{0.3}{O}
\AfficheTexte{3}{0.3}{M}
\AfficheTexte{6}{0.3}{A}
\AfficheTexte{8}{-0.3}{1}
\AfficheTexte{9}{0.3}{T}
\end{DroiteGraduee}

\columnbreak
\item Setze folgende Punkte auf dem Zahlenstrahl : \\
$R(\dfrac{5}{6})$ ; $I(\dfrac{3}{2})$ ; $Z(\dfrac{2}{3})$

\begin{DroiteGraduee}[none]{8}{0}{13}{1}{1}{0}{0}
\AfficheTexte{0}{-0.3}{0}
\AfficheTexte{6}{-0.3}{1}
\end{DroiteGraduee}

\end{enumerate}
\end{multicols}

\exo{"Ubung 4 : Gleichwertige Br"uche}\\
Erg"anze die L"ucken.\\
\begin{multicols}{4}
$\dfrac{2}{3}=\dfrac{10}{\ldots}=\dfrac{\ldots}{12}$

\columnbreak

$\dfrac{15}{20}=\dfrac{3}{\ldots}=\dfrac{\ldots}{24}$
\columnbreak

$\dfrac{64}{56}=\dfrac{\ldots}{7}=\dfrac{40}{\ldots}$

\columnbreak

$0,25=\dfrac{1}{\ldots}=\dfrac{\ldots}{8}$

\end{multicols}

\exo{Exercice 5 : Probl\`emes}
\begin{multicols}{2}
\begin{enumerate}
\item J\'er\^ome a particip\'e \`a une course en montagne de $21~km$ dont $3~km$ se passe dans les vignes, une deuxi\`eme partie en for\^et, et le reste sur la cr\^ete.
\begin{enumerate}
\item Les $\dfrac{4}{7}$ de la distance ont \'et\'e parcouru en for\^et.\\ Combien cela repr\'esente-t-il de $km$ ?
\item Quelle distance a \'et\'e parcourue sur la cr\^ete?\\
\end{enumerate}

\columnbreak

\item Un aquarium est rempli d'eau ; il a une contenance de $320~l$. \\
On retire les $\dfrac{5}{8}$ de l'eau qu'il contient.\\
Ensuite on retire les $\dfrac{7}{11}$ de l'eau qui reste dans l'aquarium.
\begin{enumerate}
\item Quelle quantit\'e d'eau a-t-on retir\'e en tout ?
\item Quelle fraction de la contenance totale repr\'esente l'eau retir\'ee?
\end{enumerate}

\end{enumerate}
\end{multicols}

\end{document}