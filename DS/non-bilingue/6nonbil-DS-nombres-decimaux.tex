%%!TEX TS-program = latex
\documentclass[12pt]{article}
\usepackage[utf8]{inputenc}
\usepackage[T1]{fontenc}
\usepackage[french]{babel}
\usepackage{amsmath,amssymb}
\usepackage{geometry}
\geometry{a4paper, margin=2cm}
\usepackage{pstricks,pstricks-add,pst-plot,pst-tree,pst-eps}
\usepackage{multicol}
\usepackage{graphicx}
\usepackage{enumitem}
\usepackage{amsfonts}
\usepackage{siunitx}
\usepackage{mathrsfs}
\usepackage{pgf,tikz}
\usepackage{wasysym}

% Commandes personnalisées
\newcommand{\euro}{\text{\euro}}
\newcommand{\rep}[1]{\rule{#1cm}{0.2mm}}
\newcommand{\ladate}[1]{\hfill \textit{#1} \hfill\null}
\newcommand{\exo}[1]{\vspace{0.5cm}\textbf{#1 }}
\newcommand{\bemerkung}{\textbf{Remarque :}}

% Configuration de la page
\pagestyle{empty}
\parindent=0mm
\parskip=5mm

\begin{document}
\pagestyle{empty}

\section*{Devoir de synthèse}
\addcontentsline{toc}{section}{Devoir de synthèse-Nombres Décimaux}

\begin{multicols}{2}
\exo{Exercice 1 : Nombres décimaux}\\
Dans le nombre 135 724,689.
\begin{enumerate}
\item Quel est le chiffre des dizaines ? .........
\item Quel est le chiffre des centièmes ? .........
\item Combien y a t-il de centièmes ?\\\\ .........
\item Combien y a t-il de milliers ? .........
\end{enumerate}

\columnbreak

\exo{Exercice 2 : Le tableau de position}
\begin{enumerate}
\item Compléter le tableau suivant :\\
\renewcommand{\arraystretch}{2.5}
\begin{tabular}{| c | c | c |}
\hline
Ecriture décimale & Ecriture & Décomposition en\\
 & fractionnaire & fraction décimale\\
\hline
12,35&&\\
\hline
&$\dfrac{4~536}{1~000}$ &\\
\hline
&&$58+\dfrac{3}{10}+\dfrac{5}{100}$\\
\hline
7,05&&\\
\hline
&$\dfrac{25}{100}$&\\
\hline
\end{tabular}
\item Ranger ces nombres dans l'ordre croissant.\\\\..............................................
\end{enumerate}
\end{multicols}

[Contenu du DS suite...]

\end{document}